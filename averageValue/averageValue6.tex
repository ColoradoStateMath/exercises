\documentclass{ximera}
\usepackage{todonotes}

\usepackage{tkz-euclide}
\usetikzlibrary{backgrounds} %% for boxes around graphs
\usetikzlibrary{shapes,positioning}  %% Clouds and stars
\usetkzobj{all}
\usepackage[makeroom]{cancel} %% for strike outs
%\usepackage{mathtools} %% for pretty underbrace % Breaks Ximera
\usepackage{multicol}


\newcommand{\RR}{\mathbb R}
\renewcommand{\d}{\,d}
\newcommand{\dd}[2][]{\frac{d #1}{d #2}}
\renewcommand{\l}{\ell}
\newcommand{\ddx}{\frac{d}{dx}}
\newcommand{\zeroOverZero}{$\boldsymbol{\tfrac{0}{0}}$}
\newcommand{\numOverZero}{$\boldsymbol{\tfrac{\#}{0}}$}
\newcommand{\dfn}{\textbf}
\newcommand{\eval}[1]{\bigg[ #1 \bigg]}
\renewcommand{\epsilon}{\varepsilon}
\renewcommand{\iff}{\Leftrightarrow}

\DeclareMathOperator{\arccot}{arccot}
\DeclareMathOperator{\arcsec}{arcsec}
\DeclareMathOperator{\arccsc}{arccsc}


\colorlet{textColor}{black} 
\colorlet{background}{white}
\colorlet{penColor}{blue!50!black} % Color of a curve in a plot
\colorlet{penColor2}{red!50!black}% Color of a curve in a plot
\colorlet{penColor3}{red!50!blue} % Color of a curve in a plot
\colorlet{penColor4}{green!50!black} % Color of a curve in a plot
\colorlet{penColor5}{orange!80!black} % Color of a curve in a plot
                                      \colorlet{fill1}{blue!50!black!20} % Color of fill in a plot
\colorlet{fill2}{blue!10} % Color of fill in a plot
\colorlet{fillp}{fill1} % Color of positive area
\colorlet{filln}{red!50!black!20} % Color of negative area
\colorlet{gridColor}{gray!50} % Color of grid in a plot

\pgfmathdeclarefunction{gauss}{2}{% gives gaussian
  \pgfmathparse{1/(#2*sqrt(2*pi))*exp(-((x-#1)^2)/(2*#2^2))}%
}



\newcommand{\fullwidth}{}
\newcommand{\normalwidth}{}



%% makes a snazzy t-chart for evaluating functions
\newenvironment{tchart}{\rowcolors{2}{}{background!90!textColor}\array}{\endarray}

%%This is to help with formatting on future title pages.
\newenvironment{sectionOutcomes}{}{} 

\author{Steven Gubkin}
\license{Creative Commons 3.0 By-NC}
\begin{document}
\begin{exercise}


\tag{integral}

\begin{image}
\begin{tikzpicture}
	\begin{axis}[
            xmin=-0.2, xmax=6.2, ymin=-0.2,ymax=2.2,    
            axis lines =middle, xlabel=$x$, ylabel=$y$,
            every axis y label/.style={at=(current axis.above origin),anchor=south},
            every axis x label/.style={at=(current axis.right of origin),anchor=west},
            ticks=none,
          ]

         
          \addplot [very thick, penColor, samples=100,smooth, domain=(0:3)] {2};
          \addplot [very thick, penColor, samples=100,smooth, domain=(3:6)] {(-2/3)*(x-3)+2};


	\node at (axis cs: 4.5, 1.4) [penColor,anchor=north] {$V$};
	\node at (axis cs: 3, -0.2) [textColor,anchor=south] {$3$};
	\node at (axis cs: 6, -0.2) [textColor,anchor=south] {$6$};
         \addplot [textColor,dashed] plot coordinates {(3,0) (3,2)};

 
        \end{axis}
\end{tikzpicture}
\end{image}

Above is a graph of the velocity of an object over time.

What is the average velocity on the interval $[0,6]$
	\begin{prompt}
		The average velocity is $\answer{1.5}$
	\end{prompt}

What value of $c$ verifies the mean value theorem for integrals?  In other words, at what time $c$ does the average value of the function on $[0,6]$ equal the value of the function?
	\begin{prompt}
		At $c = \answer{3.75}$
	\end{prompt} 
\end{exercise}
\end{document}