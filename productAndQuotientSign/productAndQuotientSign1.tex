\documentclass{ximera}
\usepackage{todonotes}

\usepackage{tkz-euclide}
\usetikzlibrary{backgrounds} %% for boxes around graphs
\usetikzlibrary{shapes,positioning}  %% Clouds and stars
\usetkzobj{all}
\usepackage[makeroom]{cancel} %% for strike outs
%\usepackage{mathtools} %% for pretty underbrace % Breaks Ximera
\usepackage{multicol}


\newcommand{\RR}{\mathbb R}
\renewcommand{\d}{\,d}
\newcommand{\dd}[2][]{\frac{d #1}{d #2}}
\renewcommand{\l}{\ell}
\newcommand{\ddx}{\frac{d}{dx}}
\newcommand{\zeroOverZero}{$\boldsymbol{\tfrac{0}{0}}$}
\newcommand{\numOverZero}{$\boldsymbol{\tfrac{\#}{0}}$}
\newcommand{\dfn}{\textbf}
\newcommand{\eval}[1]{\bigg[ #1 \bigg]}
\renewcommand{\epsilon}{\varepsilon}
\renewcommand{\iff}{\Leftrightarrow}

\DeclareMathOperator{\arccot}{arccot}
\DeclareMathOperator{\arcsec}{arcsec}
\DeclareMathOperator{\arccsc}{arccsc}


\colorlet{textColor}{black} 
\colorlet{background}{white}
\colorlet{penColor}{blue!50!black} % Color of a curve in a plot
\colorlet{penColor2}{red!50!black}% Color of a curve in a plot
\colorlet{penColor3}{red!50!blue} % Color of a curve in a plot
\colorlet{penColor4}{green!50!black} % Color of a curve in a plot
\colorlet{penColor5}{orange!80!black} % Color of a curve in a plot
                                      \colorlet{fill1}{blue!50!black!20} % Color of fill in a plot
\colorlet{fill2}{blue!10} % Color of fill in a plot
\colorlet{fillp}{fill1} % Color of positive area
\colorlet{filln}{red!50!black!20} % Color of negative area
\colorlet{gridColor}{gray!50} % Color of grid in a plot

\pgfmathdeclarefunction{gauss}{2}{% gives gaussian
  \pgfmathparse{1/(#2*sqrt(2*pi))*exp(-((x-#1)^2)/(2*#2^2))}%
}



\newcommand{\fullwidth}{}
\newcommand{\normalwidth}{}



%% makes a snazzy t-chart for evaluating functions
\newenvironment{tchart}{\rowcolors{2}{}{background!90!textColor}\array}{\endarray}

%%This is to help with formatting on future title pages.
\newenvironment{sectionOutcomes}{}{} 

\author{Steven Gubkin}
\license{Creative Commons 3.0 By-NC}
\begin{document}

\begin{exercise}

\outcome{Use the first derivative to determine whether a function is increasing or decreasing.}
\outcome{Define higher order derivatives.}
\outcome{Compare differing notations for higher order derivatives.}
\outcome{Identify the relationships between the function and its first and second derivatives.}
\outcome{Sketch a graph of the second derivative, given the original function.}
\outcome{Sketch a graph of the original function, given the graph of its first and second derivatives.}
\outcome{Sketch a graph of a function satisfying certain constraints on its higher-order derivatives.}
\outcome{State the relationship between concavity and the second derivative.}
\outcome{Interpret the second derivative of a position function as acceleration.}
\outcome{Calculate higher order derivatives.}

\tag{derivative}

Let $P(x) = A(x)B(x)$

If you know that $A(2) > 0$, $A'(2) > 0$, $B(2) > 0$, and $B'(2) < 0$, what can you say about the sign of $P'(2)$?

\begin{multipleChoice}
\choice{$P'(2)>0$}
\choice{$P'(2)<0$}
\choice{$P'(2) = 0$}
\choice[correct]{We cannot determine the sign of $P'(2)$}
\end{multipleChoice}

Try to make sense of this by thinking about small increases/decreases in the quantities $A$ and $B$, in addition to symbolic calculation.

\end{exercise}

\end{document}
