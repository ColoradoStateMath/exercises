\documentclass{ximera}

\usepackage{todonotes}

\usepackage{tkz-euclide}
\usetikzlibrary{backgrounds} %% for boxes around graphs
\usetikzlibrary{shapes,positioning}  %% Clouds and stars
\usetkzobj{all}
\usepackage[makeroom]{cancel} %% for strike outs
%\usepackage{mathtools} %% for pretty underbrace % Breaks Ximera
\usepackage{multicol}


\newcommand{\RR}{\mathbb R}
\renewcommand{\d}{\,d}
\newcommand{\dd}[2][]{\frac{d #1}{d #2}}
\renewcommand{\l}{\ell}
\newcommand{\ddx}{\frac{d}{dx}}
\newcommand{\zeroOverZero}{$\boldsymbol{\tfrac{0}{0}}$}
\newcommand{\numOverZero}{$\boldsymbol{\tfrac{\#}{0}}$}
\newcommand{\dfn}{\textbf}
\newcommand{\eval}[1]{\bigg[ #1 \bigg]}
\renewcommand{\epsilon}{\varepsilon}
\renewcommand{\iff}{\Leftrightarrow}

\DeclareMathOperator{\arccot}{arccot}
\DeclareMathOperator{\arcsec}{arcsec}
\DeclareMathOperator{\arccsc}{arccsc}


\colorlet{textColor}{black} 
\colorlet{background}{white}
\colorlet{penColor}{blue!50!black} % Color of a curve in a plot
\colorlet{penColor2}{red!50!black}% Color of a curve in a plot
\colorlet{penColor3}{red!50!blue} % Color of a curve in a plot
\colorlet{penColor4}{green!50!black} % Color of a curve in a plot
\colorlet{penColor5}{orange!80!black} % Color of a curve in a plot
                                      \colorlet{fill1}{blue!50!black!20} % Color of fill in a plot
\colorlet{fill2}{blue!10} % Color of fill in a plot
\colorlet{fillp}{fill1} % Color of positive area
\colorlet{filln}{red!50!black!20} % Color of negative area
\colorlet{gridColor}{gray!50} % Color of grid in a plot

\pgfmathdeclarefunction{gauss}{2}{% gives gaussian
  \pgfmathparse{1/(#2*sqrt(2*pi))*exp(-((x-#1)^2)/(2*#2^2))}%
}



\newcommand{\fullwidth}{}
\newcommand{\normalwidth}{}



%% makes a snazzy t-chart for evaluating functions
\newenvironment{tchart}{\rowcolors{2}{}{background!90!textColor}\array}{\endarray}

%%This is to help with formatting on future title pages.
\newenvironment{sectionOutcomes}{}{} 


\author{Gregory Hartman \and Matthew Carr}
\license{Creative Commons 3.0 By-NC}
\acknowledgement{https://github.com/APEXCalculus}

\begin{document}
\begin{exercise}

\outcome{Identify the relationships between the function and its first and second derivatives.}


\tag{concavity}
\tag{derivative}

True or False? 
\begin{quote}
It is possible for a twice differentiable function to be \textbf{increasing} and \textbf{concave up} on $(0,\infty)$ with a horizontal asymptote of $y=1$.
\end{quote}

\begin{prompt}
\begin{multipleChoice}
\choice{True}
\choice[correct]{False}
\end{multipleChoice}
\end{prompt}

\begin{hint}
What would such a function look like? Consider the first and second derivative tests for a point $c>0$. What does this say about the original function?
\end{hint}
\begin{hint}
Notice that a concave upward function is such that the tangent line to any point of $f$ lies below the graph of the function $f$. 
\end{hint}
\begin{hint}
It is easily that the line tangent to $f$ at $x=x_0$ crosses the horizontal line given by $y=1$ at some $x=x_1$ depending on the tangent line's slope (i.e., the tangent line to $f$ at $x_0$ is $y=f'(x_0)(x-x_0)+f(x_0)$), so it must be the case that $f(x)\ge1$ for some $x_0\le x\le x_0$ as $f$ is increasing and $f$ lies above the every tangent line, in particular, above $y=f'(x_0)(x-x_0)+f(x_0)$.
\end{hint}
\end{exercise}
\end{document}