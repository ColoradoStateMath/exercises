\documentclass{ximera}
\usepackage{todonotes}

\usepackage{tkz-euclide}
\usetikzlibrary{backgrounds} %% for boxes around graphs
\usetikzlibrary{shapes,positioning}  %% Clouds and stars
\usetkzobj{all}
\usepackage[makeroom]{cancel} %% for strike outs
%\usepackage{mathtools} %% for pretty underbrace % Breaks Ximera
\usepackage{multicol}


\newcommand{\RR}{\mathbb R}
\renewcommand{\d}{\,d}
\newcommand{\dd}[2][]{\frac{d #1}{d #2}}
\renewcommand{\l}{\ell}
\newcommand{\ddx}{\frac{d}{dx}}
\newcommand{\zeroOverZero}{$\boldsymbol{\tfrac{0}{0}}$}
\newcommand{\numOverZero}{$\boldsymbol{\tfrac{\#}{0}}$}
\newcommand{\dfn}{\textbf}
\newcommand{\eval}[1]{\bigg[ #1 \bigg]}
\renewcommand{\epsilon}{\varepsilon}
\renewcommand{\iff}{\Leftrightarrow}

\DeclareMathOperator{\arccot}{arccot}
\DeclareMathOperator{\arcsec}{arcsec}
\DeclareMathOperator{\arccsc}{arccsc}


\colorlet{textColor}{black} 
\colorlet{background}{white}
\colorlet{penColor}{blue!50!black} % Color of a curve in a plot
\colorlet{penColor2}{red!50!black}% Color of a curve in a plot
\colorlet{penColor3}{red!50!blue} % Color of a curve in a plot
\colorlet{penColor4}{green!50!black} % Color of a curve in a plot
\colorlet{penColor5}{orange!80!black} % Color of a curve in a plot
                                      \colorlet{fill1}{blue!50!black!20} % Color of fill in a plot
\colorlet{fill2}{blue!10} % Color of fill in a plot
\colorlet{fillp}{fill1} % Color of positive area
\colorlet{filln}{red!50!black!20} % Color of negative area
\colorlet{gridColor}{gray!50} % Color of grid in a plot

\pgfmathdeclarefunction{gauss}{2}{% gives gaussian
  \pgfmathparse{1/(#2*sqrt(2*pi))*exp(-((x-#1)^2)/(2*#2^2))}%
}



\newcommand{\fullwidth}{}
\newcommand{\normalwidth}{}



%% makes a snazzy t-chart for evaluating functions
\newenvironment{tchart}{\rowcolors{2}{}{background!90!textColor}\array}{\endarray}

%%This is to help with formatting on future title pages.
\newenvironment{sectionOutcomes}{}{} 

\author{Steven Gubkin}
\license{Creative Commons 3.0 By-NC}
\begin{document}
\begin{exercise}

\outcome{Recall how to find limits for forms that are not indeterminate.}
\outcome{Define an indeterminate form.}
\outcome{Determine if a form is indeterminate.}
\outcome{Convert indeterminate forms to the form zero over zero or infinity over infinity.}
\outcome{Define L’hopital’s Rule and identify when it can be used.}
\outcome{Use L’hopital’s Rule to find limits.}

\tag{derivative}

Decide whether l'H\^{o}pital's rule immediatly applies to the following limit.  If it does not, explain why not.  Find the limit by any means necessary, or state that it does not exist. 

\[
\lim_{x \to 0} \frac{\sin^2(x)\sin(\frac{1}{x})}{x\sin(\frac{1}{2x})}
\]

\begin{prompt}
	Call the numerator $f(x)$ and the denominator $g(x)$. 

	\begin{multipleChoice}
	\choice{l'H\^{o}pital's rule applies}
	\choice{l'H\^{o}pital's rule does not apply since the limit is not an indeterminate form }
	\choice[correct]{l'H\^{o}pital's rule does not apply $g'(x)$ has zeros on every interval $(a, \infty)$ }
	\choice{l'H\^{o}pital's rule does not apply because $\lim_{x \to \infty} \frac{f'(x)}{g'(x)}$ does not exist.}
\end{multipleChoice}

If the limit does not exist, write $DNE$.

\[
\lim_{x \to 0} \frac{\sin^2(x)\sin(\frac{1}{x})}{x\sin(\frac{1}{2x})} = \answer{0}
\]

\end{prompt}

\end{exercise}
\end{document}