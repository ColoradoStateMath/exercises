\documentclass{ximera}
\usepackage{todonotes}

\usepackage{tkz-euclide}
\usetikzlibrary{backgrounds} %% for boxes around graphs
\usetikzlibrary{shapes,positioning}  %% Clouds and stars
\usetkzobj{all}
\usepackage[makeroom]{cancel} %% for strike outs
%\usepackage{mathtools} %% for pretty underbrace % Breaks Ximera
\usepackage{multicol}


\newcommand{\RR}{\mathbb R}
\renewcommand{\d}{\,d}
\newcommand{\dd}[2][]{\frac{d #1}{d #2}}
\renewcommand{\l}{\ell}
\newcommand{\ddx}{\frac{d}{dx}}
\newcommand{\zeroOverZero}{$\boldsymbol{\tfrac{0}{0}}$}
\newcommand{\numOverZero}{$\boldsymbol{\tfrac{\#}{0}}$}
\newcommand{\dfn}{\textbf}
\newcommand{\eval}[1]{\bigg[ #1 \bigg]}
\renewcommand{\epsilon}{\varepsilon}
\renewcommand{\iff}{\Leftrightarrow}

\DeclareMathOperator{\arccot}{arccot}
\DeclareMathOperator{\arcsec}{arcsec}
\DeclareMathOperator{\arccsc}{arccsc}


\colorlet{textColor}{black} 
\colorlet{background}{white}
\colorlet{penColor}{blue!50!black} % Color of a curve in a plot
\colorlet{penColor2}{red!50!black}% Color of a curve in a plot
\colorlet{penColor3}{red!50!blue} % Color of a curve in a plot
\colorlet{penColor4}{green!50!black} % Color of a curve in a plot
\colorlet{penColor5}{orange!80!black} % Color of a curve in a plot
                                      \colorlet{fill1}{blue!50!black!20} % Color of fill in a plot
\colorlet{fill2}{blue!10} % Color of fill in a plot
\colorlet{fillp}{fill1} % Color of positive area
\colorlet{filln}{red!50!black!20} % Color of negative area
\colorlet{gridColor}{gray!50} % Color of grid in a plot

\pgfmathdeclarefunction{gauss}{2}{% gives gaussian
  \pgfmathparse{1/(#2*sqrt(2*pi))*exp(-((x-#1)^2)/(2*#2^2))}%
}



\newcommand{\fullwidth}{}
\newcommand{\normalwidth}{}



%% makes a snazzy t-chart for evaluating functions
\newenvironment{tchart}{\rowcolors{2}{}{background!90!textColor}\array}{\endarray}

%%This is to help with formatting on future title pages.
\newenvironment{sectionOutcomes}{}{} 

\author{Steven Gubkin}
\license{Creative Commons 3.0 By-NC}
\begin{document}
\begin{exercise}

\outcome{Identify cross sections.}
\outcome{Compute volumes of shapes with arbitrary cross sections}
\outcome{Use the disk method to compute volumes.}
%\outcome{Use the washer method to compute volumes.}
%\outcome{Compute volumes of revolution around the $x$-axis.}
\outcome{Compute volumes of revolution around the $y$-axis.}
%\outcome{Compute volumes of revolution around arbitrary lines.}

\tag{integral}

Consider the region bounded by $y =\ln(x)$ , the $x$ axis, the $y$ axis, and the horizontal line $y=1$.  What is the volume of the solid obtained by revolving this region about the $y$ axis?


\begin{hint}
	Draw a picture!
\end{hint}

\begin{hint}
	Solving for $x$, we have $x = e^y$.  
\end{hint}

\begin{hint}
	We can decompose the solid into infinitesmal disks with width $dy$ and radius $e^y$.  The volume of each washer is $\pi (e^y)^2\d y$.  Summing these volumes from $y=0$ to $y=1$, we obtain

	\[
	\textrm{Volume} = \int_0^1 \pi (e^y)^2\d y
	\]
\end{hint}

\begin{hint}
	\begin{align*}
		 \int_0^1 \pi (e^y)^2\d y &= \pi \int_0^1e^{2y}\d y \\
			&=\frac{\pi}{2}\eval{e^{2y}}_0^1\\
			&=\frac{pi}{2}(e^2-e)
	\end{align*}
\end{hint}

\begin{prompt}
	\[
		\textrm{Volume} = \answer{\frac{pi}{2}(e^2-e)}
	\]
\end{prompt}

\end{exercise}
\end{document}