\documentclass{ximera}
\usepackage{todonotes}

\usepackage{tkz-euclide}
\usetikzlibrary{backgrounds} %% for boxes around graphs
\usetikzlibrary{shapes,positioning}  %% Clouds and stars
\usetkzobj{all}
\usepackage[makeroom]{cancel} %% for strike outs
%\usepackage{mathtools} %% for pretty underbrace % Breaks Ximera
\usepackage{multicol}


\newcommand{\RR}{\mathbb R}
\renewcommand{\d}{\,d}
\newcommand{\dd}[2][]{\frac{d #1}{d #2}}
\renewcommand{\l}{\ell}
\newcommand{\ddx}{\frac{d}{dx}}
\newcommand{\zeroOverZero}{$\boldsymbol{\tfrac{0}{0}}$}
\newcommand{\numOverZero}{$\boldsymbol{\tfrac{\#}{0}}$}
\newcommand{\dfn}{\textbf}
\newcommand{\eval}[1]{\bigg[ #1 \bigg]}
\renewcommand{\epsilon}{\varepsilon}
\renewcommand{\iff}{\Leftrightarrow}

\DeclareMathOperator{\arccot}{arccot}
\DeclareMathOperator{\arcsec}{arcsec}
\DeclareMathOperator{\arccsc}{arccsc}


\colorlet{textColor}{black} 
\colorlet{background}{white}
\colorlet{penColor}{blue!50!black} % Color of a curve in a plot
\colorlet{penColor2}{red!50!black}% Color of a curve in a plot
\colorlet{penColor3}{red!50!blue} % Color of a curve in a plot
\colorlet{penColor4}{green!50!black} % Color of a curve in a plot
\colorlet{penColor5}{orange!80!black} % Color of a curve in a plot
                                      \colorlet{fill1}{blue!50!black!20} % Color of fill in a plot
\colorlet{fill2}{blue!10} % Color of fill in a plot
\colorlet{fillp}{fill1} % Color of positive area
\colorlet{filln}{red!50!black!20} % Color of negative area
\colorlet{gridColor}{gray!50} % Color of grid in a plot

\pgfmathdeclarefunction{gauss}{2}{% gives gaussian
  \pgfmathparse{1/(#2*sqrt(2*pi))*exp(-((x-#1)^2)/(2*#2^2))}%
}



\newcommand{\fullwidth}{}
\newcommand{\normalwidth}{}



%% makes a snazzy t-chart for evaluating functions
\newenvironment{tchart}{\rowcolors{2}{}{background!90!textColor}\array}{\endarray}

%%This is to help with formatting on future title pages.
\newenvironment{sectionOutcomes}{}{} 

\author{Bart Snapp}
\license{Creative Commons 3.0 By-NC}
\begin{document}
\begin{exercise}

\outcome{Understand the Squeeze Theorem and how it can be used to find limit values.}
\outcome{Calculate limits using the Squeeze Theorem.}
\tag{limit}
\tag{squeeze theorem}

Consider:
\[
\lim_{x\to 0} \left(x 2^{\tan ^{-1}\left(\frac{\pi }{x}\right)}\right)
\]
A good way to compute this limit would be to use \wordChoice{\choice{limit laws}\choice{indeterminate forms}\choice[correct]{the Squeeze Theorem}\choice{the Intermediate Value Theorem}}.
\begin{exercise}
List two functions $g$ and $h$ such that
\[
g(x)\le x 2^{\tan ^{-1}\left(\frac{\pi }{x}\right)} \le h(x)
\]
for all $x$ on some interval containing $x=0$.
\[
g(x)=\answer{2^{-\pi /2} \left| x\right|}\qquad h(x) =\answer{2^{\pi /2} \left| x\right|}
\]
\begin{exercise}
Compute:
\[
\lim_{x \to 0}2^{-\pi /2} \left| x\right| = \answer{0}\qquad \lim_{x\to 0}2^{\pi /2} \left| x\right| = \answer{0}
\]
\begin{exercise}
By the Squeeze Theorem:
\[
\lim_{x\to 0} \left(x 2^{\tan ^{-1}\left(\frac{\pi }{x}\right)}\right) = \answer{0}
\]
\end{exercise}
\end{exercise}
\end{exercise}
\end{exercise}
\end{document}