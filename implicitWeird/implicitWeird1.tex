\documentclass{ximera}
\usepackage{todonotes}

\usepackage{tkz-euclide}
\usetikzlibrary{backgrounds} %% for boxes around graphs
\usetikzlibrary{shapes,positioning}  %% Clouds and stars
\usetkzobj{all}
\usepackage[makeroom]{cancel} %% for strike outs
%\usepackage{mathtools} %% for pretty underbrace % Breaks Ximera
\usepackage{multicol}


\newcommand{\RR}{\mathbb R}
\renewcommand{\d}{\,d}
\newcommand{\dd}[2][]{\frac{d #1}{d #2}}
\renewcommand{\l}{\ell}
\newcommand{\ddx}{\frac{d}{dx}}
\newcommand{\zeroOverZero}{$\boldsymbol{\tfrac{0}{0}}$}
\newcommand{\numOverZero}{$\boldsymbol{\tfrac{\#}{0}}$}
\newcommand{\dfn}{\textbf}
\newcommand{\eval}[1]{\bigg[ #1 \bigg]}
\renewcommand{\epsilon}{\varepsilon}
\renewcommand{\iff}{\Leftrightarrow}

\DeclareMathOperator{\arccot}{arccot}
\DeclareMathOperator{\arcsec}{arcsec}
\DeclareMathOperator{\arccsc}{arccsc}


\colorlet{textColor}{black} 
\colorlet{background}{white}
\colorlet{penColor}{blue!50!black} % Color of a curve in a plot
\colorlet{penColor2}{red!50!black}% Color of a curve in a plot
\colorlet{penColor3}{red!50!blue} % Color of a curve in a plot
\colorlet{penColor4}{green!50!black} % Color of a curve in a plot
\colorlet{penColor5}{orange!80!black} % Color of a curve in a plot
                                      \colorlet{fill1}{blue!50!black!20} % Color of fill in a plot
\colorlet{fill2}{blue!10} % Color of fill in a plot
\colorlet{fillp}{fill1} % Color of positive area
\colorlet{filln}{red!50!black!20} % Color of negative area
\colorlet{gridColor}{gray!50} % Color of grid in a plot

\pgfmathdeclarefunction{gauss}{2}{% gives gaussian
  \pgfmathparse{1/(#2*sqrt(2*pi))*exp(-((x-#1)^2)/(2*#2^2))}%
}



\newcommand{\fullwidth}{}
\newcommand{\normalwidth}{}



%% makes a snazzy t-chart for evaluating functions
\newenvironment{tchart}{\rowcolors{2}{}{background!90!textColor}\array}{\endarray}

%%This is to help with formatting on future title pages.
\newenvironment{sectionOutcomes}{}{} 

\author{Steven Gubkin}
\license{Creative Commons 3.0 By-NC}

\begin{document}
\begin{exercise}

\outcome{Implicitly differentiate expressions}
\outcome{Find the equation of the tangent line for curves that are not plots of functions}
\outcome{Understand how changing the variable changes how we take the derivative}
\outcome{Understand the derivatives of expressions that are not functions or not solved for y}
%\outcome{Use implicit differentiation to demonstrate the power rule for rational exponents}

Consider the curve  $y^2 = x^2(x+1)$, whose graph is given below:

\begin{image}
\begin{tikzpicture}
	\begin{axis}[
            xmin=-2,xmax=2,ymin=-2,ymax=2,
            axis lines=center,
	   ticks=none,
            xlabel=$x$, ylabel=$y$,
            every axis y label/.style={at=(current axis.above origin),anchor=south},
            every axis x label/.style={at=(current axis.right of origin),anchor=west},
          ]        
          \addplot [very thick, penColor, smooth, samples=100, domain=(-2:2)] ({x^2-1},{x^3-x});
        \end{axis}
\end{tikzpicture}
\end{image}

Use implicit differentiation to write $\frac{dy}{dx}$ as a function of both $x$ and $y$.

\[
\frac{dy}{dx} = \answer{\frac{3x^2+2x}{2y}}
\]

The equation of the tangent line to this curve at the point $(3,6)$ is

\[
y = \answer{\frac{11}{4}(x-3)+6}
\]

A difficult point to analyze on this graph is the point $(0,0)$, since it looks like two intersecting lines there.  We would like to know the equation of these two "tangent lines".  Our formula for $\frac{dy}{dx}$ provides no help, initially, since we get $\frac{0}{0}$ when we plug $(0,0)$ into that formula.

Since the formula only breaks down at the point $(0,0)$, a way forward is to try to take a limit of slopes as $(x,y)$ approaches $(0,0)$ along the curve.

We can solve for the "top part" of the curve as $y = \sqrt{x^2(x+1)}$.  The other half of the curve is $y  = -\sqrt{x^2(x+1)}$.

Using $y = \sqrt{x^2(x+1)}$, we see that for points in the first quadrant of the plane, we can write $\frac{dy}{dx}$ as a function of $x$ as

\[
\frac{dy}{dx} = \answer{\frac{3x^2+2x}{2\sqrt{x^2(x+1)}}}
\]

and

\[
\lim_{x \to 0^+} \frac{dy}{dx} = 1
\]

Thus equations of the two "tangent lines" are $y=x$ and $y=-x$.

Here is another way we could approach the problem:  by parameterizing the curve.  If we could find a way to describe the motion of a particle along the curve, maybe the bad point wouldn't seem so bad from the point of view of the particle. See \href{https://www.desmos.com/calculator/py0camtwlg}{this interactive graph} to get some ideas about this approach.  If your curiosity compells you, try to figure out $\frac{dy}{dt}$, $\frac{dx}{dt}$, and the two times that the particle intersects $(0,0)$, and use these to confirm that $\frac{dy}{dx} = \pm 1$ at these points.

We could have also arrived at this result more intuitively in the following way:

In the equation $y^2 = x^2+x^3$, when $x$ is very close to $0$, all of the terms will be very small, but the term $x^3$ will be small even compared to $x^2$.  So the curve should look a lot like the curve $y^2 = x^2$ close to $(0,0)$.  But $y^2 = x^2$ is equivalent to $(y-x)(y+x)=0$, which is exactly the pair of lines $y=x$ and $y=-x$.
\end{exercise}
\end{document}