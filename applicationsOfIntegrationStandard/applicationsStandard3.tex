\documentclass{ximera}
\usepackage{todonotes}

\usepackage{tkz-euclide}
\usetikzlibrary{backgrounds} %% for boxes around graphs
\usetikzlibrary{shapes,positioning}  %% Clouds and stars
\usetkzobj{all}
\usepackage[makeroom]{cancel} %% for strike outs
%\usepackage{mathtools} %% for pretty underbrace % Breaks Ximera
\usepackage{multicol}


\newcommand{\RR}{\mathbb R}
\renewcommand{\d}{\,d}
\newcommand{\dd}[2][]{\frac{d #1}{d #2}}
\renewcommand{\l}{\ell}
\newcommand{\ddx}{\frac{d}{dx}}
\newcommand{\zeroOverZero}{$\boldsymbol{\tfrac{0}{0}}$}
\newcommand{\numOverZero}{$\boldsymbol{\tfrac{\#}{0}}$}
\newcommand{\dfn}{\textbf}
\newcommand{\eval}[1]{\bigg[ #1 \bigg]}
\renewcommand{\epsilon}{\varepsilon}
\renewcommand{\iff}{\Leftrightarrow}

\DeclareMathOperator{\arccot}{arccot}
\DeclareMathOperator{\arcsec}{arcsec}
\DeclareMathOperator{\arccsc}{arccsc}


\colorlet{textColor}{black} 
\colorlet{background}{white}
\colorlet{penColor}{blue!50!black} % Color of a curve in a plot
\colorlet{penColor2}{red!50!black}% Color of a curve in a plot
\colorlet{penColor3}{red!50!blue} % Color of a curve in a plot
\colorlet{penColor4}{green!50!black} % Color of a curve in a plot
\colorlet{penColor5}{orange!80!black} % Color of a curve in a plot
                                      \colorlet{fill1}{blue!50!black!20} % Color of fill in a plot
\colorlet{fill2}{blue!10} % Color of fill in a plot
\colorlet{fillp}{fill1} % Color of positive area
\colorlet{filln}{red!50!black!20} % Color of negative area
\colorlet{gridColor}{gray!50} % Color of grid in a plot

\pgfmathdeclarefunction{gauss}{2}{% gives gaussian
  \pgfmathparse{1/(#2*sqrt(2*pi))*exp(-((x-#1)^2)/(2*#2^2))}%
}



\newcommand{\fullwidth}{}
\newcommand{\normalwidth}{}



%% makes a snazzy t-chart for evaluating functions
\newenvironment{tchart}{\rowcolors{2}{}{background!90!textColor}\array}{\endarray}

%%This is to help with formatting on future title pages.
\newenvironment{sectionOutcomes}{}{} 

\author{Steven Gubkin}
\license{Creative Commons 3.0 By-NC}
\begin{document}

\begin{exercise}

\tag{integral}

When $20 \unit{lb}$ of force are placed on a spring, it is $1 \unit{ft}$ long.  When $60 \unit{lb}$ are applied, the spring stretches to $2 \unit{ft}$ long.

What is the natural length of the spring?

How much work is done, in $\unit{ft - lb}$, to stretch the spring from $1 \unit{ft}$ to $3 \unit{ft}$? 

\begin{hint}
Let $L$ be the natural length of the spring, and $C$ be the spring constant.  

Then, converting all of the units to feet, we know

\[
\begin{cases}
20 \unit{lb} = C(1\unit{ft} - L)\\
60 \unit{lb}= C(2 \unit{ft}- L)
\end{cases}
\]

Solving this system of equations, we obtain

\[
\begin{cases}
C = 40 \unit{lb}/\unit{ft}\\
L = \frac{1}{2} \unit{ft}
\end{cases}
\]
\end{hint}

\begin{hint}
	Letting $x$ be the distance that the spring is stretched beyond its natural length, we thus have that the force acting on the spring is $F(x) = 40x$.  The work done by stretching the spring an infinitesimal amount $\d x$ is $\d W = 40x\d x$.  Note that when the spring is $1 \unit{ft}$, $x = \frac{1}{2}$, and when the spring is $3 \unit{ft}$, $x$ is $\frac{5}{2}$.
\end{hint}

\begin{hint}
Thus the work done by stretching the spring from its natural length to $1 \unit{ft}$ to $3 \unit{ft}$ is given by

\begin{align*}
\int_{\frac{1}{2}}^{\frac{5}{2}} 40 x \d x &= \eval{20x^2}_{\frac{1}{2}}^\frac{5}{2}\\
	&=20(\frac{25}{4} - \frac{1}{4})\\
	&=120
\end{align*}
\end{hint}

\begin{prompt}
	\[
	\textrm{Work} = \answer{120} \unit{ft-lb}
	\]
\end{prompt}

\end{exercise}
\end{document}

