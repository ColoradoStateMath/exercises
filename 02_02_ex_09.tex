\documentclass{ximera}

\usepackage{todonotes}

\usepackage{tkz-euclide}
\usetikzlibrary{backgrounds} %% for boxes around graphs
\usetikzlibrary{shapes,positioning}  %% Clouds and stars
\usetkzobj{all}
\usepackage[makeroom]{cancel} %% for strike outs
%\usepackage{mathtools} %% for pretty underbrace % Breaks Ximera
\usepackage{multicol}


\newcommand{\RR}{\mathbb R}
\renewcommand{\d}{\,d}
\newcommand{\dd}[2][]{\frac{d #1}{d #2}}
\renewcommand{\l}{\ell}
\newcommand{\ddx}{\frac{d}{dx}}
\newcommand{\zeroOverZero}{$\boldsymbol{\tfrac{0}{0}}$}
\newcommand{\numOverZero}{$\boldsymbol{\tfrac{\#}{0}}$}
\newcommand{\dfn}{\textbf}
\newcommand{\eval}[1]{\bigg[ #1 \bigg]}
\renewcommand{\epsilon}{\varepsilon}
\renewcommand{\iff}{\Leftrightarrow}

\DeclareMathOperator{\arccot}{arccot}
\DeclareMathOperator{\arcsec}{arcsec}
\DeclareMathOperator{\arccsc}{arccsc}


\colorlet{textColor}{black} 
\colorlet{background}{white}
\colorlet{penColor}{blue!50!black} % Color of a curve in a plot
\colorlet{penColor2}{red!50!black}% Color of a curve in a plot
\colorlet{penColor3}{red!50!blue} % Color of a curve in a plot
\colorlet{penColor4}{green!50!black} % Color of a curve in a plot
\colorlet{penColor5}{orange!80!black} % Color of a curve in a plot
                                      \colorlet{fill1}{blue!50!black!20} % Color of fill in a plot
\colorlet{fill2}{blue!10} % Color of fill in a plot
\colorlet{fillp}{fill1} % Color of positive area
\colorlet{filln}{red!50!black!20} % Color of negative area
\colorlet{gridColor}{gray!50} % Color of grid in a plot

\pgfmathdeclarefunction{gauss}{2}{% gives gaussian
  \pgfmathparse{1/(#2*sqrt(2*pi))*exp(-((x-#1)^2)/(2*#2^2))}%
}



\newcommand{\fullwidth}{}
\newcommand{\normalwidth}{}



%% makes a snazzy t-chart for evaluating functions
\newenvironment{tchart}{\rowcolors{2}{}{background!90!textColor}\array}{\endarray}

%%This is to help with formatting on future title pages.
\newenvironment{sectionOutcomes}{}{} 


\author{Gregory Hartman \and Matthew Carr}
\license{Creative Commons 3.0 By-NC}
\acknowledgement{https://github.com/APEXCalculus}

\begin{document}
\begin{exercise}

\tag{derivative}
\tag{linear approximation}
\tag{approximation}


\outcome{Understand the derivative as a function related to the original function.}
\outcome{Find the linear approximation to a function at a point and use it to approximate the function value.}


Given $H(0)=17$ and $H(2)=29$, approximate $H'(2)$ by linear approximation.
\[
H'(2)
\begin{prompt}
\approx \answer{6}
\end{prompt}
\]

\begin{hint}
We are given that $H(2)=29$ and $H(0)=17$. Recall that the line tangent to $H$ at $x=x_0$ is given by $y=H(x_0)+H'(x_0)(x-x_0)$. This is the linear approximation to $H$ at $x=x_0$. With this, we have an approximation $H(x)\approx H(x_0)+H'(x_0)(x-x_0)$ for each $x$. How might you choose $x$ and $x_0$ to relate what you know about $H(0)$ and $H(2)$ to $H'(2)$?
\end{hint}
\begin{hint}
The linear approximation for $H$ at $x=2$ is $y=H(2)+H'(2)(x-2)$. We can approximate the point $H(0)$ by letting the $x$ in $H(2)+H'(2)(x-2)$ be $0$, then $H(0)\approx H(2)-2H'(2)$. Hence, $H'(2)\approx\frac{1}{2}\left(H(2)-H(0)\right)$ and we know that $H(2)-H(0)=29-17=12$. Dividing by $2$, we obtain $H'(2)\approx6$.
\end{hint}
\end{exercise}
\end{document}