\documentclass{ximera}
\usepackage{todonotes}

\usepackage{tkz-euclide}
\usetikzlibrary{backgrounds} %% for boxes around graphs
\usetikzlibrary{shapes,positioning}  %% Clouds and stars
\usetkzobj{all}
\usepackage[makeroom]{cancel} %% for strike outs
%\usepackage{mathtools} %% for pretty underbrace % Breaks Ximera
\usepackage{multicol}


\newcommand{\RR}{\mathbb R}
\renewcommand{\d}{\,d}
\newcommand{\dd}[2][]{\frac{d #1}{d #2}}
\renewcommand{\l}{\ell}
\newcommand{\ddx}{\frac{d}{dx}}
\newcommand{\zeroOverZero}{$\boldsymbol{\tfrac{0}{0}}$}
\newcommand{\numOverZero}{$\boldsymbol{\tfrac{\#}{0}}$}
\newcommand{\dfn}{\textbf}
\newcommand{\eval}[1]{\bigg[ #1 \bigg]}
\renewcommand{\epsilon}{\varepsilon}
\renewcommand{\iff}{\Leftrightarrow}

\DeclareMathOperator{\arccot}{arccot}
\DeclareMathOperator{\arcsec}{arcsec}
\DeclareMathOperator{\arccsc}{arccsc}


\colorlet{textColor}{black} 
\colorlet{background}{white}
\colorlet{penColor}{blue!50!black} % Color of a curve in a plot
\colorlet{penColor2}{red!50!black}% Color of a curve in a plot
\colorlet{penColor3}{red!50!blue} % Color of a curve in a plot
\colorlet{penColor4}{green!50!black} % Color of a curve in a plot
\colorlet{penColor5}{orange!80!black} % Color of a curve in a plot
                                      \colorlet{fill1}{blue!50!black!20} % Color of fill in a plot
\colorlet{fill2}{blue!10} % Color of fill in a plot
\colorlet{fillp}{fill1} % Color of positive area
\colorlet{filln}{red!50!black!20} % Color of negative area
\colorlet{gridColor}{gray!50} % Color of grid in a plot

\pgfmathdeclarefunction{gauss}{2}{% gives gaussian
  \pgfmathparse{1/(#2*sqrt(2*pi))*exp(-((x-#1)^2)/(2*#2^2))}%
}



\newcommand{\fullwidth}{}
\newcommand{\normalwidth}{}



%% makes a snazzy t-chart for evaluating functions
\newenvironment{tchart}{\rowcolors{2}{}{background!90!textColor}\array}{\endarray}

%%This is to help with formatting on future title pages.
\newenvironment{sectionOutcomes}{}{} 

\author{Steven Gubkin}
\license{Creative Commons 3.0 By-NC}
\begin{document}
\begin{exercise}

\tag{integral}

The \textbf{Error function} is defined by

\[
\mathrm{Erf}(x) = \frac{2}{\sqrt{\pi}} \int_0^x e^{-t^2} \d t
\]

This function is very important in statistics, but cannot be expressed as an elemenary function (you cannot write it down in terms of rational functions, trig functions, exponentials, etc without an integral).

\[
\begin{array}{c|c}
 x & \mathrm{Erf}(x)\\ \hline
0.1 & 0.11246\\
0.2 & 0.22270\\
0.3 & 0.32863\\
0.4 & 0.42839\\
0.5 & 0.52050\\
0.6 & 0.60386\\
0.7 & 0.67780\\
0.8 & 0.74210\\
0.9 & 0.79691\\
1.0 & 0.84270\\
\end{array}
\]

Using this table of values, evaluate the following definite integrals:

\begin{exercise}
	\[
	\frac{1}{\sqrt{\pi} }\int_\frac{1}{10}^\frac{1}{2} e^{-4t^2} \d t = \answer{0.155}
	\]
\end{exercise}

\begin{exercise}
	\[
	\frac{1}{\sqrt{\pi}} \int_{-0.4}^{0.4} e^{-t^2} \d t = \answer{0.42839}
	\]
\end{exercise}

\begin{exercise}
	\[
	 \int_{0.1}^{0.6} e^{-(t-0.2)^2} \d t = \answer{0.4793}
	\]
\end{exercise}

\end{exercise}
\end{document}