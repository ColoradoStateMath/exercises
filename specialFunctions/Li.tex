\documentclass{ximera}
\usepackage{todonotes}

\usepackage{tkz-euclide}
\usetikzlibrary{backgrounds} %% for boxes around graphs
\usetikzlibrary{shapes,positioning}  %% Clouds and stars
\usetkzobj{all}
\usepackage[makeroom]{cancel} %% for strike outs
%\usepackage{mathtools} %% for pretty underbrace % Breaks Ximera
\usepackage{multicol}


\newcommand{\RR}{\mathbb R}
\renewcommand{\d}{\,d}
\newcommand{\dd}[2][]{\frac{d #1}{d #2}}
\renewcommand{\l}{\ell}
\newcommand{\ddx}{\frac{d}{dx}}
\newcommand{\zeroOverZero}{$\boldsymbol{\tfrac{0}{0}}$}
\newcommand{\numOverZero}{$\boldsymbol{\tfrac{\#}{0}}$}
\newcommand{\dfn}{\textbf}
\newcommand{\eval}[1]{\bigg[ #1 \bigg]}
\renewcommand{\epsilon}{\varepsilon}
\renewcommand{\iff}{\Leftrightarrow}

\DeclareMathOperator{\arccot}{arccot}
\DeclareMathOperator{\arcsec}{arcsec}
\DeclareMathOperator{\arccsc}{arccsc}


\colorlet{textColor}{black} 
\colorlet{background}{white}
\colorlet{penColor}{blue!50!black} % Color of a curve in a plot
\colorlet{penColor2}{red!50!black}% Color of a curve in a plot
\colorlet{penColor3}{red!50!blue} % Color of a curve in a plot
\colorlet{penColor4}{green!50!black} % Color of a curve in a plot
\colorlet{penColor5}{orange!80!black} % Color of a curve in a plot
                                      \colorlet{fill1}{blue!50!black!20} % Color of fill in a plot
\colorlet{fill2}{blue!10} % Color of fill in a plot
\colorlet{fillp}{fill1} % Color of positive area
\colorlet{filln}{red!50!black!20} % Color of negative area
\colorlet{gridColor}{gray!50} % Color of grid in a plot

\pgfmathdeclarefunction{gauss}{2}{% gives gaussian
  \pgfmathparse{1/(#2*sqrt(2*pi))*exp(-((x-#1)^2)/(2*#2^2))}%
}



\newcommand{\fullwidth}{}
\newcommand{\normalwidth}{}



%% makes a snazzy t-chart for evaluating functions
\newenvironment{tchart}{\rowcolors{2}{}{background!90!textColor}\array}{\endarray}

%%This is to help with formatting on future title pages.
\newenvironment{sectionOutcomes}{}{} 

\author{Steven Gubkin}
\license{Creative Commons 3.0 By-NC}
\begin{document}
\begin{exercise}

\tag{integral}

The \textbf{Offset Logarithmic Intergral} function is defined by

\[
\mathrm{Li}(x) = \int_2^x \frac{1)}{\ln(t)} \d t
\]

This function is very important in Analytic Number Theory.  It turns out that $Li(x)$ is a very good approximation for the number of prime numbers less than $x$. However, it cannot be expressed as an elemenary function (you cannot write it down in terms of rational functions, trig functions, exponentials, etc without an integral).

\begin{exercise}
	What is $\lim_{x \to \infty} Li(x)$?  Explain how you know.
	
	\begin{prompt}
	\[
	\lim_{x \to \infty} Li(x) = \answer{\infty}
	\]
	\end{prompt}
\end{exercise}


Here is a table of values for $Si(x)$:

\[
\begin{array}{c|c}
 x & \mathrm{Li}(x)\\ \hline
3 &  1.118\\
4 & 1.922\\
5 & 2.589\\
6 & 3.177\\
7& 3.712\\
8 & 4.209\\
9 & 4.676
\end{array}
\]


Using this table of values, evaluate the following definite integrals:

\begin{exercise}
	\[
	\int_2^4 \frac{1}{\ln(3+t)} \d t \begin{prompt} = \answer{1.123} \end{prompt} 
	\]
\end{exercise}

\begin{exercise}
	\[
	\int_2^3 \frac{t}{\ln(t)} \d t \begin{prompt} = \answer{2.754} \end{prompt} 
	\]
\end{exercise}

\end{exercise}
\end{document}