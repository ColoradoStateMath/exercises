\documentclass{ximera}

\usepackage{todonotes}

\usepackage{tkz-euclide}
\usetikzlibrary{backgrounds} %% for boxes around graphs
\usetikzlibrary{shapes,positioning}  %% Clouds and stars
\usetkzobj{all}
\usepackage[makeroom]{cancel} %% for strike outs
%\usepackage{mathtools} %% for pretty underbrace % Breaks Ximera
\usepackage{multicol}


\newcommand{\RR}{\mathbb R}
\renewcommand{\d}{\,d}
\newcommand{\dd}[2][]{\frac{d #1}{d #2}}
\renewcommand{\l}{\ell}
\newcommand{\ddx}{\frac{d}{dx}}
\newcommand{\zeroOverZero}{$\boldsymbol{\tfrac{0}{0}}$}
\newcommand{\numOverZero}{$\boldsymbol{\tfrac{\#}{0}}$}
\newcommand{\dfn}{\textbf}
\newcommand{\eval}[1]{\bigg[ #1 \bigg]}
\renewcommand{\epsilon}{\varepsilon}
\renewcommand{\iff}{\Leftrightarrow}

\DeclareMathOperator{\arccot}{arccot}
\DeclareMathOperator{\arcsec}{arcsec}
\DeclareMathOperator{\arccsc}{arccsc}


\colorlet{textColor}{black} 
\colorlet{background}{white}
\colorlet{penColor}{blue!50!black} % Color of a curve in a plot
\colorlet{penColor2}{red!50!black}% Color of a curve in a plot
\colorlet{penColor3}{red!50!blue} % Color of a curve in a plot
\colorlet{penColor4}{green!50!black} % Color of a curve in a plot
\colorlet{penColor5}{orange!80!black} % Color of a curve in a plot
                                      \colorlet{fill1}{blue!50!black!20} % Color of fill in a plot
\colorlet{fill2}{blue!10} % Color of fill in a plot
\colorlet{fillp}{fill1} % Color of positive area
\colorlet{filln}{red!50!black!20} % Color of negative area
\colorlet{gridColor}{gray!50} % Color of grid in a plot

\pgfmathdeclarefunction{gauss}{2}{% gives gaussian
  \pgfmathparse{1/(#2*sqrt(2*pi))*exp(-((x-#1)^2)/(2*#2^2))}%
}



\newcommand{\fullwidth}{}
\newcommand{\normalwidth}{}



%% makes a snazzy t-chart for evaluating functions
\newenvironment{tchart}{\rowcolors{2}{}{background!90!textColor}\array}{\endarray}

%%This is to help with formatting on future title pages.
\newenvironment{sectionOutcomes}{}{} 


\author{Gregory Hartman \and Matthew Carr}
\license{Creative Commons 3.0 By-NC}
\acknowledgement{https://github.com/APEXCalculus}

\begin{document}
\begin{exercise}

\outcome{State the precise definition of a limit.}
\outcome{Understand the concept of a limit.}

\tag{limit}
\tag{formal definition of the limit}

% I'm not sure how we give the formal definition of the limit, if we do give it.
% I've edited the source to conform to my preferred convention as I find
% statements using `whenever' to be confusing.


What is wrong with the following ``definition'' of a limit?
	\begin{quote}
``The limit of $f(x)$, as $x$ approaches $a$, is $L$'' means that given any $\delta>0$ there exists $\epsilon>0$ such that if $\left|{f(x)-L}\right|< \epsilon$, then we have $\left|{x-a}\right|<\delta$.
	\end{quote}
	
\begin{prompt}
\begin{multipleChoice}
 \choice{Nothing, this definition is correct.}
 \choice{It should be ``given any $\epsilon>0$," not $\delta>0$.}
 \choice{It should be that if $\left|{x-a}\right|<\delta$, then $\left|{f(x)-L}\right|<\epsilon$, not the other way around.}
 \choice[correct]{It should be ``given any $\epsilon>0$," not $\delta>0$, and it should be that if $\left|{x-a}\right|<\delta$, then we have $\left|{f(x)-L}\right|< \epsilon$, not the other way around.}
\end{multipleChoice}
\end{prompt}

\end{exercise}
\end{document}