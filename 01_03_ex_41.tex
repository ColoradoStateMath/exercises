\documentclass{ximera}

\usepackage{todonotes}

\usepackage{tkz-euclide}
\usetikzlibrary{backgrounds} %% for boxes around graphs
\usetikzlibrary{shapes,positioning}  %% Clouds and stars
\usetkzobj{all}
\usepackage[makeroom]{cancel} %% for strike outs
%\usepackage{mathtools} %% for pretty underbrace % Breaks Ximera
\usepackage{multicol}


\newcommand{\RR}{\mathbb R}
\renewcommand{\d}{\,d}
\newcommand{\dd}[2][]{\frac{d #1}{d #2}}
\renewcommand{\l}{\ell}
\newcommand{\ddx}{\frac{d}{dx}}
\newcommand{\zeroOverZero}{$\boldsymbol{\tfrac{0}{0}}$}
\newcommand{\numOverZero}{$\boldsymbol{\tfrac{\#}{0}}$}
\newcommand{\dfn}{\textbf}
\newcommand{\eval}[1]{\bigg[ #1 \bigg]}
\renewcommand{\epsilon}{\varepsilon}
\renewcommand{\iff}{\Leftrightarrow}

\DeclareMathOperator{\arccot}{arccot}
\DeclareMathOperator{\arcsec}{arcsec}
\DeclareMathOperator{\arccsc}{arccsc}


\colorlet{textColor}{black} 
\colorlet{background}{white}
\colorlet{penColor}{blue!50!black} % Color of a curve in a plot
\colorlet{penColor2}{red!50!black}% Color of a curve in a plot
\colorlet{penColor3}{red!50!blue} % Color of a curve in a plot
\colorlet{penColor4}{green!50!black} % Color of a curve in a plot
\colorlet{penColor5}{orange!80!black} % Color of a curve in a plot
                                      \colorlet{fill1}{blue!50!black!20} % Color of fill in a plot
\colorlet{fill2}{blue!10} % Color of fill in a plot
\colorlet{fillp}{fill1} % Color of positive area
\colorlet{filln}{red!50!black!20} % Color of negative area
\colorlet{gridColor}{gray!50} % Color of grid in a plot

\pgfmathdeclarefunction{gauss}{2}{% gives gaussian
  \pgfmathparse{1/(#2*sqrt(2*pi))*exp(-((x-#1)^2)/(2*#2^2))}%
}



\newcommand{\fullwidth}{}
\newcommand{\normalwidth}{}



%% makes a snazzy t-chart for evaluating functions
\newenvironment{tchart}{\rowcolors{2}{}{background!90!textColor}\array}{\endarray}

%%This is to help with formatting on future title pages.
\newenvironment{sectionOutcomes}{}{} 


\author{Gregory Hartman \and Matthew Carr}
\license{Creative Commons 3.0 By-NC}
\acknowledgement{https://github.com/APEXCalculus}

\begin{document}
\begin{exercise}

\outcome{Calculate limits using the limit laws.}
\outcome{Calculate limits using the Squeeze Theorem.}

\tag{limit} 
\tag{continuous}

% BADBAD
% I believe the author meant \lim_{x\to3^{-}} (i.e., 3^-) instead of 3^+. 
% As discussed (e.g., that we don't expect them to know that \sqrt{x} is continuous at x=0)
% I have extended the interval from [0,3] to [0,5] and let the limit be x\to3 

Find 
\[
\lim_{x\to3}f(x)
\begin{prompt}
= \answer{9}.
\end{prompt}
\]
given, on the interval $\left[0,5\right]$, that $6x-9\le f(x)\le x^2$.
\begin{hint}
Use the Squeeze Theorem on the two bounds for $f(x)$. Observe that $\lim_{x\to3}\left({6x-9}\right)=6\cdot\lim_{x\to3}(x)-\lim_{x\to3}\left({9}\right)=9$ and $\lim_{x\to3}\left({x^2}\right)=\left({\lim_{x\to3}(x)}\right)^2=9$.
\end{hint}
\begin{hint}
Since we are given that $6x-9\le f(x)\le x^2$ on $\left[0,5\right]$, and we have seen that $\lim_{x\to3}\left({6x-3}\right)=\lim_{x\to3}\left({x^2}\right)=9$, it follows by the Squeeze Theorem that $\lim_{x\to3}f(x)=9$.
\end{hint}
\end{exercise}
\end{document}