\documentclass{ximera}
\usepackage{todonotes}

\usepackage{tkz-euclide}
\usetikzlibrary{backgrounds} %% for boxes around graphs
\usetikzlibrary{shapes,positioning}  %% Clouds and stars
\usetkzobj{all}
\usepackage[makeroom]{cancel} %% for strike outs
%\usepackage{mathtools} %% for pretty underbrace % Breaks Ximera
\usepackage{multicol}


\newcommand{\RR}{\mathbb R}
\renewcommand{\d}{\,d}
\newcommand{\dd}[2][]{\frac{d #1}{d #2}}
\renewcommand{\l}{\ell}
\newcommand{\ddx}{\frac{d}{dx}}
\newcommand{\zeroOverZero}{$\boldsymbol{\tfrac{0}{0}}$}
\newcommand{\numOverZero}{$\boldsymbol{\tfrac{\#}{0}}$}
\newcommand{\dfn}{\textbf}
\newcommand{\eval}[1]{\bigg[ #1 \bigg]}
\renewcommand{\epsilon}{\varepsilon}
\renewcommand{\iff}{\Leftrightarrow}

\DeclareMathOperator{\arccot}{arccot}
\DeclareMathOperator{\arcsec}{arcsec}
\DeclareMathOperator{\arccsc}{arccsc}


\colorlet{textColor}{black} 
\colorlet{background}{white}
\colorlet{penColor}{blue!50!black} % Color of a curve in a plot
\colorlet{penColor2}{red!50!black}% Color of a curve in a plot
\colorlet{penColor3}{red!50!blue} % Color of a curve in a plot
\colorlet{penColor4}{green!50!black} % Color of a curve in a plot
\colorlet{penColor5}{orange!80!black} % Color of a curve in a plot
                                      \colorlet{fill1}{blue!50!black!20} % Color of fill in a plot
\colorlet{fill2}{blue!10} % Color of fill in a plot
\colorlet{fillp}{fill1} % Color of positive area
\colorlet{filln}{red!50!black!20} % Color of negative area
\colorlet{gridColor}{gray!50} % Color of grid in a plot

\pgfmathdeclarefunction{gauss}{2}{% gives gaussian
  \pgfmathparse{1/(#2*sqrt(2*pi))*exp(-((x-#1)^2)/(2*#2^2))}%
}



\newcommand{\fullwidth}{}
\newcommand{\normalwidth}{}



%% makes a snazzy t-chart for evaluating functions
\newenvironment{tchart}{\rowcolors{2}{}{background!90!textColor}\array}{\endarray}

%%This is to help with formatting on future title pages.
\newenvironment{sectionOutcomes}{}{} 

\author{Steven Gubkin}
\license{Creative Commons 3.0 By-NC}
\begin{document}
\begin{exercise}

\outcome{Define linear approximation as an application of the tangent to a curve.}
 \outcome{Find the linear approximation to a function at a point and use it to approximate the function value.}
 \outcome{Identify when a linear approximation can be used.}

\tag{derivative}

In Einstein's theory of relativity, we can derive that

$$E(v) = \frac{mc^2}{\sqrt{1-\frac{v^2}{c^2}}}$$

where $E(v)$ is the energy of an object with ``Rest mass'' $m$ and velocity $v$.

Let us analyze this more closely.

First find the linear approximation to the function $f(u) = \frac{mc^2}{\sqrt{1-u}}$ at $u=0$.

Using this approximation, and substituting $u = \frac{v^2}{c^2}$, we can obtain an approximation for $E(v)$ which is valid for small velocities $v$.

The approximation you obtain should have two terms.  One of which is the famous $E = mc^2$ (representing the resting energy) and the other should be the classical kinetic energy of the object.

\begin{prompt}
	The local linearization of $\frac{m c^2}{\sqrt{1-u}}$ at $u=0$ is $\answer{m c^2 +\frac{m c^2}{2}u}$.

	So we get that $E(v) \approx \answer{mc^2+\frac{1}{2}mv^2}$.
\end{prompt}

\end{exercise}
\end{document}
