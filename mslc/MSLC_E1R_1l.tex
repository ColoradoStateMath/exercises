\documentclass{ximera}

\usepackage{todonotes}

\usepackage{tkz-euclide}
\usetikzlibrary{backgrounds} %% for boxes around graphs
\usetikzlibrary{shapes,positioning}  %% Clouds and stars
\usetkzobj{all}
\usepackage[makeroom]{cancel} %% for strike outs
%\usepackage{mathtools} %% for pretty underbrace % Breaks Ximera
\usepackage{multicol}


\newcommand{\RR}{\mathbb R}
\renewcommand{\d}{\,d}
\newcommand{\dd}[2][]{\frac{d #1}{d #2}}
\renewcommand{\l}{\ell}
\newcommand{\ddx}{\frac{d}{dx}}
\newcommand{\zeroOverZero}{$\boldsymbol{\tfrac{0}{0}}$}
\newcommand{\numOverZero}{$\boldsymbol{\tfrac{\#}{0}}$}
\newcommand{\dfn}{\textbf}
\newcommand{\eval}[1]{\bigg[ #1 \bigg]}
\renewcommand{\epsilon}{\varepsilon}
\renewcommand{\iff}{\Leftrightarrow}

\DeclareMathOperator{\arccot}{arccot}
\DeclareMathOperator{\arcsec}{arcsec}
\DeclareMathOperator{\arccsc}{arccsc}


\colorlet{textColor}{black} 
\colorlet{background}{white}
\colorlet{penColor}{blue!50!black} % Color of a curve in a plot
\colorlet{penColor2}{red!50!black}% Color of a curve in a plot
\colorlet{penColor3}{red!50!blue} % Color of a curve in a plot
\colorlet{penColor4}{green!50!black} % Color of a curve in a plot
\colorlet{penColor5}{orange!80!black} % Color of a curve in a plot
                                      \colorlet{fill1}{blue!50!black!20} % Color of fill in a plot
\colorlet{fill2}{blue!10} % Color of fill in a plot
\colorlet{fillp}{fill1} % Color of positive area
\colorlet{filln}{red!50!black!20} % Color of negative area
\colorlet{gridColor}{gray!50} % Color of grid in a plot

\pgfmathdeclarefunction{gauss}{2}{% gives gaussian
  \pgfmathparse{1/(#2*sqrt(2*pi))*exp(-((x-#1)^2)/(2*#2^2))}%
}



\newcommand{\fullwidth}{}
\newcommand{\normalwidth}{}



%% makes a snazzy t-chart for evaluating functions
\newenvironment{tchart}{\rowcolors{2}{}{background!90!textColor}\array}{\endarray}

%%This is to help with formatting on future title pages.
\newenvironment{sectionOutcomes}{}{} 


\author{Matthew Carr}
%% BADBAD
%% From OSU 1151 Sample Exams
%% License?
%% Acknowledgement?
\license{BADBAD}
\acknowledgement{BADBAD}

\begin{document}
\begin{exercise}

\outcome{Calculate limits using the limit laws.}
\outcome{Calculate the limit as $x$ approaches $\pm\infty$ of common functions algebraically.}

\tag{limit at infinity}

Find
\[
\lim_{x\to\infty}\left(2x-\sqrt{4x^2-x}\right)
\begin{prompt}
= \answer{1/4}
\end{prompt}
\]

\begin{hint}
Recall that the conjugate of $a+\sqrt{b}$ is $a-\sqrt{b}$ (similarly, the conjugate of $a-\sqrt{b}$ is $a+\sqrt{b}$). Multiply through by the clever form of $1$.
\end{hint}
\begin{hint}
The clever form of $1$ alluded to is simply the fraction containing the conjugate: $1=\frac{2x+\sqrt{4x^2-x}}{2x+\sqrt{4x^2-x}}$. After some algebra, you should see that the new numerator is $\left(2x-\sqrt{4x^2-x}\right)\left(2x+\sqrt{4x^2-x}\right)=4x^2-4x^2+x=x$. Similarly, the new denominator is $2x+\sqrt{4x^2-x}$. Hence, $2x-\sqrt{4x^2-x}=\frac{x}{2x+\sqrt{4x^2-x}}$. This is certainly a more a tractable form! 
\end{hint}
\begin{hint}
Multiplying both the numerator and the denominator by $1$ over the highest power of $x$ (i.e., $1=\frac{\frac{1}{x}}{\frac{1}{x}}$) we obtain the expression $\frac{x\frac{1}{x}}{2x\frac{1}{x}+\sqrt{4x^2-x}\frac{1}{x}}=\frac{1}{2+\sqrt{4x^2-x}\frac{1}{x}}$. Of course, $\sqrt{4x^2-x}\frac{1}{x}=\sqrt{\frac{4x^2-x}{x^2}}=\sqrt{4-\frac{1}{x}}$. How exceedingly fortunate, since we have transformed our formally intractable problem into $\lim_{x\to\infty}\left(\frac{1}{2+\sqrt{4-\frac{1}{x}}}\right)$ and $\lim_{x\to\infty}\left(\frac{1}{2+\sqrt{4-\frac{1}{x}}}\right)=\frac{1}{2+2}=\frac{1}{4}$, hence, $\lim_{x\to\infty}\left(2x-\sqrt{4x^2-x}\right)=\frac{1}{4}$.
\end{hint}
\end{exercise}
\end{document}
