\documentclass{ximera}

\usepackage{todonotes}

\usepackage{tkz-euclide}
\usetikzlibrary{backgrounds} %% for boxes around graphs
\usetikzlibrary{shapes,positioning}  %% Clouds and stars
\usetkzobj{all}
\usepackage[makeroom]{cancel} %% for strike outs
%\usepackage{mathtools} %% for pretty underbrace % Breaks Ximera
\usepackage{multicol}


\newcommand{\RR}{\mathbb R}
\renewcommand{\d}{\,d}
\newcommand{\dd}[2][]{\frac{d #1}{d #2}}
\renewcommand{\l}{\ell}
\newcommand{\ddx}{\frac{d}{dx}}
\newcommand{\zeroOverZero}{$\boldsymbol{\tfrac{0}{0}}$}
\newcommand{\numOverZero}{$\boldsymbol{\tfrac{\#}{0}}$}
\newcommand{\dfn}{\textbf}
\newcommand{\eval}[1]{\bigg[ #1 \bigg]}
\renewcommand{\epsilon}{\varepsilon}
\renewcommand{\iff}{\Leftrightarrow}

\DeclareMathOperator{\arccot}{arccot}
\DeclareMathOperator{\arcsec}{arcsec}
\DeclareMathOperator{\arccsc}{arccsc}


\colorlet{textColor}{black} 
\colorlet{background}{white}
\colorlet{penColor}{blue!50!black} % Color of a curve in a plot
\colorlet{penColor2}{red!50!black}% Color of a curve in a plot
\colorlet{penColor3}{red!50!blue} % Color of a curve in a plot
\colorlet{penColor4}{green!50!black} % Color of a curve in a plot
\colorlet{penColor5}{orange!80!black} % Color of a curve in a plot
                                      \colorlet{fill1}{blue!50!black!20} % Color of fill in a plot
\colorlet{fill2}{blue!10} % Color of fill in a plot
\colorlet{fillp}{fill1} % Color of positive area
\colorlet{filln}{red!50!black!20} % Color of negative area
\colorlet{gridColor}{gray!50} % Color of grid in a plot

\pgfmathdeclarefunction{gauss}{2}{% gives gaussian
  \pgfmathparse{1/(#2*sqrt(2*pi))*exp(-((x-#1)^2)/(2*#2^2))}%
}



\newcommand{\fullwidth}{}
\newcommand{\normalwidth}{}



%% makes a snazzy t-chart for evaluating functions
\newenvironment{tchart}{\rowcolors{2}{}{background!90!textColor}\array}{\endarray}

%%This is to help with formatting on future title pages.
\newenvironment{sectionOutcomes}{}{} 


\author{Carl Stitz \and Jeff Zeager \and Bart Snapp \and Matthew Carr}
\license{CC-By-SA-NC}
\acknowledgement{http://www.stitz-zeager.com/}

\begin{document}
\begin{exercise}


\outcome{Find domain and range.}
\outcome{Define and work with inverse functions.}
\outcome{Determine where a function is positive or negative}
\outcome{Find the intervals where a function is increasing or decreasing}

\tag{domain}
\tag{inverse}
\tag{increasing}
\tag{decreasing}

Now use the graph of $y = f(x)$ given below to answer the question.
\begin{image}
  \begin{tikzpicture}
    \begin{axis}[
        xmin=-5, xmax=5, ymin=-6,ymax=6,    
        unit vector ratio*=1 1 1,
        axis lines =middle, xlabel=$x$, ylabel=$y$,
        every axis y label/.style={at=(current axis.above origin),anchor=south},
        every axis x label/.style={at=(current axis.right of origin),anchor=west},
        xtick={-5,...,5}, ytick={-5,...,5},
        %xticklabels={-4,,-2,,0,,2,,4,,6}, yticklabels={,-2,,0,,2,,4,,6,,8,,10},
        grid=major,width=4in,
        grid style={dashed, gridColor},
      ]
      \addplot[color=penColor,very thick,smooth,domain=-4:4] {5*sin(x*180/4)};
      \addplot[color=penColor,fill=background,only marks,mark=*] coordinates{(2,5)};  %% open hole
      \addplot[color=penColor,fill=penColor,only marks,mark=*] coordinates{(2,3)};  %% closed hole
      
    \end{axis}
  \end{tikzpicture}
\end{image}


List the intervals where $f$ is increasing written from left to right. If none exist, write DNE. \[[\answer{-2},\answer{2})\]

\end{exercise}
\end{document}