\documentclass{ximera}
%\usepackage{todonotes}

\usepackage{tkz-euclide}
\usetikzlibrary{backgrounds} %% for boxes around graphs
\usetikzlibrary{shapes,positioning}  %% Clouds and stars
\usetkzobj{all}
\usepackage[makeroom]{cancel} %% for strike outs
%\usepackage{mathtools} %% for pretty underbrace % Breaks Ximera
\usepackage{multicol}


\newcommand{\RR}{\mathbb R}
\renewcommand{\d}{\,d}
\newcommand{\dd}[2][]{\frac{d #1}{d #2}}
\renewcommand{\l}{\ell}
\newcommand{\ddx}{\frac{d}{dx}}
\newcommand{\zeroOverZero}{$\boldsymbol{\tfrac{0}{0}}$}
\newcommand{\numOverZero}{$\boldsymbol{\tfrac{\#}{0}}$}
\newcommand{\dfn}{\textbf}
\newcommand{\eval}[1]{\bigg[ #1 \bigg]}
\renewcommand{\epsilon}{\varepsilon}
\renewcommand{\iff}{\Leftrightarrow}

\DeclareMathOperator{\arccot}{arccot}
\DeclareMathOperator{\arcsec}{arcsec}
\DeclareMathOperator{\arccsc}{arccsc}


\colorlet{textColor}{black} 
\colorlet{background}{white}
\colorlet{penColor}{blue!50!black} % Color of a curve in a plot
\colorlet{penColor2}{red!50!black}% Color of a curve in a plot
\colorlet{penColor3}{red!50!blue} % Color of a curve in a plot
\colorlet{penColor4}{green!50!black} % Color of a curve in a plot
\colorlet{penColor5}{orange!80!black} % Color of a curve in a plot
                                      \colorlet{fill1}{blue!50!black!20} % Color of fill in a plot
\colorlet{fill2}{blue!10} % Color of fill in a plot
\colorlet{fillp}{fill1} % Color of positive area
\colorlet{filln}{red!50!black!20} % Color of negative area
\colorlet{gridColor}{gray!50} % Color of grid in a plot

\pgfmathdeclarefunction{gauss}{2}{% gives gaussian
  \pgfmathparse{1/(#2*sqrt(2*pi))*exp(-((x-#1)^2)/(2*#2^2))}%
}



\newcommand{\fullwidth}{}
\newcommand{\normalwidth}{}



%% makes a snazzy t-chart for evaluating functions
\newenvironment{tchart}{\rowcolors{2}{}{background!90!textColor}\array}{\endarray}

%%This is to help with formatting on future title pages.
\newenvironment{sectionOutcomes}{}{} 

\author{Emma Smith Zbarsky}
\license{Creative Commons Attribution 3.0 Unported}
\acknowledgement{https://quadbase.org/questions/q14279v1}
\begin{document}

\begin{exercise}

Calculate the derivative with respect to $\theta$ of
\[r(\theta) = \left(8\pi\theta-\frac{3\pi}{2}\right)\tan(\theta+\pi/2).\]


\begin{hint}
This is a mixed derivative rule problem. There is a product rule and a
chain rule.
\end{hint}


\begin{hint}
Letting $r(\theta) = ab$ we could take $a = 8\pi\theta-\frac{3\pi}{2}$
and $b = \tan(\theta+\pi/2)$. Then $a'=8\pi$ is a simple power rule
problem but $b = f(g(\theta))$ with $f = \tan(g)$ and $g = \theta+\pi/2$
involves a chain rule to calculate
\[b' = \sec^2(\theta+\pi/2)\cdot(1) = \sec^2(\theta+\pi/2).\] Putting it
all together, we have: \begin{align*}
r'(\theta) &= 8\pi\tan(\theta+\pi/2) + \left(8\pi\theta-\frac{3\pi}{2}\right)\sec^2(\theta+\pi/2)
\end{align*}
\end{hint}


\begin{multipleChoice}
\choice{$r'(\theta) = 8\pi\sec^2(1)$}
\choice{$r'(\theta) = 8\pi\tan(\theta+\pi/2)-\left(8\pi\theta-\frac{3\pi}{2}\right)\sec^2(\theta+\pi/2)$}
\choice{$r'(\theta) = 8\pi\sec^2(\theta+\pi/2)$}
\choice[correct]{$r'(\theta) = 8\pi\tan(\theta+\pi/2)+\left(8\pi\theta-\frac{3\pi}{2}\right)\sec^2(\theta+\pi/2)$}
\choice{$r'(\theta) = 0$}
\end{multipleChoice}

\end{exercise}
\end{document}
