\documentclass{ximera}
%\usepackage{todonotes}

\usepackage{tkz-euclide}
\usetikzlibrary{backgrounds} %% for boxes around graphs
\usetikzlibrary{shapes,positioning}  %% Clouds and stars
\usetkzobj{all}
\usepackage[makeroom]{cancel} %% for strike outs
%\usepackage{mathtools} %% for pretty underbrace % Breaks Ximera
\usepackage{multicol}


\newcommand{\RR}{\mathbb R}
\renewcommand{\d}{\,d}
\newcommand{\dd}[2][]{\frac{d #1}{d #2}}
\renewcommand{\l}{\ell}
\newcommand{\ddx}{\frac{d}{dx}}
\newcommand{\zeroOverZero}{$\boldsymbol{\tfrac{0}{0}}$}
\newcommand{\numOverZero}{$\boldsymbol{\tfrac{\#}{0}}$}
\newcommand{\dfn}{\textbf}
\newcommand{\eval}[1]{\bigg[ #1 \bigg]}
\renewcommand{\epsilon}{\varepsilon}
\renewcommand{\iff}{\Leftrightarrow}

\DeclareMathOperator{\arccot}{arccot}
\DeclareMathOperator{\arcsec}{arcsec}
\DeclareMathOperator{\arccsc}{arccsc}


\colorlet{textColor}{black} 
\colorlet{background}{white}
\colorlet{penColor}{blue!50!black} % Color of a curve in a plot
\colorlet{penColor2}{red!50!black}% Color of a curve in a plot
\colorlet{penColor3}{red!50!blue} % Color of a curve in a plot
\colorlet{penColor4}{green!50!black} % Color of a curve in a plot
\colorlet{penColor5}{orange!80!black} % Color of a curve in a plot
                                      \colorlet{fill1}{blue!50!black!20} % Color of fill in a plot
\colorlet{fill2}{blue!10} % Color of fill in a plot
\colorlet{fillp}{fill1} % Color of positive area
\colorlet{filln}{red!50!black!20} % Color of negative area
\colorlet{gridColor}{gray!50} % Color of grid in a plot

\pgfmathdeclarefunction{gauss}{2}{% gives gaussian
  \pgfmathparse{1/(#2*sqrt(2*pi))*exp(-((x-#1)^2)/(2*#2^2))}%
}



\newcommand{\fullwidth}{}
\newcommand{\normalwidth}{}



%% makes a snazzy t-chart for evaluating functions
\newenvironment{tchart}{\rowcolors{2}{}{background!90!textColor}\array}{\endarray}

%%This is to help with formatting on future title pages.
\newenvironment{sectionOutcomes}{}{} 

\author{Emma Smith Zbarsky}
\license{Creative Commons Attribution 3.0 Unported}
\acknowledgement{https://quadbase.org/questions/q13648v1}
\begin{document}

\begin{exercise}

Say we have a hurricane with a circular shape. If the radius of the
storm is growing at 5 miles per hour, how fast is the area of the storm
growing when the storm is 120 miles in diameter?


\begin{hint}
Identify the relationship between the known and unknown variables and
differentiate it to relate the known and unknown rates.
\end{hint}


\begin{hint}
Name the variables:

*area = $A$ in square miles

*radius = $r$ in miles

*time = $t$ in hours

Known values: \[\frac{dr}{dt} = 5 \mbox{ miles per hour }\]
\[\mbox{diameter} = 120 \mbox{ miles} \Rightarrow r = 60 \mbox{ miles}\]

Known formulas (relate the variables):

\begin{itemize}
\itemsep1pt\parskip0pt\parsep0pt
\item
  Area of a circle: \[A = \pi r^2\]
\end{itemize}

Related rates (take the derivative with respect to time):
\[\frac{dA}{dt} = 2\pi r \cdot \frac{dr}{dt}\]

Plug in the known values:
\[\frac{dA}{dt} = 2\cdot \pi \cdot 60 \cdot 5 = 600 \pi \mbox{ square miles per hour}\]
\end{hint}


\begin{multipleChoice}
\choice{$300 \pi$ miles squared per hour}
\choice{$240 \pi$ miles squared per hour}
\choice{$10 \pi$ miles squared per hour}
\choice{$5\pi$ miles per hour}
\choice{$120\pi$ miles squared per hour}
\choice{$10 \pi$ miles per hour}
\choice[correct]{$600 \pi$ miles squared per hour}
\choice{$5\pi$ miles squared per hour}
\choice{$1200 \pi$ miles squared per hour}
\end{multipleChoice}

\end{exercise}
\end{document}
