\documentclass{ximera}
%\usepackage{todonotes}

\usepackage{tkz-euclide}
\usetikzlibrary{backgrounds} %% for boxes around graphs
\usetikzlibrary{shapes,positioning}  %% Clouds and stars
\usetkzobj{all}
\usepackage[makeroom]{cancel} %% for strike outs
%\usepackage{mathtools} %% for pretty underbrace % Breaks Ximera
\usepackage{multicol}


\newcommand{\RR}{\mathbb R}
\renewcommand{\d}{\,d}
\newcommand{\dd}[2][]{\frac{d #1}{d #2}}
\renewcommand{\l}{\ell}
\newcommand{\ddx}{\frac{d}{dx}}
\newcommand{\zeroOverZero}{$\boldsymbol{\tfrac{0}{0}}$}
\newcommand{\numOverZero}{$\boldsymbol{\tfrac{\#}{0}}$}
\newcommand{\dfn}{\textbf}
\newcommand{\eval}[1]{\bigg[ #1 \bigg]}
\renewcommand{\epsilon}{\varepsilon}
\renewcommand{\iff}{\Leftrightarrow}

\DeclareMathOperator{\arccot}{arccot}
\DeclareMathOperator{\arcsec}{arcsec}
\DeclareMathOperator{\arccsc}{arccsc}


\colorlet{textColor}{black} 
\colorlet{background}{white}
\colorlet{penColor}{blue!50!black} % Color of a curve in a plot
\colorlet{penColor2}{red!50!black}% Color of a curve in a plot
\colorlet{penColor3}{red!50!blue} % Color of a curve in a plot
\colorlet{penColor4}{green!50!black} % Color of a curve in a plot
\colorlet{penColor5}{orange!80!black} % Color of a curve in a plot
                                      \colorlet{fill1}{blue!50!black!20} % Color of fill in a plot
\colorlet{fill2}{blue!10} % Color of fill in a plot
\colorlet{fillp}{fill1} % Color of positive area
\colorlet{filln}{red!50!black!20} % Color of negative area
\colorlet{gridColor}{gray!50} % Color of grid in a plot

\pgfmathdeclarefunction{gauss}{2}{% gives gaussian
  \pgfmathparse{1/(#2*sqrt(2*pi))*exp(-((x-#1)^2)/(2*#2^2))}%
}



\newcommand{\fullwidth}{}
\newcommand{\normalwidth}{}



%% makes a snazzy t-chart for evaluating functions
\newenvironment{tchart}{\rowcolors{2}{}{background!90!textColor}\array}{\endarray}

%%This is to help with formatting on future title pages.
\newenvironment{sectionOutcomes}{}{} 

\author{Emma Smith Zbarsky}
\license{Creative Commons Attribution 3.0 Unported}
\acknowledgement{https://quadbase.org/questions/q14535v1}
\begin{document}

\begin{exercise}

Consider a trip along the Mass Pike from the intersection with I-95 to
the Berkshires. You pass a toll booth on entering at the I-95
interchange at 12:35 pm. You pass a toll both when exiting to the
Berkshires at 2:05 pm. Having gone 116 miles, if the maximum speed limit
on the route is 70 miles per hour, do you deserve a ticket? Why or why
not?


\begin{hint}
The mean value theorem says that if $f(x)$ is defined and continuous on
the interval $[a,b]$, then there is at least one point $c \in (a,b)$
with $f'(c) = \frac{f(b)-f(a)}{b-a}.$
\end{hint}


\begin{hint}
Applying the mean value theorem in this case, we see that we traveled
\[\frac{116 \mbox{ miles}}{2:05-12:35 \mbox{ hours}} = \frac{116}{1.5} \mbox{ mph} = 77.\overline{3} \mbox{ mph}.\]
Therefore, if we consider our position as $f(t)$, there is some time $t$
between 12:35 pm and 2:05 pm when $f'(t) = 77.\overline{3}$ mph so we
must have been driving over the speed limit.
\end{hint}


\begin{multipleChoice}
\choice{No, you traveled at an average speed of just over 66 mph, so while you
may have exceeded the speed limit, the toll booth operators cannot prove
it.}
\choice[correct]{Yes, you traveled at an average speed of just over 77 mph, so you must
have exceeded the speed limit for a large part of your journey (or gone
extremely fast for a short time).}
\end{multipleChoice}

\end{exercise}
\end{document}
