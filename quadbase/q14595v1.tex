\documentclass{ximera}
%\usepackage{todonotes}

\usepackage{tkz-euclide}
\usetikzlibrary{backgrounds} %% for boxes around graphs
\usetikzlibrary{shapes,positioning}  %% Clouds and stars
\usetkzobj{all}
\usepackage[makeroom]{cancel} %% for strike outs
%\usepackage{mathtools} %% for pretty underbrace % Breaks Ximera
\usepackage{multicol}


\newcommand{\RR}{\mathbb R}
\renewcommand{\d}{\,d}
\newcommand{\dd}[2][]{\frac{d #1}{d #2}}
\renewcommand{\l}{\ell}
\newcommand{\ddx}{\frac{d}{dx}}
\newcommand{\zeroOverZero}{$\boldsymbol{\tfrac{0}{0}}$}
\newcommand{\numOverZero}{$\boldsymbol{\tfrac{\#}{0}}$}
\newcommand{\dfn}{\textbf}
\newcommand{\eval}[1]{\bigg[ #1 \bigg]}
\renewcommand{\epsilon}{\varepsilon}
\renewcommand{\iff}{\Leftrightarrow}

\DeclareMathOperator{\arccot}{arccot}
\DeclareMathOperator{\arcsec}{arcsec}
\DeclareMathOperator{\arccsc}{arccsc}


\colorlet{textColor}{black} 
\colorlet{background}{white}
\colorlet{penColor}{blue!50!black} % Color of a curve in a plot
\colorlet{penColor2}{red!50!black}% Color of a curve in a plot
\colorlet{penColor3}{red!50!blue} % Color of a curve in a plot
\colorlet{penColor4}{green!50!black} % Color of a curve in a plot
\colorlet{penColor5}{orange!80!black} % Color of a curve in a plot
                                      \colorlet{fill1}{blue!50!black!20} % Color of fill in a plot
\colorlet{fill2}{blue!10} % Color of fill in a plot
\colorlet{fillp}{fill1} % Color of positive area
\colorlet{filln}{red!50!black!20} % Color of negative area
\colorlet{gridColor}{gray!50} % Color of grid in a plot

\pgfmathdeclarefunction{gauss}{2}{% gives gaussian
  \pgfmathparse{1/(#2*sqrt(2*pi))*exp(-((x-#1)^2)/(2*#2^2))}%
}



\newcommand{\fullwidth}{}
\newcommand{\normalwidth}{}



%% makes a snazzy t-chart for evaluating functions
\newenvironment{tchart}{\rowcolors{2}{}{background!90!textColor}\array}{\endarray}

%%This is to help with formatting on future title pages.
\newenvironment{sectionOutcomes}{}{} 

\author{Emma Smith Zbarsky}
\license{Creative Commons Attribution 3.0 Unported}
\acknowledgement{https://quadbase.org/questions/q14595v1}
\begin{document}

\begin{exercise}

True or False.

The midpoint approximation Riemann sum is always closer, for any fixed
set of subintervals, to the exact value of the signed area under the
curve than either the left-endpoint Riemann sum or the right-endpoint
Riemann sum.


\begin{hint}
Think conceptually about functions that might have strange properties,
for instance radically different values around their midpoints, then
check the details to see if you can argue for truth or falsehood.
\end{hint}


\begin{hint}
Consider \[f(x) = \begin{cases} 10 & |x|>1 \\
10|x| & |x|\leq 1 \end{cases}\] The actual area under the curve over the
interval $[-10,10]$ is
\[10*9+\frac{1}{2}(1)(10)+\frac{1}{2}(1)(10)+10*9 = 2(90+5) = 190.\] If
we take one subinterval $[-10,10]$ then the right endpoint approximation
is \[f(10)*\left(10-(-10)\right) = 10*20 = 200\] and the left endpoint
approximation is \[f(-10)*\left(10-(-10)\right) = 10*20 = 200\] but the
midpoint approximation is
\[f\left(\frac{10+(-10)}{2}\right)*\left(10-(-10)\right) = f(0)*20 = 0*20 = 0.\]
In this case, both the right- and left-hand Riemann sums are closer to
the exact value.
\end{hint}


\begin{multipleChoice}
\choice[correct]{False}
\choice{True}
\end{multipleChoice}

\end{exercise}
\end{document}
