\documentclass{ximera}
%\usepackage{todonotes}

\usepackage{tkz-euclide}
\usetikzlibrary{backgrounds} %% for boxes around graphs
\usetikzlibrary{shapes,positioning}  %% Clouds and stars
\usetkzobj{all}
\usepackage[makeroom]{cancel} %% for strike outs
%\usepackage{mathtools} %% for pretty underbrace % Breaks Ximera
\usepackage{multicol}


\newcommand{\RR}{\mathbb R}
\renewcommand{\d}{\,d}
\newcommand{\dd}[2][]{\frac{d #1}{d #2}}
\renewcommand{\l}{\ell}
\newcommand{\ddx}{\frac{d}{dx}}
\newcommand{\zeroOverZero}{$\boldsymbol{\tfrac{0}{0}}$}
\newcommand{\numOverZero}{$\boldsymbol{\tfrac{\#}{0}}$}
\newcommand{\dfn}{\textbf}
\newcommand{\eval}[1]{\bigg[ #1 \bigg]}
\renewcommand{\epsilon}{\varepsilon}
\renewcommand{\iff}{\Leftrightarrow}

\DeclareMathOperator{\arccot}{arccot}
\DeclareMathOperator{\arcsec}{arcsec}
\DeclareMathOperator{\arccsc}{arccsc}


\colorlet{textColor}{black} 
\colorlet{background}{white}
\colorlet{penColor}{blue!50!black} % Color of a curve in a plot
\colorlet{penColor2}{red!50!black}% Color of a curve in a plot
\colorlet{penColor3}{red!50!blue} % Color of a curve in a plot
\colorlet{penColor4}{green!50!black} % Color of a curve in a plot
\colorlet{penColor5}{orange!80!black} % Color of a curve in a plot
                                      \colorlet{fill1}{blue!50!black!20} % Color of fill in a plot
\colorlet{fill2}{blue!10} % Color of fill in a plot
\colorlet{fillp}{fill1} % Color of positive area
\colorlet{filln}{red!50!black!20} % Color of negative area
\colorlet{gridColor}{gray!50} % Color of grid in a plot

\pgfmathdeclarefunction{gauss}{2}{% gives gaussian
  \pgfmathparse{1/(#2*sqrt(2*pi))*exp(-((x-#1)^2)/(2*#2^2))}%
}



\newcommand{\fullwidth}{}
\newcommand{\normalwidth}{}



%% makes a snazzy t-chart for evaluating functions
\newenvironment{tchart}{\rowcolors{2}{}{background!90!textColor}\array}{\endarray}

%%This is to help with formatting on future title pages.
\newenvironment{sectionOutcomes}{}{} 

\author{Emma Smith Zbarsky}
\license{Creative Commons Attribution 3.0 Unported}
\acknowledgement{https://quadbase.org/questions/q14429v1}
\begin{document}

\begin{exercise}

Find the antiderivative: \[\int x+3\; dx\]


\begin{hint}
This is an antiderivative problem. It is asking the question, what is
the most general form of an equation $f(x)$ such that $f'(x) = x+3$.
\end{hint}


\begin{hint}
To solve this problem, we need to apply the power rule ``backwards''.
That is, $g(x) = ax^n+C$ and $g'(x) = nax^{n-1}$, so we first use the
power on $x$ to solve for $n$ and then we can solve for $a$.

In this case, if we take $g'(x) = x = x^{2-1}$ we see that $n=2$. Since
$na=2a=1$, that means $a=1/2$. Therefore,
\[\int x \; dx = \frac{1}{2}x^2+C.\] Similarly, if we take
$g'(x) = 3 = 3x^{1-1}$ we see that $n=1$ and $na = a = 3$. Thus,
\[\int 3\; dx = 3x+C.\] Putting these together, we have:
\[\int x+3 \; dx = \frac{x^2}{2}+3x+C.\]
\end{hint}


\begin{multipleChoice}
\choice{$x^2+3x+C$}
\choice{$1$}
\choice[correct]{$\frac{x^2}{2}+3x+C$}
\choice{$1+C$}
\choice{$\frac{x^2}{2}+3x$}
\end{multipleChoice}

\end{exercise}
\end{document}
