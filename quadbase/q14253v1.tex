\documentclass{ximera}
%\usepackage{todonotes}

\usepackage{tkz-euclide}
\usetikzlibrary{backgrounds} %% for boxes around graphs
\usetikzlibrary{shapes,positioning}  %% Clouds and stars
\usetkzobj{all}
\usepackage[makeroom]{cancel} %% for strike outs
%\usepackage{mathtools} %% for pretty underbrace % Breaks Ximera
\usepackage{multicol}


\newcommand{\RR}{\mathbb R}
\renewcommand{\d}{\,d}
\newcommand{\dd}[2][]{\frac{d #1}{d #2}}
\renewcommand{\l}{\ell}
\newcommand{\ddx}{\frac{d}{dx}}
\newcommand{\zeroOverZero}{$\boldsymbol{\tfrac{0}{0}}$}
\newcommand{\numOverZero}{$\boldsymbol{\tfrac{\#}{0}}$}
\newcommand{\dfn}{\textbf}
\newcommand{\eval}[1]{\bigg[ #1 \bigg]}
\renewcommand{\epsilon}{\varepsilon}
\renewcommand{\iff}{\Leftrightarrow}

\DeclareMathOperator{\arccot}{arccot}
\DeclareMathOperator{\arcsec}{arcsec}
\DeclareMathOperator{\arccsc}{arccsc}


\colorlet{textColor}{black} 
\colorlet{background}{white}
\colorlet{penColor}{blue!50!black} % Color of a curve in a plot
\colorlet{penColor2}{red!50!black}% Color of a curve in a plot
\colorlet{penColor3}{red!50!blue} % Color of a curve in a plot
\colorlet{penColor4}{green!50!black} % Color of a curve in a plot
\colorlet{penColor5}{orange!80!black} % Color of a curve in a plot
                                      \colorlet{fill1}{blue!50!black!20} % Color of fill in a plot
\colorlet{fill2}{blue!10} % Color of fill in a plot
\colorlet{fillp}{fill1} % Color of positive area
\colorlet{filln}{red!50!black!20} % Color of negative area
\colorlet{gridColor}{gray!50} % Color of grid in a plot

\pgfmathdeclarefunction{gauss}{2}{% gives gaussian
  \pgfmathparse{1/(#2*sqrt(2*pi))*exp(-((x-#1)^2)/(2*#2^2))}%
}



\newcommand{\fullwidth}{}
\newcommand{\normalwidth}{}



%% makes a snazzy t-chart for evaluating functions
\newenvironment{tchart}{\rowcolors{2}{}{background!90!textColor}\array}{\endarray}

%%This is to help with formatting on future title pages.
\newenvironment{sectionOutcomes}{}{} 

\author{Emma Smith Zbarsky}
\license{Creative Commons Attribution 3.0 Unported}
\acknowledgement{https://quadbase.org/questions/q14253v1}
\begin{document}

\begin{exercise}

Compute the derivative with respect to $T$ of the ideal gas law:
\[V(T) = \frac{nRT}{P},\] where $V$ is the volume, $n$ is a constant
giving the number of moles of the gas, $R$ is the universal gas
constant, and $P$ is a constant pressure.

Note that $\frac{dV}{dT}$ expresses how the volume of the gas changes as
the temperature changes. You should particularly pay attention to the
sign of your result--does gas expand or contract as it is warmed up?


\begin{hint}
This is a simple power rule derivative. The complications all arise from
the use of actual named constants, $n$, $R$, and $P$.
\end{hint}


\begin{hint}
Because $n$, $R$, and $P$ are all constant terms, we could write this
problem as
\[\frac{dV}{dT} = \frac{d}{dT}\left(\left(\frac{nR}{P}\right)T\right).\]
Using the first linearity rule for constants, this becomes
\begin{align*}
\frac{dV}{dT} &= \left(\frac{nR}{P}\right)\frac{d}{dT}\left(T\right) \\
&= \left(\frac{nR}{P}\right)(1) \\
&= \boxed{\frac{nR}{P}}
\end{align*}
\end{hint}


\begin{multipleChoice}
\choice{$V'(T) = \frac{nT}{P}$}
\choice{$V'(T) = 0$}
\choice{$V'(T) = \frac{RT}{P}$}
\choice[correct]{$V'(T) = \frac{nR}{P}$}
\choice{$V'(T) = -\frac{nRT}{P^2}$}
\end{multipleChoice}

\end{exercise}
\end{document}
