\documentclass{ximera}
%\usepackage{todonotes}

\usepackage{tkz-euclide}
\usetikzlibrary{backgrounds} %% for boxes around graphs
\usetikzlibrary{shapes,positioning}  %% Clouds and stars
\usetkzobj{all}
\usepackage[makeroom]{cancel} %% for strike outs
%\usepackage{mathtools} %% for pretty underbrace % Breaks Ximera
\usepackage{multicol}


\newcommand{\RR}{\mathbb R}
\renewcommand{\d}{\,d}
\newcommand{\dd}[2][]{\frac{d #1}{d #2}}
\renewcommand{\l}{\ell}
\newcommand{\ddx}{\frac{d}{dx}}
\newcommand{\zeroOverZero}{$\boldsymbol{\tfrac{0}{0}}$}
\newcommand{\numOverZero}{$\boldsymbol{\tfrac{\#}{0}}$}
\newcommand{\dfn}{\textbf}
\newcommand{\eval}[1]{\bigg[ #1 \bigg]}
\renewcommand{\epsilon}{\varepsilon}
\renewcommand{\iff}{\Leftrightarrow}

\DeclareMathOperator{\arccot}{arccot}
\DeclareMathOperator{\arcsec}{arcsec}
\DeclareMathOperator{\arccsc}{arccsc}


\colorlet{textColor}{black} 
\colorlet{background}{white}
\colorlet{penColor}{blue!50!black} % Color of a curve in a plot
\colorlet{penColor2}{red!50!black}% Color of a curve in a plot
\colorlet{penColor3}{red!50!blue} % Color of a curve in a plot
\colorlet{penColor4}{green!50!black} % Color of a curve in a plot
\colorlet{penColor5}{orange!80!black} % Color of a curve in a plot
                                      \colorlet{fill1}{blue!50!black!20} % Color of fill in a plot
\colorlet{fill2}{blue!10} % Color of fill in a plot
\colorlet{fillp}{fill1} % Color of positive area
\colorlet{filln}{red!50!black!20} % Color of negative area
\colorlet{gridColor}{gray!50} % Color of grid in a plot

\pgfmathdeclarefunction{gauss}{2}{% gives gaussian
  \pgfmathparse{1/(#2*sqrt(2*pi))*exp(-((x-#1)^2)/(2*#2^2))}%
}



\newcommand{\fullwidth}{}
\newcommand{\normalwidth}{}



%% makes a snazzy t-chart for evaluating functions
\newenvironment{tchart}{\rowcolors{2}{}{background!90!textColor}\array}{\endarray}

%%This is to help with formatting on future title pages.
\newenvironment{sectionOutcomes}{}{} 

\author{Emma Smith Zbarsky}
\license{Creative Commons Attribution 3.0 Unported}
\acknowledgement{https://quadbase.org/questions/q14536v1}
\begin{document}

\begin{exercise}

Is there a solution of the equation $f(x) = x^5-4x^3+2x-3 = 0$ between
$x=0$ and $x=2$? Why or why not?


\begin{hint}
The mean value theorem does not give us anything useful here, but the
intermediate value theorem does.
\end{hint}


\begin{hint}
Yes, the Intermediate Value Theorem says that since
$f(0) = 0^5-4(0^4)+2(0)-3 = -3 < 0$ and
$f(2) = 2^5-4(2^3)+2(2)-3 = 1 > 0$ there must be a point $c \in (0,2)$
with $f(c) = c^5-4c^3+2c-3 = 0$.

The mean value theorem tells us about the derivative of a function,
while the intermediate value theorem tells us about the function itself.
\end{hint}


\begin{multipleChoice}
\choice[correct]{Yes, the Intermediate Value Theorem says that since
$f(0) = 0^5-4(0^4)+2(0)-3 = -3 < 0$ and
$f(2) = 2^5-4(2^3)+2(2)-3 = 1 > 0$ there must be a point $c \in (0,2)$
with $f(c) = c^5-4c^3+2c-3 = 0$.}
\choice{We cannot determine whether $f$ ever reaches 0 in the interval $(0,2)$.
The Mean Value Theorem says that since
\[\frac{f(2)-f(0)}{2-0} = \frac{-3-1}{2} = -2,\] there is a point
$c \in (0,2)$ where $f'(x) = -2$, but since this is only one point it
doesn't give us sufficient information to determine.}
\choice{No, the Mean Value Theorem says that since
\[\frac{f(2)-f(0)}{2-0} = \frac{-3-1}{2} = -2,\] there is a point
$c \in (0,2)$ where $f'(x) = -2$ so the negative slope and negative
starting value mean we never reach 0.}
\end{multipleChoice}

\end{exercise}
\end{document}
