\documentclass{ximera}
%\usepackage{todonotes}

\usepackage{tkz-euclide}
\usetikzlibrary{backgrounds} %% for boxes around graphs
\usetikzlibrary{shapes,positioning}  %% Clouds and stars
\usetkzobj{all}
\usepackage[makeroom]{cancel} %% for strike outs
%\usepackage{mathtools} %% for pretty underbrace % Breaks Ximera
\usepackage{multicol}


\newcommand{\RR}{\mathbb R}
\renewcommand{\d}{\,d}
\newcommand{\dd}[2][]{\frac{d #1}{d #2}}
\renewcommand{\l}{\ell}
\newcommand{\ddx}{\frac{d}{dx}}
\newcommand{\zeroOverZero}{$\boldsymbol{\tfrac{0}{0}}$}
\newcommand{\numOverZero}{$\boldsymbol{\tfrac{\#}{0}}$}
\newcommand{\dfn}{\textbf}
\newcommand{\eval}[1]{\bigg[ #1 \bigg]}
\renewcommand{\epsilon}{\varepsilon}
\renewcommand{\iff}{\Leftrightarrow}

\DeclareMathOperator{\arccot}{arccot}
\DeclareMathOperator{\arcsec}{arcsec}
\DeclareMathOperator{\arccsc}{arccsc}


\colorlet{textColor}{black} 
\colorlet{background}{white}
\colorlet{penColor}{blue!50!black} % Color of a curve in a plot
\colorlet{penColor2}{red!50!black}% Color of a curve in a plot
\colorlet{penColor3}{red!50!blue} % Color of a curve in a plot
\colorlet{penColor4}{green!50!black} % Color of a curve in a plot
\colorlet{penColor5}{orange!80!black} % Color of a curve in a plot
                                      \colorlet{fill1}{blue!50!black!20} % Color of fill in a plot
\colorlet{fill2}{blue!10} % Color of fill in a plot
\colorlet{fillp}{fill1} % Color of positive area
\colorlet{filln}{red!50!black!20} % Color of negative area
\colorlet{gridColor}{gray!50} % Color of grid in a plot

\pgfmathdeclarefunction{gauss}{2}{% gives gaussian
  \pgfmathparse{1/(#2*sqrt(2*pi))*exp(-((x-#1)^2)/(2*#2^2))}%
}



\newcommand{\fullwidth}{}
\newcommand{\normalwidth}{}



%% makes a snazzy t-chart for evaluating functions
\newenvironment{tchart}{\rowcolors{2}{}{background!90!textColor}\array}{\endarray}

%%This is to help with formatting on future title pages.
\newenvironment{sectionOutcomes}{}{} 

\author{Emma Smith Zbarsky}
\license{Creative Commons Attribution 3.0 Unported}
\acknowledgement{https://quadbase.org/questions/q14594v1}
\begin{document}

\begin{exercise}

Evaluate the left-endpoint Riemann sum with $\Delta x = .5$ of $y=x^3$
over the interval $[-1,1]$. How does your result compare to
$\int_{-1}^1 x^3 \; dx$?


\begin{hint}
This is a Riemann sum computation, so you want to set up your intervals
and evaluation points and then compute.
\end{hint}


\begin{hint}
We are computing a Riemann sum with $\Delta x = .5$ over the interval
$[-1,1]$. Therefore, our subintervals are:
\[[-1,-.5], [-.5,0], [0,.5], [.5, 1].\] Setting up a 4-interval
left-endpoint Riemann sum we have: \begin{align*}
\sum_{i=1}^4 f(x_{i-1}) \Delta x &= \sum_{i=1}^4 \left(x_{i-1}\right)^3 (.5) \\
&= .5\left((-1)^3+(-.5)^3+(0)^3+(.5)^3\right) \\
&= .5\left(-1-.5^3+0+.5^3\right) \\
&= .5(-1) = -.5
\end{align*} The definite integral, thinking geometrically, is clearly
0 over this interval by symmetry. If we use the Fundamental Theorem of
Calculus, however, we see that
\[\int_{-1}^1 x^3\; dx = \left.\frac{x^4}{4}\right|_{-1}^1 = \left(\frac{1^4}{4}-\frac{(-1)^4}{4}\right) = 0.\]
Thus, the left-endpoint Riemann approximation underestimates the
definite integral. That is because $x^3$ is an ``increasing'' function,
the first derivative is non-negative everywhere.
\end{hint}


\begin{multipleChoice}
\choice{$0$, equal to the area under the curve}
\choice{$.5$, larger than the area under the curve}
\choice{$-1$, smaller than the area under the curve}
\choice{$1$, larger than the area under the curve}
\choice[correct]{$-.5$, smaller than the area under the curve.}
\end{multipleChoice}

\end{exercise}
\end{document}
