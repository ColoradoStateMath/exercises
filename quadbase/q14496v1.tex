\documentclass{ximera}
%\usepackage{todonotes}

\usepackage{tkz-euclide}
\usetikzlibrary{backgrounds} %% for boxes around graphs
\usetikzlibrary{shapes,positioning}  %% Clouds and stars
\usetkzobj{all}
\usepackage[makeroom]{cancel} %% for strike outs
%\usepackage{mathtools} %% for pretty underbrace % Breaks Ximera
\usepackage{multicol}


\newcommand{\RR}{\mathbb R}
\renewcommand{\d}{\,d}
\newcommand{\dd}[2][]{\frac{d #1}{d #2}}
\renewcommand{\l}{\ell}
\newcommand{\ddx}{\frac{d}{dx}}
\newcommand{\zeroOverZero}{$\boldsymbol{\tfrac{0}{0}}$}
\newcommand{\numOverZero}{$\boldsymbol{\tfrac{\#}{0}}$}
\newcommand{\dfn}{\textbf}
\newcommand{\eval}[1]{\bigg[ #1 \bigg]}
\renewcommand{\epsilon}{\varepsilon}
\renewcommand{\iff}{\Leftrightarrow}

\DeclareMathOperator{\arccot}{arccot}
\DeclareMathOperator{\arcsec}{arcsec}
\DeclareMathOperator{\arccsc}{arccsc}


\colorlet{textColor}{black} 
\colorlet{background}{white}
\colorlet{penColor}{blue!50!black} % Color of a curve in a plot
\colorlet{penColor2}{red!50!black}% Color of a curve in a plot
\colorlet{penColor3}{red!50!blue} % Color of a curve in a plot
\colorlet{penColor4}{green!50!black} % Color of a curve in a plot
\colorlet{penColor5}{orange!80!black} % Color of a curve in a plot
                                      \colorlet{fill1}{blue!50!black!20} % Color of fill in a plot
\colorlet{fill2}{blue!10} % Color of fill in a plot
\colorlet{fillp}{fill1} % Color of positive area
\colorlet{filln}{red!50!black!20} % Color of negative area
\colorlet{gridColor}{gray!50} % Color of grid in a plot

\pgfmathdeclarefunction{gauss}{2}{% gives gaussian
  \pgfmathparse{1/(#2*sqrt(2*pi))*exp(-((x-#1)^2)/(2*#2^2))}%
}



\newcommand{\fullwidth}{}
\newcommand{\normalwidth}{}



%% makes a snazzy t-chart for evaluating functions
\newenvironment{tchart}{\rowcolors{2}{}{background!90!textColor}\array}{\endarray}

%%This is to help with formatting on future title pages.
\newenvironment{sectionOutcomes}{}{} 

\author{Emma Smith Zbarsky}
\license{Creative Commons Attribution 3.0 Unported}
\acknowledgement{https://quadbase.org/questions/q14496v1}
\begin{document}

\begin{exercise}

Consider the ideal gas law, \[PV = nRT.\] How fast is the temperature of
a balloon changing when the temperature of the gas inside is 326.15
degrees K, the pressure is 101325 N/m$^2$, $R=8.31432$ Nm/(mol K),
$V=.25$ m$^3$ and $n=345$ mols if you are inflating the balloon at 1
m$^3$/min while the pressure increases at a rate of $1.135$ N/m$^2$ per
minute?

Give your answer to the nearest tenth of a degree per minute.


\begin{hint}
This is a related rates problem with a given equation. Take the
derivative with respect to time and plug in the given values to compute
$\frac{dT}{dt}$.
\end{hint}


\begin{hint}
\begin{align*}
\frac{dP}{dt}V + P\frac{dV}{dt} &= nR\frac{dT}{dt} \\
(1.135)(.25) + 101325(1) &= 345(8.31432) \frac{dT}{dt} \\
\frac{dT}{dt} &\simeq 35.3
\end{align*} So the temperature is increasing at a rate of $35.3$
degrees Kelvin per minute.
\end{hint}


\begin{multipleChoice}
\choice{$35.5$ degrees K per minute}
\choice{$35.2$ degrees K per minute}
\choice{$35.4$ degrees K per minute}
\choice{$35.6$ degrees K per minute}
\choice[correct]{$35.3$ degrees K per minute}
\end{multipleChoice}

\end{exercise}
\end{document}
