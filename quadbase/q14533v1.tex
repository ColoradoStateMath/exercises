\documentclass{ximera}
%\usepackage{todonotes}

\usepackage{tkz-euclide}
\usetikzlibrary{backgrounds} %% for boxes around graphs
\usetikzlibrary{shapes,positioning}  %% Clouds and stars
\usetkzobj{all}
\usepackage[makeroom]{cancel} %% for strike outs
%\usepackage{mathtools} %% for pretty underbrace % Breaks Ximera
\usepackage{multicol}


\newcommand{\RR}{\mathbb R}
\renewcommand{\d}{\,d}
\newcommand{\dd}[2][]{\frac{d #1}{d #2}}
\renewcommand{\l}{\ell}
\newcommand{\ddx}{\frac{d}{dx}}
\newcommand{\zeroOverZero}{$\boldsymbol{\tfrac{0}{0}}$}
\newcommand{\numOverZero}{$\boldsymbol{\tfrac{\#}{0}}$}
\newcommand{\dfn}{\textbf}
\newcommand{\eval}[1]{\bigg[ #1 \bigg]}
\renewcommand{\epsilon}{\varepsilon}
\renewcommand{\iff}{\Leftrightarrow}

\DeclareMathOperator{\arccot}{arccot}
\DeclareMathOperator{\arcsec}{arcsec}
\DeclareMathOperator{\arccsc}{arccsc}


\colorlet{textColor}{black} 
\colorlet{background}{white}
\colorlet{penColor}{blue!50!black} % Color of a curve in a plot
\colorlet{penColor2}{red!50!black}% Color of a curve in a plot
\colorlet{penColor3}{red!50!blue} % Color of a curve in a plot
\colorlet{penColor4}{green!50!black} % Color of a curve in a plot
\colorlet{penColor5}{orange!80!black} % Color of a curve in a plot
                                      \colorlet{fill1}{blue!50!black!20} % Color of fill in a plot
\colorlet{fill2}{blue!10} % Color of fill in a plot
\colorlet{fillp}{fill1} % Color of positive area
\colorlet{filln}{red!50!black!20} % Color of negative area
\colorlet{gridColor}{gray!50} % Color of grid in a plot

\pgfmathdeclarefunction{gauss}{2}{% gives gaussian
  \pgfmathparse{1/(#2*sqrt(2*pi))*exp(-((x-#1)^2)/(2*#2^2))}%
}



\newcommand{\fullwidth}{}
\newcommand{\normalwidth}{}



%% makes a snazzy t-chart for evaluating functions
\newenvironment{tchart}{\rowcolors{2}{}{background!90!textColor}\array}{\endarray}

%%This is to help with formatting on future title pages.
\newenvironment{sectionOutcomes}{}{} 

\author{Emma Smith Zbarsky}
\license{Creative Commons Attribution 3.0 Unported}
\acknowledgement{https://quadbase.org/questions/q14533v1}
\begin{document}

\begin{exercise}

Find, and identify, the extrema of the unit step function
\[H(t) = \begin{cases} 1 & t > 0 \\
\frac{1}{2} & t = 0 \\ 0 & t < 0 \end{cases}.\]

`'Note: $H(t)$ is also called the Heaviside unit step function. It
arises frequently in solving differential equations.''


\begin{hint}
This is an extreme value problem where every point is a critical point,
so think carefully about what it means to be a maximum or a minimum.
\end{hint}


\begin{hint}
Because \[H'(t) = \begin{cases} 0 & t \neq 0 \\
DNE & t = 0 \end{cases},\] every point is a critical point for $H$. The
second derivative is the same, however, so it is not a helpful test.
Using our definition of maximum, we see that $H(t^+)=1\geq H(t)$ for any
$t^+ > 0$ and any real number $t$, so each $t^+$ is a maximum.
Similarly, $H(t^-) = 0 \leq H(t)$ for any $t^- < 0$ and any real number
$t$ so each $t^-$ is a minimum. This only misses $H(0) = \frac{1}{2}$,
which is larger than 0 but smaller than 1.
\end{hint}


\begin{multipleChoice}
\choice[correct]{Every point $t < 0$ is a (global) minimum and every point $t > 0$ is a
(global) maximum.}
\choice{Because every point is a critical point, every value $t$ is an extremum
for $H(t)$.}
\choice{There are no extrema for $H(t)$.}
\end{multipleChoice}

\end{exercise}
\end{document}
