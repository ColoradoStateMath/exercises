\documentclass{ximera}
%\usepackage{todonotes}

\usepackage{tkz-euclide}
\usetikzlibrary{backgrounds} %% for boxes around graphs
\usetikzlibrary{shapes,positioning}  %% Clouds and stars
\usetkzobj{all}
\usepackage[makeroom]{cancel} %% for strike outs
%\usepackage{mathtools} %% for pretty underbrace % Breaks Ximera
\usepackage{multicol}


\newcommand{\RR}{\mathbb R}
\renewcommand{\d}{\,d}
\newcommand{\dd}[2][]{\frac{d #1}{d #2}}
\renewcommand{\l}{\ell}
\newcommand{\ddx}{\frac{d}{dx}}
\newcommand{\zeroOverZero}{$\boldsymbol{\tfrac{0}{0}}$}
\newcommand{\numOverZero}{$\boldsymbol{\tfrac{\#}{0}}$}
\newcommand{\dfn}{\textbf}
\newcommand{\eval}[1]{\bigg[ #1 \bigg]}
\renewcommand{\epsilon}{\varepsilon}
\renewcommand{\iff}{\Leftrightarrow}

\DeclareMathOperator{\arccot}{arccot}
\DeclareMathOperator{\arcsec}{arcsec}
\DeclareMathOperator{\arccsc}{arccsc}


\colorlet{textColor}{black} 
\colorlet{background}{white}
\colorlet{penColor}{blue!50!black} % Color of a curve in a plot
\colorlet{penColor2}{red!50!black}% Color of a curve in a plot
\colorlet{penColor3}{red!50!blue} % Color of a curve in a plot
\colorlet{penColor4}{green!50!black} % Color of a curve in a plot
\colorlet{penColor5}{orange!80!black} % Color of a curve in a plot
                                      \colorlet{fill1}{blue!50!black!20} % Color of fill in a plot
\colorlet{fill2}{blue!10} % Color of fill in a plot
\colorlet{fillp}{fill1} % Color of positive area
\colorlet{filln}{red!50!black!20} % Color of negative area
\colorlet{gridColor}{gray!50} % Color of grid in a plot

\pgfmathdeclarefunction{gauss}{2}{% gives gaussian
  \pgfmathparse{1/(#2*sqrt(2*pi))*exp(-((x-#1)^2)/(2*#2^2))}%
}



\newcommand{\fullwidth}{}
\newcommand{\normalwidth}{}



%% makes a snazzy t-chart for evaluating functions
\newenvironment{tchart}{\rowcolors{2}{}{background!90!textColor}\array}{\endarray}

%%This is to help with formatting on future title pages.
\newenvironment{sectionOutcomes}{}{} 

\author{Emma Smith Zbarsky}
\license{Creative Commons Attribution 3.0 Unported}
\acknowledgement{https://quadbase.org/questions/q14537v1}
\begin{document}

\begin{exercise}

Find two values of $t$ that allow you to use the mean value theorem to
argue that there is a critical point between 0 and 4 for
$f(t) = \sin(3t)e^{t-5}-t^4(t-\pi)^3$.


\begin{hint}
We need to find two points on this curve which achieve the same exact
value. You need to be very careful because calculators will offer many
approximate solutions that are not exact for this purpose.
\end{hint}


\begin{hint}
One easy value to look for is always zero. In this case, if we start
with $t=0$ we have
\[f(0) = \sin(3(0))e^{0-5}-0^4(0-\pi)^3 = 0e^{-5}+0(\pi^3) = 0.\] There
are no values that will make $e^{t-5}$ be zero, but any multiple of
$\pi/3$ will make $\sin(3t)$ zero. In the other term, the zeros are
$t=0$ and $t=\pi$. The intersection of these two sets is $t=0$ and
$t=\pi$.

Therefore, there exists $0 < c < \pi$ such that
\[f'(c) = \frac{f(\pi)-f(0)}{\pi-0} = \frac{0-0}{\pi} = 0,\] and
therefore $c$ is a critical point of $f$.
\end{hint}


\begin{multipleChoice}
\choice{$t=0$ and $t=1$}
\choice{$t=0$ and $t=\pi/2$}
\choice[correct]{$t=0$ and $t=\pi$}
\choice{$t=0$ and $t=5$}
\end{multipleChoice}

\end{exercise}
\end{document}
