\documentclass{ximera}
\usepackage{todonotes}

\usepackage{tkz-euclide}
\usetikzlibrary{backgrounds} %% for boxes around graphs
\usetikzlibrary{shapes,positioning}  %% Clouds and stars
\usetkzobj{all}
\usepackage[makeroom]{cancel} %% for strike outs
%\usepackage{mathtools} %% for pretty underbrace % Breaks Ximera
\usepackage{multicol}


\newcommand{\RR}{\mathbb R}
\renewcommand{\d}{\,d}
\newcommand{\dd}[2][]{\frac{d #1}{d #2}}
\renewcommand{\l}{\ell}
\newcommand{\ddx}{\frac{d}{dx}}
\newcommand{\zeroOverZero}{$\boldsymbol{\tfrac{0}{0}}$}
\newcommand{\numOverZero}{$\boldsymbol{\tfrac{\#}{0}}$}
\newcommand{\dfn}{\textbf}
\newcommand{\eval}[1]{\bigg[ #1 \bigg]}
\renewcommand{\epsilon}{\varepsilon}
\renewcommand{\iff}{\Leftrightarrow}

\DeclareMathOperator{\arccot}{arccot}
\DeclareMathOperator{\arcsec}{arcsec}
\DeclareMathOperator{\arccsc}{arccsc}


\colorlet{textColor}{black} 
\colorlet{background}{white}
\colorlet{penColor}{blue!50!black} % Color of a curve in a plot
\colorlet{penColor2}{red!50!black}% Color of a curve in a plot
\colorlet{penColor3}{red!50!blue} % Color of a curve in a plot
\colorlet{penColor4}{green!50!black} % Color of a curve in a plot
\colorlet{penColor5}{orange!80!black} % Color of a curve in a plot
                                      \colorlet{fill1}{blue!50!black!20} % Color of fill in a plot
\colorlet{fill2}{blue!10} % Color of fill in a plot
\colorlet{fillp}{fill1} % Color of positive area
\colorlet{filln}{red!50!black!20} % Color of negative area
\colorlet{gridColor}{gray!50} % Color of grid in a plot

\pgfmathdeclarefunction{gauss}{2}{% gives gaussian
  \pgfmathparse{1/(#2*sqrt(2*pi))*exp(-((x-#1)^2)/(2*#2^2))}%
}



\newcommand{\fullwidth}{}
\newcommand{\normalwidth}{}



%% makes a snazzy t-chart for evaluating functions
\newenvironment{tchart}{\rowcolors{2}{}{background!90!textColor}\array}{\endarray}

%%This is to help with formatting on future title pages.
\newenvironment{sectionOutcomes}{}{} 

\author{Steven Gubkin}
\license{Creative Commons 3.0 By-NC}
\begin{document}

\begin{exercise}

\tag{integral}

A surface has is formed by revolving the curve $y = \frac{1}{x}$ from $x=1$ to $x=4$ about the $x$ -axis.  The $x$ axis is measuring distances in meters.  This surface has a variable density of $x^2 \unit{g}/\unit{m}^2$.  Set up and numerically evaluate an integral to find the mass of the surface in grams.  Report your answer to two decimal places.

\begin{hint}
	Accumulate the mass of infinitely many infinitesimal frustums.
\end{hint}

\begin{hint}
	The area of the frustum based at $x$ and with width $dx$ is $2\pi \frac{1}{x} \sqrt{1+\left(\frac{\d}{\d x} x^{-1}\right)^2} \d x = 2\pi \frac{1}{x} \sqrt{1+\frac{1}{x^2}} \d x$
\end{hint}

\begin{hint}
	Since the density of this frustum is $x^2$, the total mass of this frustum is $2 \pi x \sqrt{1+\frac{1}{x^2}} \d x$
\end{hint}

\begin{hint}
	We need to accumulate these infinitesimal masses from $x=1$ to $x=4$, so the total mass is
	
	\[
	\int_1^4 2 \pi x \sqrt{1+\frac{1}{x^2}} \d x \approx 51.18
	\]
\end{hint}

\begin{prompt}
	\[
	\textrm{Total Mass} = \answer{51.18} \unit{g}
	\]
\end{prompt}
\end{exercise}
\end{document}