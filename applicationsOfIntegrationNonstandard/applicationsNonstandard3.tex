\documentclass{ximera}
\usepackage{todonotes}

\usepackage{tkz-euclide}
\usetikzlibrary{backgrounds} %% for boxes around graphs
\usetikzlibrary{shapes,positioning}  %% Clouds and stars
\usetkzobj{all}
\usepackage[makeroom]{cancel} %% for strike outs
%\usepackage{mathtools} %% for pretty underbrace % Breaks Ximera
\usepackage{multicol}


\newcommand{\RR}{\mathbb R}
\renewcommand{\d}{\,d}
\newcommand{\dd}[2][]{\frac{d #1}{d #2}}
\renewcommand{\l}{\ell}
\newcommand{\ddx}{\frac{d}{dx}}
\newcommand{\zeroOverZero}{$\boldsymbol{\tfrac{0}{0}}$}
\newcommand{\numOverZero}{$\boldsymbol{\tfrac{\#}{0}}$}
\newcommand{\dfn}{\textbf}
\newcommand{\eval}[1]{\bigg[ #1 \bigg]}
\renewcommand{\epsilon}{\varepsilon}
\renewcommand{\iff}{\Leftrightarrow}

\DeclareMathOperator{\arccot}{arccot}
\DeclareMathOperator{\arcsec}{arcsec}
\DeclareMathOperator{\arccsc}{arccsc}


\colorlet{textColor}{black} 
\colorlet{background}{white}
\colorlet{penColor}{blue!50!black} % Color of a curve in a plot
\colorlet{penColor2}{red!50!black}% Color of a curve in a plot
\colorlet{penColor3}{red!50!blue} % Color of a curve in a plot
\colorlet{penColor4}{green!50!black} % Color of a curve in a plot
\colorlet{penColor5}{orange!80!black} % Color of a curve in a plot
                                      \colorlet{fill1}{blue!50!black!20} % Color of fill in a plot
\colorlet{fill2}{blue!10} % Color of fill in a plot
\colorlet{fillp}{fill1} % Color of positive area
\colorlet{filln}{red!50!black!20} % Color of negative area
\colorlet{gridColor}{gray!50} % Color of grid in a plot

\pgfmathdeclarefunction{gauss}{2}{% gives gaussian
  \pgfmathparse{1/(#2*sqrt(2*pi))*exp(-((x-#1)^2)/(2*#2^2))}%
}



\newcommand{\fullwidth}{}
\newcommand{\normalwidth}{}



%% makes a snazzy t-chart for evaluating functions
\newenvironment{tchart}{\rowcolors{2}{}{background!90!textColor}\array}{\endarray}

%%This is to help with formatting on future title pages.
\newenvironment{sectionOutcomes}{}{} 

\author{Steven Gubkin}
\license{Creative Commons 3.0 By-NC}
\begin{document}

\begin{exercise}

\tag{integral}

The Van der Waals force between two atoms is proportional to $r^6$, where $r$ is the distance between them.  This force, while small, is responsible for many interesting phenomena, including the ability of geckos to cling to glass.

Assume that two atoms obey $F(r) = \frac{Q}{r^6}$, where $r$ is the distance measured in nanometers ($\unit{nm}$).

How much work is done, in nanoNewtons ($\unit{nN}$), by moving the atoms from a distance of  $3 \unit{nm}$ to $5 \unit{nm}$?

\begin{hint}
The work done by moving the atoms $\d x \unit{nm}$ when they are $x \unit{nm}$ apart is $\d W = \frac{c}{x^6} \d x$.
\end{hint}

\begin{hint}
Accumulating these infinitesimal amounts of work from $x = 3$ to $x = 5$ yields

\begin{align*}
\textrm{Work} &=  \int_3^5 \frac{Q}{x^6} \d x\\
	&= \eval{\frac{-Q}{7x^7}}_3^5\\
	&=\frac{Q}{7(3^7)} - \frac{Q}{7(5)^5}\\
	&\approx -0.0005860609 Q\\
	&\approx 5.86 \times 10^{-4} Q
\end{align*}
\end{hint}

\begin{prompt}
	Express your answer is scientific notation, where the mantissa has two decimal places.
	
	\[
	\textrm{Work} = \answer{5.84} \times 10^{\answer{4}} Q \unit{nN}
	\]
\end{prompt}

\end{exercise}
\end{document}