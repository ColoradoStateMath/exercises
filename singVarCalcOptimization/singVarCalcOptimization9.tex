\documentclass{ximera}
\usepackage{todonotes}

\usepackage{tkz-euclide}
\usetikzlibrary{backgrounds} %% for boxes around graphs
\usetikzlibrary{shapes,positioning}  %% Clouds and stars
\usetkzobj{all}
\usepackage[makeroom]{cancel} %% for strike outs
%\usepackage{mathtools} %% for pretty underbrace % Breaks Ximera
\usepackage{multicol}


\newcommand{\RR}{\mathbb R}
\renewcommand{\d}{\,d}
\newcommand{\dd}[2][]{\frac{d #1}{d #2}}
\renewcommand{\l}{\ell}
\newcommand{\ddx}{\frac{d}{dx}}
\newcommand{\zeroOverZero}{$\boldsymbol{\tfrac{0}{0}}$}
\newcommand{\numOverZero}{$\boldsymbol{\tfrac{\#}{0}}$}
\newcommand{\dfn}{\textbf}
\newcommand{\eval}[1]{\bigg[ #1 \bigg]}
\renewcommand{\epsilon}{\varepsilon}
\renewcommand{\iff}{\Leftrightarrow}

\DeclareMathOperator{\arccot}{arccot}
\DeclareMathOperator{\arcsec}{arcsec}
\DeclareMathOperator{\arccsc}{arccsc}


\colorlet{textColor}{black} 
\colorlet{background}{white}
\colorlet{penColor}{blue!50!black} % Color of a curve in a plot
\colorlet{penColor2}{red!50!black}% Color of a curve in a plot
\colorlet{penColor3}{red!50!blue} % Color of a curve in a plot
\colorlet{penColor4}{green!50!black} % Color of a curve in a plot
\colorlet{penColor5}{orange!80!black} % Color of a curve in a plot
                                      \colorlet{fill1}{blue!50!black!20} % Color of fill in a plot
\colorlet{fill2}{blue!10} % Color of fill in a plot
\colorlet{fillp}{fill1} % Color of positive area
\colorlet{filln}{red!50!black!20} % Color of negative area
\colorlet{gridColor}{gray!50} % Color of grid in a plot

\pgfmathdeclarefunction{gauss}{2}{% gives gaussian
  \pgfmathparse{1/(#2*sqrt(2*pi))*exp(-((x-#1)^2)/(2*#2^2))}%
}



\newcommand{\fullwidth}{}
\newcommand{\normalwidth}{}



%% makes a snazzy t-chart for evaluating functions
\newenvironment{tchart}{\rowcolors{2}{}{background!90!textColor}\array}{\endarray}

%%This is to help with formatting on future title pages.
\newenvironment{sectionOutcomes}{}{} 

\author{Bart Snapp}
\license{Creative Commons 3.0 By-NC}
\acknowledgement{https://www.whitman.edu/mathematics/calculus/}
\begin{document}
\begin{exercise}
  \outcome{Interpret an optimization problem as the procedure used to make a system or design as effective or functional as possible.}
  \outcome{Set up an optimization problem by identifying the objective function and appropriate constraints.}
  \outcome{Solve optimization problems by finding the appropriate absolute extremum.}
  \outcome{Solve basic word problems involving maxima or minima.}

  \tag{extrema}
  \tag{constrained optimization}
  \tag{first derivative test}
  \tag{second derivative test}

  You want to make cylindrical containers to hold 1 liter using the
  least amount of construction material.  The side is made from a
  rectangular piece of material, and this can be done with no material
  wasted.  However, the top and bottom are cut from squares of side
  $2r$, so that $2(2r)^2=8r^2$ of material is needed (rather than
  $2\pi r^2$, which is the total area of the top and bottom).  Find
  the dimensions of the container using the least amount of material.
  \begin{prompt}
  \[
  \text{radius}=\answer{5}\qquad\text{height}=\answer{40/\pi}
  \]
  \end{prompt}
\end{exercise}
\end{document}
