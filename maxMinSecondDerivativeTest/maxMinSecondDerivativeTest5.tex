\documentclass{ximera}
\usepackage{todonotes}

\usepackage{tkz-euclide}
\usetikzlibrary{backgrounds} %% for boxes around graphs
\usetikzlibrary{shapes,positioning}  %% Clouds and stars
\usetkzobj{all}
\usepackage[makeroom]{cancel} %% for strike outs
%\usepackage{mathtools} %% for pretty underbrace % Breaks Ximera
\usepackage{multicol}


\newcommand{\RR}{\mathbb R}
\renewcommand{\d}{\,d}
\newcommand{\dd}[2][]{\frac{d #1}{d #2}}
\renewcommand{\l}{\ell}
\newcommand{\ddx}{\frac{d}{dx}}
\newcommand{\zeroOverZero}{$\boldsymbol{\tfrac{0}{0}}$}
\newcommand{\numOverZero}{$\boldsymbol{\tfrac{\#}{0}}$}
\newcommand{\dfn}{\textbf}
\newcommand{\eval}[1]{\bigg[ #1 \bigg]}
\renewcommand{\epsilon}{\varepsilon}
\renewcommand{\iff}{\Leftrightarrow}

\DeclareMathOperator{\arccot}{arccot}
\DeclareMathOperator{\arcsec}{arcsec}
\DeclareMathOperator{\arccsc}{arccsc}


\colorlet{textColor}{black} 
\colorlet{background}{white}
\colorlet{penColor}{blue!50!black} % Color of a curve in a plot
\colorlet{penColor2}{red!50!black}% Color of a curve in a plot
\colorlet{penColor3}{red!50!blue} % Color of a curve in a plot
\colorlet{penColor4}{green!50!black} % Color of a curve in a plot
\colorlet{penColor5}{orange!80!black} % Color of a curve in a plot
                                      \colorlet{fill1}{blue!50!black!20} % Color of fill in a plot
\colorlet{fill2}{blue!10} % Color of fill in a plot
\colorlet{fillp}{fill1} % Color of positive area
\colorlet{filln}{red!50!black!20} % Color of negative area
\colorlet{gridColor}{gray!50} % Color of grid in a plot

\pgfmathdeclarefunction{gauss}{2}{% gives gaussian
  \pgfmathparse{1/(#2*sqrt(2*pi))*exp(-((x-#1)^2)/(2*#2^2))}%
}



\newcommand{\fullwidth}{}
\newcommand{\normalwidth}{}



%% makes a snazzy t-chart for evaluating functions
\newenvironment{tchart}{\rowcolors{2}{}{background!90!textColor}\array}{\endarray}

%%This is to help with formatting on future title pages.
\newenvironment{sectionOutcomes}{}{} 

\author{Steven Gubkin}
\license{Creative Commons 3.0 By-NC}
\begin{document}

\begin{exercise}
\outcome{ Define a critical point.}
\outcome{ Find critical points.}
\outcome{ Define absolute maximum and absolute minimum.}
\outcome{ Find the absolute max or min of a continuous function on a closed interval.}
\outcome{ Define local maximum and local minimum.}
\outcome{ Compare and contrast local and absolute maxima and minima.}
\outcome{ Identify situations in which an absolute maximum or minimum is guaranteed.}
\outcome{ Classify critical points.}
\outcome{ State the First Derivative Test.}
\outcome{ Apply the First Derivative Test.}
\outcome{ State the Second Derivative Test.}
\outcome{ Apply the Second Derivative Test.}
\tag{derivative}

The function $f(x) = x^4-4x^3+16x-3$ has two critical points. If we
call these critical points $a$ and $b$, and order them such that $a <
b$, then

$$
a = \answer{-1}
$$

$$
b=\answer{2}
$$

At $x=a$, the second derivative test
\begin{multipleChoice}
\choice{Indicates a local maxima}
\choice[correct]{Indicates a local minima}
\choice{Fails, but the First derivative test indicates a local max}
\choice{Fails, but the First derivative test indicates a local min}
\choice{Fails, and the First derivative test indicates that it is not a local extrema}
\end{multipleChoice}

At $x=b$, the second derivative test
\begin{multipleChoice}
\choice{Indicates a local maxima}
\choice{Indicates a local minima}
\choice{Fails, but the First derivative test indicates a local max}
\choice{Fails, but the First derivative test indicates a local min}
\choice[correct]{Fails, and the First derivative test indicates that it is not a local extrema}
\end{multipleChoice}

\end{exercise}
\end{document}

