\documentclass{ximera}

\usepackage{todonotes}

\usepackage{tkz-euclide}
\usetikzlibrary{backgrounds} %% for boxes around graphs
\usetikzlibrary{shapes,positioning}  %% Clouds and stars
\usetkzobj{all}
\usepackage[makeroom]{cancel} %% for strike outs
%\usepackage{mathtools} %% for pretty underbrace % Breaks Ximera
\usepackage{multicol}


\newcommand{\RR}{\mathbb R}
\renewcommand{\d}{\,d}
\newcommand{\dd}[2][]{\frac{d #1}{d #2}}
\renewcommand{\l}{\ell}
\newcommand{\ddx}{\frac{d}{dx}}
\newcommand{\zeroOverZero}{$\boldsymbol{\tfrac{0}{0}}$}
\newcommand{\numOverZero}{$\boldsymbol{\tfrac{\#}{0}}$}
\newcommand{\dfn}{\textbf}
\newcommand{\eval}[1]{\bigg[ #1 \bigg]}
\renewcommand{\epsilon}{\varepsilon}
\renewcommand{\iff}{\Leftrightarrow}

\DeclareMathOperator{\arccot}{arccot}
\DeclareMathOperator{\arcsec}{arcsec}
\DeclareMathOperator{\arccsc}{arccsc}


\colorlet{textColor}{black} 
\colorlet{background}{white}
\colorlet{penColor}{blue!50!black} % Color of a curve in a plot
\colorlet{penColor2}{red!50!black}% Color of a curve in a plot
\colorlet{penColor3}{red!50!blue} % Color of a curve in a plot
\colorlet{penColor4}{green!50!black} % Color of a curve in a plot
\colorlet{penColor5}{orange!80!black} % Color of a curve in a plot
                                      \colorlet{fill1}{blue!50!black!20} % Color of fill in a plot
\colorlet{fill2}{blue!10} % Color of fill in a plot
\colorlet{fillp}{fill1} % Color of positive area
\colorlet{filln}{red!50!black!20} % Color of negative area
\colorlet{gridColor}{gray!50} % Color of grid in a plot

\pgfmathdeclarefunction{gauss}{2}{% gives gaussian
  \pgfmathparse{1/(#2*sqrt(2*pi))*exp(-((x-#1)^2)/(2*#2^2))}%
}



\newcommand{\fullwidth}{}
\newcommand{\normalwidth}{}



%% makes a snazzy t-chart for evaluating functions
\newenvironment{tchart}{\rowcolors{2}{}{background!90!textColor}\array}{\endarray}

%%This is to help with formatting on future title pages.
\newenvironment{sectionOutcomes}{}{} 


\author{Gregory Hartman \and Matthew Carr}
\license{Creative Commons 3.0 By-NC}
\acknowledgement{https://github.com/APEXCalculus}

\begin{document}
\begin{exercise}

\outcome{Compute basic antiderivatives.}
\outcome{Use integral notation for both antiderivatives and definite integrals.}
\outcome{Undo the Chain Rule.}
\outcome{Calculate indefinite integrals (antiderivatives) using basic substitution.}

%% BADBAD 
%% Absolute values here?

\tag{antiderivatives}
\tag{integral}
\tag{substitution}

Find the indefinite integral of $\frac{x^2+5 x-2}{x^2-10 x+32}$ with respect to $x$.

\[
\int \frac{x^2+5 x-2}{x^2-10 x+32}\d x=\answer{\frac{15}{2} \ln|x^2-10 x+32|+x+\frac{41\arctan(\frac{x-5}{\sqrt{7}})}{\sqrt{7}}+C}
\]

%%\begin{hint}
%%Notice that  $(x^2-10x+32)+(15x-34)=x^2+5x-2$, so we can rewrite the integrand as $1+\frac{15x-34}{x^2-10 x+32}$.
%%\end{hint}
%%\begin{hint}
%%Complete the square for $x^2-10 x+32$ (i.e., find constants, $a$, $b$, and $c$ such that $x^2-10 x+32=a(x-b)^2+c$. What does this suggest as a choice for substitution?
%%\end{hint}
%%\begin{hint}
%%Clearly $a=1$. Expanding the rest and setting the result equal to $x^2-10x+32$, we have $b^2+c=32$ and $-2b=-10$, so $b=5$ and $c=7$. This suggests that we choose $u=x-5$.
%%\end{hint}
%%\begin{hint}
%%In particular, $du=dx$ and $x=u+5$, so we can rewrite $1+\frac{15x-34}{x^2-10 x+32}$ as $1+\frac{41+15u}{u^2+7}=1+\frac{41}{u^2+7}+\frac{15u}{u^2+7}$. Now make the usual trigonometric substitution and integrate.
%%\end{hint}
%%\begin{hint}
%%Let $u=\sqrt{7}\tan(\theta)$, then $du=\sqrt{7}\sec^2(\theta)d\theta$ and so we are left with $\int\left(1+\frac{41}{u^2+7}+\frac{15u}{u^2+7})\d u=\int\frac{48}{\sqrt{7}}+15\tan(\theta)+\sqrt{7}\tan^2(\theta)\d \theta$. Since $\tan^2(\theta)=\sec^2(\theta)+1$, we know the integral of each of these and we combine them to $\frac{48 \theta }{\sqrt{7}}-15 \ln(\cos(\theta))-\sqrt{7}\theta+\sqrt{7} \tan(\theta)$ and since $u=\sqrt{7}\tan(\theta)$, it follows that $\theta=\arctan(\frac{u}{\sqrt{7}})$ and since $u=x-5$, this is $\theta=\arctan(\frac{x-5}{\sqrt{7}})$.
%%\end{hint}
\end{exercise}
\end{document}