\documentclass{ximera}
\usepackage{todonotes}

\usepackage{tkz-euclide}
\usetikzlibrary{backgrounds} %% for boxes around graphs
\usetikzlibrary{shapes,positioning}  %% Clouds and stars
\usetkzobj{all}
\usepackage[makeroom]{cancel} %% for strike outs
%\usepackage{mathtools} %% for pretty underbrace % Breaks Ximera
\usepackage{multicol}


\newcommand{\RR}{\mathbb R}
\renewcommand{\d}{\,d}
\newcommand{\dd}[2][]{\frac{d #1}{d #2}}
\renewcommand{\l}{\ell}
\newcommand{\ddx}{\frac{d}{dx}}
\newcommand{\zeroOverZero}{$\boldsymbol{\tfrac{0}{0}}$}
\newcommand{\numOverZero}{$\boldsymbol{\tfrac{\#}{0}}$}
\newcommand{\dfn}{\textbf}
\newcommand{\eval}[1]{\bigg[ #1 \bigg]}
\renewcommand{\epsilon}{\varepsilon}
\renewcommand{\iff}{\Leftrightarrow}

\DeclareMathOperator{\arccot}{arccot}
\DeclareMathOperator{\arcsec}{arcsec}
\DeclareMathOperator{\arccsc}{arccsc}


\colorlet{textColor}{black} 
\colorlet{background}{white}
\colorlet{penColor}{blue!50!black} % Color of a curve in a plot
\colorlet{penColor2}{red!50!black}% Color of a curve in a plot
\colorlet{penColor3}{red!50!blue} % Color of a curve in a plot
\colorlet{penColor4}{green!50!black} % Color of a curve in a plot
\colorlet{penColor5}{orange!80!black} % Color of a curve in a plot
                                      \colorlet{fill1}{blue!50!black!20} % Color of fill in a plot
\colorlet{fill2}{blue!10} % Color of fill in a plot
\colorlet{fillp}{fill1} % Color of positive area
\colorlet{filln}{red!50!black!20} % Color of negative area
\colorlet{gridColor}{gray!50} % Color of grid in a plot

\pgfmathdeclarefunction{gauss}{2}{% gives gaussian
  \pgfmathparse{1/(#2*sqrt(2*pi))*exp(-((x-#1)^2)/(2*#2^2))}%
}



\newcommand{\fullwidth}{}
\newcommand{\normalwidth}{}



%% makes a snazzy t-chart for evaluating functions
\newenvironment{tchart}{\rowcolors{2}{}{background!90!textColor}\array}{\endarray}

%%This is to help with formatting on future title pages.
\newenvironment{sectionOutcomes}{}{} 

\author{Steven Gubkin}
\license{Creative Commons 3.0 By-NC}
\begin{document}

\begin{exercise}

\outcome{Solve basic related rates word problems.}
\outcome{Understand the process of solving related rates problems.}
\outcome{Calculate derivatives of expressions with multiple variables implicitly.}

\tag{derivative}

A right circular cone is growing.  As it grows, its height is remains equal to its diameter. The diameter is growing at a constant rate of $2 \frac{\textrm{units}}{\textrm{s}}$. At what rate is its volume growing at the time the diameter is $6 \textrm{units}$?


\begin{hint}
	If we let $V$be the volume, $r$ the radius, and $D$ the diameter,  and $h$ the height, then we know

\[
\begin{cases}
	D = 2r\\
	V = \frac{1}{3} \pi r^2 h\\
	h = D\\
\end{cases}
\]
\end{hint}

\begin{hint}
	While we could blindly differentiate, and everything would work out fine, it pays to do a little algebra first since we only care about the relationship between the volume and the diameter.  A little rearranging yields

\[
V = \frac{1}{3} \pi \left(\frac{D}{2}\right)^2D
\]

so

\[
V = \frac{1}{12} \pi D^3
\]
\end{hint}

\begin{hint}
	Differentiating with respect to time, we obtain

\[\frac{\d V}{\d t} = \frac{1}{4} \pi D^2 \frac{\d D}{\d t}\]
\end{hint}

\begin{hint}
	At the time of interest, we have

\[\frac{\d V}{\d t} = \frac{1}{4} \pi (6) (2) = 3\pi\]
\end{hint}

\begin{prompt}
	\[\frac{\d V}{\d t} = \answer{ 3\pi}\]
\end{prompt}

\end{exercise}

\end{document}
