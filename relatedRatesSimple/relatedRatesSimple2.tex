\documentclass{ximera}
\usepackage{todonotes}

\usepackage{tkz-euclide}
\usetikzlibrary{backgrounds} %% for boxes around graphs
\usetikzlibrary{shapes,positioning}  %% Clouds and stars
\usetkzobj{all}
\usepackage[makeroom]{cancel} %% for strike outs
%\usepackage{mathtools} %% for pretty underbrace % Breaks Ximera
\usepackage{multicol}


\newcommand{\RR}{\mathbb R}
\renewcommand{\d}{\,d}
\newcommand{\dd}[2][]{\frac{d #1}{d #2}}
\renewcommand{\l}{\ell}
\newcommand{\ddx}{\frac{d}{dx}}
\newcommand{\zeroOverZero}{$\boldsymbol{\tfrac{0}{0}}$}
\newcommand{\numOverZero}{$\boldsymbol{\tfrac{\#}{0}}$}
\newcommand{\dfn}{\textbf}
\newcommand{\eval}[1]{\bigg[ #1 \bigg]}
\renewcommand{\epsilon}{\varepsilon}
\renewcommand{\iff}{\Leftrightarrow}

\DeclareMathOperator{\arccot}{arccot}
\DeclareMathOperator{\arcsec}{arcsec}
\DeclareMathOperator{\arccsc}{arccsc}


\colorlet{textColor}{black} 
\colorlet{background}{white}
\colorlet{penColor}{blue!50!black} % Color of a curve in a plot
\colorlet{penColor2}{red!50!black}% Color of a curve in a plot
\colorlet{penColor3}{red!50!blue} % Color of a curve in a plot
\colorlet{penColor4}{green!50!black} % Color of a curve in a plot
\colorlet{penColor5}{orange!80!black} % Color of a curve in a plot
                                      \colorlet{fill1}{blue!50!black!20} % Color of fill in a plot
\colorlet{fill2}{blue!10} % Color of fill in a plot
\colorlet{fillp}{fill1} % Color of positive area
\colorlet{filln}{red!50!black!20} % Color of negative area
\colorlet{gridColor}{gray!50} % Color of grid in a plot

\pgfmathdeclarefunction{gauss}{2}{% gives gaussian
  \pgfmathparse{1/(#2*sqrt(2*pi))*exp(-((x-#1)^2)/(2*#2^2))}%
}



\newcommand{\fullwidth}{}
\newcommand{\normalwidth}{}



%% makes a snazzy t-chart for evaluating functions
\newenvironment{tchart}{\rowcolors{2}{}{background!90!textColor}\array}{\endarray}

%%This is to help with formatting on future title pages.
\newenvironment{sectionOutcomes}{}{} 

\author{Steven Gubkin}
\license{Creative Commons 3.0 By-NC}
\begin{document}

\begin{exercise}

\outcome{Solve basic related rates word problems.}
\outcome{Understand the process of solving related rates problems.}
\outcome{Calculate derivatives of expressions with multiple variables implicitly.}

\tag{derivative}

The hypotenuse of an isosceles right triangle is increasing at a rate of $2 \frac{\textrm{m}}{\textrm{s}}$.

At what rate is the area of this triangle increasing when the hypotenuse has length $4 \textrm{m}$?

\begin{hint}
	Let $s$ be the length of the legs, $h$ be the length of the hypotenuse, and $A$ be the area. 

	Then we know that

\[
\begin{cases}
	2s^2 = h^2\\
	A=\frac{s^2}{2}
\end{cases}
\]

\end{hint}

\begin{hint}
	There are a lot of ways to procede.  We could just blindly differentiate, but a little algebra first might clean things up.  In particular, we can notice that the two equations imply that

\[
A = \frac{1}{4} h^2
\]
\end{hint}

\begin{hint}
	Differentiating this implicitly with respect to $t$, we have

\[
\frac{\d A}{\d t} = \frac{1}{2} h \frac{\d h}{\d t}
\]
\end{hint}

\begin{hint}
	Substituting what we know about these quantities at the time in question, we have

$\frac{\d A}{\d t} = \frac{1}{2} (4) (2) = 4$
\end{hint}

\begin{prompt}
	$\frac{\d A}{\d t} = \answer{4}$
\end{prompt}

\end{exercise}

\end{document}
