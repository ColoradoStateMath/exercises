\documentclass{ximera}
\usepackage{todonotes}

\usepackage{tkz-euclide}
\usetikzlibrary{backgrounds} %% for boxes around graphs
\usetikzlibrary{shapes,positioning}  %% Clouds and stars
\usetkzobj{all}
\usepackage[makeroom]{cancel} %% for strike outs
%\usepackage{mathtools} %% for pretty underbrace % Breaks Ximera
\usepackage{multicol}


\newcommand{\RR}{\mathbb R}
\renewcommand{\d}{\,d}
\newcommand{\dd}[2][]{\frac{d #1}{d #2}}
\renewcommand{\l}{\ell}
\newcommand{\ddx}{\frac{d}{dx}}
\newcommand{\zeroOverZero}{$\boldsymbol{\tfrac{0}{0}}$}
\newcommand{\numOverZero}{$\boldsymbol{\tfrac{\#}{0}}$}
\newcommand{\dfn}{\textbf}
\newcommand{\eval}[1]{\bigg[ #1 \bigg]}
\renewcommand{\epsilon}{\varepsilon}
\renewcommand{\iff}{\Leftrightarrow}

\DeclareMathOperator{\arccot}{arccot}
\DeclareMathOperator{\arcsec}{arcsec}
\DeclareMathOperator{\arccsc}{arccsc}


\colorlet{textColor}{black} 
\colorlet{background}{white}
\colorlet{penColor}{blue!50!black} % Color of a curve in a plot
\colorlet{penColor2}{red!50!black}% Color of a curve in a plot
\colorlet{penColor3}{red!50!blue} % Color of a curve in a plot
\colorlet{penColor4}{green!50!black} % Color of a curve in a plot
\colorlet{penColor5}{orange!80!black} % Color of a curve in a plot
                                      \colorlet{fill1}{blue!50!black!20} % Color of fill in a plot
\colorlet{fill2}{blue!10} % Color of fill in a plot
\colorlet{fillp}{fill1} % Color of positive area
\colorlet{filln}{red!50!black!20} % Color of negative area
\colorlet{gridColor}{gray!50} % Color of grid in a plot

\pgfmathdeclarefunction{gauss}{2}{% gives gaussian
  \pgfmathparse{1/(#2*sqrt(2*pi))*exp(-((x-#1)^2)/(2*#2^2))}%
}



\newcommand{\fullwidth}{}
\newcommand{\normalwidth}{}



%% makes a snazzy t-chart for evaluating functions
\newenvironment{tchart}{\rowcolors{2}{}{background!90!textColor}\array}{\endarray}

%%This is to help with formatting on future title pages.
\newenvironment{sectionOutcomes}{}{} 

\author{Steven Gubkin}
\license{Creative Commons 3.0 By-NC}
\begin{document}

\begin{exercise}

\outcome{Solve basic related rates word problems.}
\outcome{Understand the process of solving related rates problems.}
\outcome{Calculate derivatives of expressions with multiple variables implicitly.}

\tag{derivative}

A sphere is expanding with time at a constant rate of $3 \frac{\textrm{m}^3}{\textrm{s}}$.  At a certain time, the radius  is $5 \textrm{m}$.  How fast is the surface area growing at that time?

\begin{hint}
	The volume of a sphere is $V = \frac{4}{3} \pi r^3$, and its surface area is $S = 4 \pi r^2$.
\end{hint}

\begin{hint}
	Differentiating  both equations implicitly with respect to $t$, we obtain

\[
\begin{cases}
\frac{\d V}{\d t} = 4 \pi r^2 \frac{\d r}{\d t} \\
\frac{\d S}{\d t} = 8\pi r \frac{\d r}{\d t}
\end{cases}
\]
\end{hint}

\begin{hint}
	Substituing what we know about these quantities, we have

\[
\begin{cases}
3= 4 \pi (5)^2 \frac{\d r}{\d t} \\
\frac{\d S}{\d t} = 8\pi (5) \frac{\d r}{\d t}
\end{cases}
\]
\end{hint}

\begin{hint}
	The first equation allows us to see that $\frac{\d r}{\d t}  = \frac{3}{100 \pi}$, which then implies, through the second equation, that $\frac{\d S}{\d t} = 1.2$
\end{hint}

\begin{prompt}
	$\frac{\d S}{\d t} = \answer{1.2}$
\end{prompt}

\end{exercise}

\end{document}
