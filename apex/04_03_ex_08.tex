\documentclass{ximera}

\usepackage{todonotes}

\usepackage{tkz-euclide}
\usetikzlibrary{backgrounds} %% for boxes around graphs
\usetikzlibrary{shapes,positioning}  %% Clouds and stars
\usetkzobj{all}
\usepackage[makeroom]{cancel} %% for strike outs
%\usepackage{mathtools} %% for pretty underbrace % Breaks Ximera
\usepackage{multicol}


\newcommand{\RR}{\mathbb R}
\renewcommand{\d}{\,d}
\newcommand{\dd}[2][]{\frac{d #1}{d #2}}
\renewcommand{\l}{\ell}
\newcommand{\ddx}{\frac{d}{dx}}
\newcommand{\zeroOverZero}{$\boldsymbol{\tfrac{0}{0}}$}
\newcommand{\numOverZero}{$\boldsymbol{\tfrac{\#}{0}}$}
\newcommand{\dfn}{\textbf}
\newcommand{\eval}[1]{\bigg[ #1 \bigg]}
\renewcommand{\epsilon}{\varepsilon}
\renewcommand{\iff}{\Leftrightarrow}

\DeclareMathOperator{\arccot}{arccot}
\DeclareMathOperator{\arcsec}{arcsec}
\DeclareMathOperator{\arccsc}{arccsc}


\colorlet{textColor}{black} 
\colorlet{background}{white}
\colorlet{penColor}{blue!50!black} % Color of a curve in a plot
\colorlet{penColor2}{red!50!black}% Color of a curve in a plot
\colorlet{penColor3}{red!50!blue} % Color of a curve in a plot
\colorlet{penColor4}{green!50!black} % Color of a curve in a plot
\colorlet{penColor5}{orange!80!black} % Color of a curve in a plot
                                      \colorlet{fill1}{blue!50!black!20} % Color of fill in a plot
\colorlet{fill2}{blue!10} % Color of fill in a plot
\colorlet{fillp}{fill1} % Color of positive area
\colorlet{filln}{red!50!black!20} % Color of negative area
\colorlet{gridColor}{gray!50} % Color of grid in a plot

\pgfmathdeclarefunction{gauss}{2}{% gives gaussian
  \pgfmathparse{1/(#2*sqrt(2*pi))*exp(-((x-#1)^2)/(2*#2^2))}%
}



\newcommand{\fullwidth}{}
\newcommand{\normalwidth}{}



%% makes a snazzy t-chart for evaluating functions
\newenvironment{tchart}{\rowcolors{2}{}{background!90!textColor}\array}{\endarray}

%%This is to help with formatting on future title pages.
\newenvironment{sectionOutcomes}{}{} 


\author{Gregory Hartman \and Matthew Carr}
\license{Creative Commons 3.0 By-NC}
\acknowledgement{https://github.com/APEXCalculus}

\begin{document}
\begin{exercise}

\outcome{Interpret an optimization problem as the procedure used to make a system or design as effective or functional as possible.}
\outcome{Set up an optimization problem by identifying the objective function and appropriate constraints.}
\outcome{Solve optimization problems by finding the appropriate absolute extremum.}
\outcome{Solve basic word problems involving maxima or minima.}

\tag{extrema}
\tag{constrained optimization}
\tag{first derivative test}
\tag{second derivative test}

%% BADBAD
%% In general, this problem is not solvable in the full generality with which it was presented
%% Since the author's solution implies that 2 pens were built, I'm putting in an
%% assumption leading to that. 
%% Otherwise we have issues with the method Lagrange multipliers not even yielding
%% a solution with integral numbers of pens 
%% Method of Lagrange yields x=y=125, m=n=2/3 and lambda=125/3 
%% Lamba value might be wrong but the rest is right

A rancher has $1000$ feet of fencing with which to construct $2$ adjacent, equally sized, rectangular pens. What dimensions should these pens have to maximize absolutely the enclosed area, assuming the ranchers uses all $1000$ feet of fencing?
\begin{enumerate}
\item		If $x$ is the longest side of the pen, then \[x=\answer{500/3}\,ft.\]
\item		If $y$ is the shorter side of the pen, then \[y=\answer{125}\,ft.\]
\begin{hint}
If the pens are adjacent in a line, then each pen shares two sides with the ones next to it except for the pens on the end of our line.
\end{hint}
\begin{hint}
If we build pens whose width is $x$ feet and length is $y$ feet, then the first pen uses $2x+2y$ feet of fencing. Building the second pen along the edge of the first pen whose length is $x$ feet, we use a total of $(2x+2y)+(x+2y)=3x+4y$ feet of fencing which we demand satisfy $3x+4y=1000$. Given this constraint, maximize the total area enclosed, $2xy$.
\end{hint}
\begin{hint}
Solving for $y$, we have $y=\frac{1}{4}(1000-3x)$. The critical points of the area, $2x\left(\frac{1}{4}(1000-3x)\right)=-\frac{3}{2}x^2+500x$ and $\frac{d}{dx}(-\frac{3}{2}x^2+500x)=0$ when $x=\frac{500}{3}$. This is 
\end{hint}
\begin{hint}
Solving for $y$ in $(2+n)x+2ny=1000$ we find $y=\frac{1000-2x-nx}{2n}$, so the area enclosed can be written as $-\frac{1}{2}(n+2)x^2+500x$. Treating $n$ as a constant, the critical points of this function are at $\frac{500}{2+n}$ and the second derivative test indicates that it is a local maximum. We should suspect that $x=\frac{500}{2+n}$. Then the area enclosed $-\frac{1}{2}(n+2)x^2+500x$ can be written as $\frac{125000}{2+n}$. Maximizing this with respect to $n$,  
\end{hint}

\end{exercise}
\end{document}