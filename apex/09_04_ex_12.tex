\documentclass{ximera}

\usepackage{todonotes}

\usepackage{tkz-euclide}
\usetikzlibrary{backgrounds} %% for boxes around graphs
\usetikzlibrary{shapes,positioning}  %% Clouds and stars
\usetkzobj{all}
\usepackage[makeroom]{cancel} %% for strike outs
%\usepackage{mathtools} %% for pretty underbrace % Breaks Ximera
\usepackage{multicol}


\newcommand{\RR}{\mathbb R}
\renewcommand{\d}{\,d}
\newcommand{\dd}[2][]{\frac{d #1}{d #2}}
\renewcommand{\l}{\ell}
\newcommand{\ddx}{\frac{d}{dx}}
\newcommand{\zeroOverZero}{$\boldsymbol{\tfrac{0}{0}}$}
\newcommand{\numOverZero}{$\boldsymbol{\tfrac{\#}{0}}$}
\newcommand{\dfn}{\textbf}
\newcommand{\eval}[1]{\bigg[ #1 \bigg]}
\renewcommand{\epsilon}{\varepsilon}
\renewcommand{\iff}{\Leftrightarrow}

\DeclareMathOperator{\arccot}{arccot}
\DeclareMathOperator{\arcsec}{arcsec}
\DeclareMathOperator{\arccsc}{arccsc}


\colorlet{textColor}{black} 
\colorlet{background}{white}
\colorlet{penColor}{blue!50!black} % Color of a curve in a plot
\colorlet{penColor2}{red!50!black}% Color of a curve in a plot
\colorlet{penColor3}{red!50!blue} % Color of a curve in a plot
\colorlet{penColor4}{green!50!black} % Color of a curve in a plot
\colorlet{penColor5}{orange!80!black} % Color of a curve in a plot
                                      \colorlet{fill1}{blue!50!black!20} % Color of fill in a plot
\colorlet{fill2}{blue!10} % Color of fill in a plot
\colorlet{fillp}{fill1} % Color of positive area
\colorlet{filln}{red!50!black!20} % Color of negative area
\colorlet{gridColor}{gray!50} % Color of grid in a plot

\pgfmathdeclarefunction{gauss}{2}{% gives gaussian
  \pgfmathparse{1/(#2*sqrt(2*pi))*exp(-((x-#1)^2)/(2*#2^2))}%
}



\newcommand{\fullwidth}{}
\newcommand{\normalwidth}{}



%% makes a snazzy t-chart for evaluating functions
\newenvironment{tchart}{\rowcolors{2}{}{background!90!textColor}\array}{\endarray}

%%This is to help with formatting on future title pages.
\newenvironment{sectionOutcomes}{}{} 


\author{Gregory Hartman \and Matthew Carr}
\license{Creative Commons 3.0 By-NC}
\acknowledgement{https://github.com/APEXCalculus}

\begin{document}
\begin{exercise}

\outcome{Convert between polar and Cartesian coordinates.}
\outcome{Use the Cartesian to polar method to plot polar graphs.}


\tag{polar coordinates}


Given the graph of a function in rectangular (Cartesian) coordinates below, find the equation representing the graph in polar coordinates subject to the correct restriction(s) on $\theta$ and $r$.

 \begin{center}
 \begin{tikzpicture}
 \begin{axis}[ymin=-1-1/10, ymax=2+1/10,
   axis lines=center, 
   axis equal,
   xlabel=$x$, 
   ylabel=$y$,
   every axis y label/.style={at=(current axis.above origin),anchor=south}, 
   every axis x label/.style={at=(current axis.right of origin),anchor=west}, 
   grid = major]
  \addplot[penColor, samples=400, very thick, domain=-1:2] {sqrt(3)/2*x};
  \draw[densely dashed, thick] (axis cs:2,{sqrt(3)})--(axis cs:0,{sqrt(3)});
  \node [draw,fill=black,circle,inner sep=0pt,minimum size=2.5pt] at (axis cs: 2,{sqrt(3)}) {};
  \node [label=left:$\sqrt{3}$,draw,fill=black,circle,inner sep=0pt,minimum size=2.5pt] at (axis cs: 0,{sqrt(3)}) {};
\end{axis} 
\end{tikzpicture}     
\end{center} 


\[
\theta(r)=\begin{prompt}\answer{\frac{\pi}{6}}\end{prompt}
\]
subject to the restriction(s) that 
\[
\begin{prompt}\answer{-1}\end{prompt}\le r\le \begin{prompt}\answer{2}\end{prompt}
\]

      
\end{exercise}
\end{document}