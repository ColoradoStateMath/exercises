\documentclass{ximera}

\usepackage{todonotes}

\usepackage{tkz-euclide}
\usetikzlibrary{backgrounds} %% for boxes around graphs
\usetikzlibrary{shapes,positioning}  %% Clouds and stars
\usetkzobj{all}
\usepackage[makeroom]{cancel} %% for strike outs
%\usepackage{mathtools} %% for pretty underbrace % Breaks Ximera
\usepackage{multicol}


\newcommand{\RR}{\mathbb R}
\renewcommand{\d}{\,d}
\newcommand{\dd}[2][]{\frac{d #1}{d #2}}
\renewcommand{\l}{\ell}
\newcommand{\ddx}{\frac{d}{dx}}
\newcommand{\zeroOverZero}{$\boldsymbol{\tfrac{0}{0}}$}
\newcommand{\numOverZero}{$\boldsymbol{\tfrac{\#}{0}}$}
\newcommand{\dfn}{\textbf}
\newcommand{\eval}[1]{\bigg[ #1 \bigg]}
\renewcommand{\epsilon}{\varepsilon}
\renewcommand{\iff}{\Leftrightarrow}

\DeclareMathOperator{\arccot}{arccot}
\DeclareMathOperator{\arcsec}{arcsec}
\DeclareMathOperator{\arccsc}{arccsc}


\colorlet{textColor}{black} 
\colorlet{background}{white}
\colorlet{penColor}{blue!50!black} % Color of a curve in a plot
\colorlet{penColor2}{red!50!black}% Color of a curve in a plot
\colorlet{penColor3}{red!50!blue} % Color of a curve in a plot
\colorlet{penColor4}{green!50!black} % Color of a curve in a plot
\colorlet{penColor5}{orange!80!black} % Color of a curve in a plot
                                      \colorlet{fill1}{blue!50!black!20} % Color of fill in a plot
\colorlet{fill2}{blue!10} % Color of fill in a plot
\colorlet{fillp}{fill1} % Color of positive area
\colorlet{filln}{red!50!black!20} % Color of negative area
\colorlet{gridColor}{gray!50} % Color of grid in a plot

\pgfmathdeclarefunction{gauss}{2}{% gives gaussian
  \pgfmathparse{1/(#2*sqrt(2*pi))*exp(-((x-#1)^2)/(2*#2^2))}%
}



\newcommand{\fullwidth}{}
\newcommand{\normalwidth}{}



%% makes a snazzy t-chart for evaluating functions
\newenvironment{tchart}{\rowcolors{2}{}{background!90!textColor}\array}{\endarray}

%%This is to help with formatting on future title pages.
\newenvironment{sectionOutcomes}{}{} 


\author{Gregory Hartman \and Matthew Carr}
\license{Creative Commons 3.0 By-NC}
\acknowledgement{https://github.com/APEXCalculus}

\begin{document}
\begin{exercise}

\outcome{Calculate limits using the limit laws.}
\outcome{Calculate limits of the form 0/0.}
\outcome{Calculate limits using the Squeeze Theorem.}

\tag{limit} 
\tag{indeterminate form}
\tag{discontinuous}

Find 
\[
\lim_{x\to0}\left({x\sin\left({\frac{1}{x}}\right)}\right)
\begin{prompt}
= \answer{0}.
\end{prompt}
\]

\begin{hint}
Notice that $-x\le x\sin({\frac{1}{x}})\le{x}$ for all $x>0$ (emphatically, $x\ne0$) and $x\le x\sin({\frac{1}{x}})\le{-x}$ for all $x<0$ (again, emphatically, $x\ne0$). This can be restated as $-\left|{x}\right|\le x\sin({\frac{1}{x}})\le\left|{x}\right|$ for $x\ne0$. Our statement follows because $-1\le\sin({\frac{1}{x}})\le1$ for all $x\ne0$, hence, we obtained our inequality by multiplying by $x$. Apply the Squeeze Theorem to the inequality.
\end{hint}
\begin{hint}
We see that $\lim_{x\to0}\left({-x}\right)=\lim_{x\to0}(x)=0$. It follows, by the Squeeze Theorem, that $\lim_{x\to0}\left({x\sin({\frac{1}{x}}})\right)=0$.
\end{hint}
\end{exercise}
\end{document}
