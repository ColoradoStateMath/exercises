\documentclass{ximera}

\usepackage{todonotes}

\usepackage{tkz-euclide}
\usetikzlibrary{backgrounds} %% for boxes around graphs
\usetikzlibrary{shapes,positioning}  %% Clouds and stars
\usetkzobj{all}
\usepackage[makeroom]{cancel} %% for strike outs
%\usepackage{mathtools} %% for pretty underbrace % Breaks Ximera
\usepackage{multicol}


\newcommand{\RR}{\mathbb R}
\renewcommand{\d}{\,d}
\newcommand{\dd}[2][]{\frac{d #1}{d #2}}
\renewcommand{\l}{\ell}
\newcommand{\ddx}{\frac{d}{dx}}
\newcommand{\zeroOverZero}{$\boldsymbol{\tfrac{0}{0}}$}
\newcommand{\numOverZero}{$\boldsymbol{\tfrac{\#}{0}}$}
\newcommand{\dfn}{\textbf}
\newcommand{\eval}[1]{\bigg[ #1 \bigg]}
\renewcommand{\epsilon}{\varepsilon}
\renewcommand{\iff}{\Leftrightarrow}

\DeclareMathOperator{\arccot}{arccot}
\DeclareMathOperator{\arcsec}{arcsec}
\DeclareMathOperator{\arccsc}{arccsc}


\colorlet{textColor}{black} 
\colorlet{background}{white}
\colorlet{penColor}{blue!50!black} % Color of a curve in a plot
\colorlet{penColor2}{red!50!black}% Color of a curve in a plot
\colorlet{penColor3}{red!50!blue} % Color of a curve in a plot
\colorlet{penColor4}{green!50!black} % Color of a curve in a plot
\colorlet{penColor5}{orange!80!black} % Color of a curve in a plot
                                      \colorlet{fill1}{blue!50!black!20} % Color of fill in a plot
\colorlet{fill2}{blue!10} % Color of fill in a plot
\colorlet{fillp}{fill1} % Color of positive area
\colorlet{filln}{red!50!black!20} % Color of negative area
\colorlet{gridColor}{gray!50} % Color of grid in a plot

\pgfmathdeclarefunction{gauss}{2}{% gives gaussian
  \pgfmathparse{1/(#2*sqrt(2*pi))*exp(-((x-#1)^2)/(2*#2^2))}%
}



\newcommand{\fullwidth}{}
\newcommand{\normalwidth}{}



%% makes a snazzy t-chart for evaluating functions
\newenvironment{tchart}{\rowcolors{2}{}{background!90!textColor}\array}{\endarray}

%%This is to help with formatting on future title pages.
\newenvironment{sectionOutcomes}{}{} 


\author{Gregory Hartman \and Matthew Carr}
\license{Creative Commons 3.0 By-NC}
\acknowledgement{https://github.com/APEXCalculus}

\begin{document}
\begin{exercise}

\outcome{Interpret an optimization problem as the procedure used to make a system or design as effective or functional as possible.}
\outcome{Set up an optimization problem by identifying the objective function and appropriate constraints.}
\outcome{Solve optimization problems by finding the appropriate absolute extremum.}
\outcome{Solve basic word problems involving maxima or minima.}

\tag{extrema}
\tag{constrained optimization}
\tag{first derivative test}
\tag{second derivative test}

Find the absolute maximum of a sum of two numbers, each of which is in $[0,300]$ whose product is $500$.\[\answer{\frac{905}{3}}\]

\begin{hint}
If the two numbers are $x$ and $y$, then $xy=500$, so $y=\frac{500}{x}$, so we are left with the job of finding the absolute maximum of $x+\frac{500}{x}$ subject to the constraint that $x$ lie in $[0,300]$ (since $\frac{500}{x}\le300$ for all $x$ in $[0,300]$, we needn't worry that $y$ lies outside of $[0,300]$).
\end{hint}
\begin{hint}
The only critical point of $x+\frac{500}{x}$ that lies in $[0,300]$ is a minimum. Since $[0,300]$ is a closed interval (i.e., it contains its endpoints), what can you say about where the maximum of $x+\frac{500}{x}$ must occur?
\end{hint}
\begin{hint}
Recall that extrema can occur at critical points or at the endpoints of a closed interval, such as $[0,300]$. 
\end{hint}
\begin{hint}
If $xy=500$, then $x$ cannot be $0$, so check the other endpoint.
\end{hint}
\begin{hint}
If $x=300$, then $300+\frac{500}{300}=\frac{905}{3}$ must be the maximum. Indeed, it is the absolute maximum, for if $x<300$, then clearly $x+\frac{500}{x}<\frac{905}{3}$.
\end{hint}
\end{exercise}
\end{document}
