\documentclass{ximera}

\usepackage{todonotes}

\usepackage{tkz-euclide}
\usetikzlibrary{backgrounds} %% for boxes around graphs
\usetikzlibrary{shapes,positioning}  %% Clouds and stars
\usetkzobj{all}
\usepackage[makeroom]{cancel} %% for strike outs
%\usepackage{mathtools} %% for pretty underbrace % Breaks Ximera
\usepackage{multicol}


\newcommand{\RR}{\mathbb R}
\renewcommand{\d}{\,d}
\newcommand{\dd}[2][]{\frac{d #1}{d #2}}
\renewcommand{\l}{\ell}
\newcommand{\ddx}{\frac{d}{dx}}
\newcommand{\zeroOverZero}{$\boldsymbol{\tfrac{0}{0}}$}
\newcommand{\numOverZero}{$\boldsymbol{\tfrac{\#}{0}}$}
\newcommand{\dfn}{\textbf}
\newcommand{\eval}[1]{\bigg[ #1 \bigg]}
\renewcommand{\epsilon}{\varepsilon}
\renewcommand{\iff}{\Leftrightarrow}

\DeclareMathOperator{\arccot}{arccot}
\DeclareMathOperator{\arcsec}{arcsec}
\DeclareMathOperator{\arccsc}{arccsc}


\colorlet{textColor}{black} 
\colorlet{background}{white}
\colorlet{penColor}{blue!50!black} % Color of a curve in a plot
\colorlet{penColor2}{red!50!black}% Color of a curve in a plot
\colorlet{penColor3}{red!50!blue} % Color of a curve in a plot
\colorlet{penColor4}{green!50!black} % Color of a curve in a plot
\colorlet{penColor5}{orange!80!black} % Color of a curve in a plot
                                      \colorlet{fill1}{blue!50!black!20} % Color of fill in a plot
\colorlet{fill2}{blue!10} % Color of fill in a plot
\colorlet{fillp}{fill1} % Color of positive area
\colorlet{filln}{red!50!black!20} % Color of negative area
\colorlet{gridColor}{gray!50} % Color of grid in a plot

\pgfmathdeclarefunction{gauss}{2}{% gives gaussian
  \pgfmathparse{1/(#2*sqrt(2*pi))*exp(-((x-#1)^2)/(2*#2^2))}%
}



\newcommand{\fullwidth}{}
\newcommand{\normalwidth}{}



%% makes a snazzy t-chart for evaluating functions
\newenvironment{tchart}{\rowcolors{2}{}{background!90!textColor}\array}{\endarray}

%%This is to help with formatting on future title pages.
\newenvironment{sectionOutcomes}{}{} 


\author{Gregory Hartman \and Matthew Carr}
\license{Creative Commons 3.0 By-NC}
\acknowledgement{https://github.com/APEXCalculus}

\begin{document}
\begin{exercise}

\outcome{Solve basic related rates word problems.}
\outcome{Identify word problems as related rates problems.}
\outcome{Understand the process of solving related rates problems.}

\tag{related rates}


An inverted cylindrical cone, $20$ft deep and $10$ft across at the top
is being filled with water at a rate of $10$ft$^3$/min. (Recall: the
volume of such a cone $V$ is given by $V=\frac{1}{3}\pi r^2 h$ where
$h$ is the height of the cone and $r$ is the radius of its base.)

What is the magnitude of the rate, $\tau$, that the water is rising at
in the tank when the depth of the water is:
\begin{enumerate}
\item $1$ft? \begin{prompt}Write DNE if the rate is undefined. \[\tau=\answer{\frac{160}{\pi}}\,ft/min\]\end{prompt}
\item $10$ft? \begin{prompt}Write DNE if the rate is undefined. \[\tau=\answer{\frac{8}{5\pi}}\,ft/min\]\end{prompt}
\item $10$ft? \begin{prompt}Write DNE if the rate is undefined. \[\tau=\answer{\frac{160}{361\pi}}\,ft/min\]\end{prompt}
\end{enumerate}
\begin{prompt}
At what time, $T$, will the tank be filled if it starts empty?
\[
T=\answer{\frac{500\pi}{30}}\,min
\]
\end{prompt}
\end{exercise}
\end{document}
