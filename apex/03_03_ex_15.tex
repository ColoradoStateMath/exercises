\documentclass{ximera}

\usepackage{todonotes}

\usepackage{tkz-euclide}
\usetikzlibrary{backgrounds} %% for boxes around graphs
\usetikzlibrary{shapes,positioning}  %% Clouds and stars
\usetkzobj{all}
\usepackage[makeroom]{cancel} %% for strike outs
%\usepackage{mathtools} %% for pretty underbrace % Breaks Ximera
\usepackage{multicol}


\newcommand{\RR}{\mathbb R}
\renewcommand{\d}{\,d}
\newcommand{\dd}[2][]{\frac{d #1}{d #2}}
\renewcommand{\l}{\ell}
\newcommand{\ddx}{\frac{d}{dx}}
\newcommand{\zeroOverZero}{$\boldsymbol{\tfrac{0}{0}}$}
\newcommand{\numOverZero}{$\boldsymbol{\tfrac{\#}{0}}$}
\newcommand{\dfn}{\textbf}
\newcommand{\eval}[1]{\bigg[ #1 \bigg]}
\renewcommand{\epsilon}{\varepsilon}
\renewcommand{\iff}{\Leftrightarrow}

\DeclareMathOperator{\arccot}{arccot}
\DeclareMathOperator{\arcsec}{arcsec}
\DeclareMathOperator{\arccsc}{arccsc}


\colorlet{textColor}{black} 
\colorlet{background}{white}
\colorlet{penColor}{blue!50!black} % Color of a curve in a plot
\colorlet{penColor2}{red!50!black}% Color of a curve in a plot
\colorlet{penColor3}{red!50!blue} % Color of a curve in a plot
\colorlet{penColor4}{green!50!black} % Color of a curve in a plot
\colorlet{penColor5}{orange!80!black} % Color of a curve in a plot
                                      \colorlet{fill1}{blue!50!black!20} % Color of fill in a plot
\colorlet{fill2}{blue!10} % Color of fill in a plot
\colorlet{fillp}{fill1} % Color of positive area
\colorlet{filln}{red!50!black!20} % Color of negative area
\colorlet{gridColor}{gray!50} % Color of grid in a plot

\pgfmathdeclarefunction{gauss}{2}{% gives gaussian
  \pgfmathparse{1/(#2*sqrt(2*pi))*exp(-((x-#1)^2)/(2*#2^2))}%
}



\newcommand{\fullwidth}{}
\newcommand{\normalwidth}{}



%% makes a snazzy t-chart for evaluating functions
\newenvironment{tchart}{\rowcolors{2}{}{background!90!textColor}\array}{\endarray}

%%This is to help with formatting on future title pages.
\newenvironment{sectionOutcomes}{}{} 


\author{Gregory Hartman \and Matthew Carr}
\license{Creative Commons 3.0 By-NC}
\acknowledgement{https://github.com/APEXCalculus}

\begin{document}
\begin{exercise}

\outcome{Use the first derivative to determine whether a function is increasing or decreasing.}
\outcome{Understand what information the derivative gives concerning when a function is increasing or decreasing.}
\outcome{Find domain and range.}
\outcome{Find critical points.}
\outcome{Find all local maximums and minimums using the 1st and 2nd derivative tests.}
\outcome{Classify critical points.}

\tag{increasing}
\tag{decreasing}
\tag{derivative test}

Let $f(x)=x^3+3x^2+3$. 
\begin{enumerate}
\item		The domain of $f$ is $\answer{(-\infty,\infty)}$.
\item		$f$ has a critical point(s) at $x = \answer{\{-2,0\}}$, write DNE if no such point(s) exist.
\item		$f$ is decreasing on $\answer{(-2,0)}$, write DNE if no such interval(s) exist.
\item		$f$ is increasing on $\answer{(-\infty,-2)\cup(0,\infty)}$, write DNE if no such interval(s) exist.
\item		At $x=-2$, $f$ has
\begin{multipleChoice}
\choice{A local minimum}
\choice[correct]{A local maximum}
\choice{Both a local maximum and a local minimum}
\choice{Neither a local maximum nor a local minimum}
\end{multipleChoice}

\item		At $x=0$, $f$ has
\begin{multipleChoice}
\choice[correct]{A local minimum}
\choice{A local maximum}
\choice{Both a local maximum and a local minimum}
\choice{Neither a local maximum nor a local minimum}
\end{multipleChoice}

\end{enumerate}

\end{exercise}
\end{document}