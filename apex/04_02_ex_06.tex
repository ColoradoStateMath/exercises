\documentclass{ximera}

\usepackage{todonotes}

\usepackage{tkz-euclide}
\usetikzlibrary{backgrounds} %% for boxes around graphs
\usetikzlibrary{shapes,positioning}  %% Clouds and stars
\usetkzobj{all}
\usepackage[makeroom]{cancel} %% for strike outs
%\usepackage{mathtools} %% for pretty underbrace % Breaks Ximera
\usepackage{multicol}


\newcommand{\RR}{\mathbb R}
\renewcommand{\d}{\,d}
\newcommand{\dd}[2][]{\frac{d #1}{d #2}}
\renewcommand{\l}{\ell}
\newcommand{\ddx}{\frac{d}{dx}}
\newcommand{\zeroOverZero}{$\boldsymbol{\tfrac{0}{0}}$}
\newcommand{\numOverZero}{$\boldsymbol{\tfrac{\#}{0}}$}
\newcommand{\dfn}{\textbf}
\newcommand{\eval}[1]{\bigg[ #1 \bigg]}
\renewcommand{\epsilon}{\varepsilon}
\renewcommand{\iff}{\Leftrightarrow}

\DeclareMathOperator{\arccot}{arccot}
\DeclareMathOperator{\arcsec}{arcsec}
\DeclareMathOperator{\arccsc}{arccsc}


\colorlet{textColor}{black} 
\colorlet{background}{white}
\colorlet{penColor}{blue!50!black} % Color of a curve in a plot
\colorlet{penColor2}{red!50!black}% Color of a curve in a plot
\colorlet{penColor3}{red!50!blue} % Color of a curve in a plot
\colorlet{penColor4}{green!50!black} % Color of a curve in a plot
\colorlet{penColor5}{orange!80!black} % Color of a curve in a plot
                                      \colorlet{fill1}{blue!50!black!20} % Color of fill in a plot
\colorlet{fill2}{blue!10} % Color of fill in a plot
\colorlet{fillp}{fill1} % Color of positive area
\colorlet{filln}{red!50!black!20} % Color of negative area
\colorlet{gridColor}{gray!50} % Color of grid in a plot

\pgfmathdeclarefunction{gauss}{2}{% gives gaussian
  \pgfmathparse{1/(#2*sqrt(2*pi))*exp(-((x-#1)^2)/(2*#2^2))}%
}



\newcommand{\fullwidth}{}
\newcommand{\normalwidth}{}



%% makes a snazzy t-chart for evaluating functions
\newenvironment{tchart}{\rowcolors{2}{}{background!90!textColor}\array}{\endarray}

%%This is to help with formatting on future title pages.
\newenvironment{sectionOutcomes}{}{} 


\author{Gregory Hartman \and Matthew Carr}
\license{Creative Commons 3.0 By-NC}
\acknowledgement{https://github.com/APEXCalculus}

\begin{document}
\begin{exercise}

\outcome{Solve basic related rates word problems.}
\outcome{Identify word problems as related rates problems.}
\outcome{Understand the process of solving related rates problems.}

\tag{related rates}

Radar guns measure the rate of distance change between the gun and the object it is measuring. For instance, a reading of ``$55$mph'' means the object is moving away from the gun at a rate of $55$ miles per hour, whereas a measurement of ``$-25$mph'' would mean that the object is approaching the gun at a rate of $25$ miles per hour.

If the radar gun is moving (say, attached to a police car) then radar readouts are only immediately understandable if the gun and the object it is measuring are moving along the same line. For instance, if a police officer is traveling $60$mph and measures the speed of a car ahead of him and gets a readout of $15$mph, he knows that the car ahead of him is moving away at a rate of $15$ miles an hour (i.e., the car's speed relative to him), meaning the car is traveling $75$mph.


Now, suppose an officer is driving due north and sees a car moving due west towards the intersection of their two roads.

What speed, $s$, is the other car traveling at in the following situations:
\begin{enumerate}
\item		The officer is traveling due north at $50$mph and is $1/2$ mile from the intersection while the other car is $1$ mile from the intersection, traveling west, and the officer's radar reading is $-80$mph? \[s = \answer{40\sqrt{5}-25}\]
\item		The officer is traveling due north at $50$mph and is $1$ mile from the intersection while the other car is $1/2$ mile from the intersection, traveling west, and the officer's radar reading is $-80$mph? \[s = \answer{80\sqrt{5}-100}\]
\end{enumerate}

\begin{hint}
Draw a picture and remember that \textbf{speed} is the absolute value of \textbf{velocity}. Recall that the radar gun measures the rate of change in the distance between the two cars. If $r(t)$ measures the distance between them as a function of time, then $r(t)$ is the hypotenuse of the right triangle formed between $x(t)$ (the function of time measuring the distance of the other car from the intersection of the roads) and $y(t)$ (the function of time measuring the distance of the police car from the intersection of the roads). 
\end{hint}
\begin{hint}
The relative speed between the vehicles is given by $\frac{dr}{dt}$. If $r=\sqrt{x(t)^2+y(t)^2}$ where $x(t)$ is the position of the other car as a function of time and $y(t)$ is the position of the police car as a function of time, what is $\frac{dr}{dt}$? What are $x(t)$ and $x'(t)$? What are $y(t)$ and $y'(t)$? How does this help? 
\end{hint}
\begin{hint}
By the chain rule, $\frac{dr}{dt}=\frac{x(t)\frac{dx}{dt}+y(t)\frac{dy}{dt}}{\sqrt{x(t)^2+y(t)^2}}$. What are $x(t)$,$x'(t)$, $y(t)$ and $y'(t)$ as functions of time?
\end{hint}
\begin{hint}
In the first scenario, we see that $x(t)=-st+1$, $\frac{dx}{dt}=-s$, $y(t)=50t-\frac{1}{2}$ and $\frac{dy}{dt}=50$ where $t$, for convenience, is time measured in hours. In the second scenario, we see that $x(t)=-st+\frac{1}{2}$, $\frac{dx}{dt}=-s$, $y(t)=50t-1$ and $\frac{dy}{dt}=50$ where $t$, for convenience, is measured in hours.
\end{hint}
\begin{hint}
In the first scenario, at $t=0$, when the radar makes its measurement, $\frac{dr}{dt}=\frac{2(-25-s)}{\sqrt{5}}$ which we know is equal to $-80$, hence, $s=40\sqrt{5}-25$.

In the second scenario, at $t=0$, when the radar makes its measurement, $\frac{dr}{dt}=\frac{-100-s}{\sqrt{5}}$ which we know is equal to $-80$, hence, $s=80\sqrt{5}-100$.
\end{hint}
\end{exercise}
\end{document}