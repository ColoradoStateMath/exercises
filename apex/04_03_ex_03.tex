\documentclass{ximera}

\usepackage{todonotes}

\usepackage{tkz-euclide}
\usetikzlibrary{backgrounds} %% for boxes around graphs
\usetikzlibrary{shapes,positioning}  %% Clouds and stars
\usetkzobj{all}
\usepackage[makeroom]{cancel} %% for strike outs
%\usepackage{mathtools} %% for pretty underbrace % Breaks Ximera
\usepackage{multicol}


\newcommand{\RR}{\mathbb R}
\renewcommand{\d}{\,d}
\newcommand{\dd}[2][]{\frac{d #1}{d #2}}
\renewcommand{\l}{\ell}
\newcommand{\ddx}{\frac{d}{dx}}
\newcommand{\zeroOverZero}{$\boldsymbol{\tfrac{0}{0}}$}
\newcommand{\numOverZero}{$\boldsymbol{\tfrac{\#}{0}}$}
\newcommand{\dfn}{\textbf}
\newcommand{\eval}[1]{\bigg[ #1 \bigg]}
\renewcommand{\epsilon}{\varepsilon}
\renewcommand{\iff}{\Leftrightarrow}

\DeclareMathOperator{\arccot}{arccot}
\DeclareMathOperator{\arcsec}{arcsec}
\DeclareMathOperator{\arccsc}{arccsc}


\colorlet{textColor}{black} 
\colorlet{background}{white}
\colorlet{penColor}{blue!50!black} % Color of a curve in a plot
\colorlet{penColor2}{red!50!black}% Color of a curve in a plot
\colorlet{penColor3}{red!50!blue} % Color of a curve in a plot
\colorlet{penColor4}{green!50!black} % Color of a curve in a plot
\colorlet{penColor5}{orange!80!black} % Color of a curve in a plot
                                      \colorlet{fill1}{blue!50!black!20} % Color of fill in a plot
\colorlet{fill2}{blue!10} % Color of fill in a plot
\colorlet{fillp}{fill1} % Color of positive area
\colorlet{filln}{red!50!black!20} % Color of negative area
\colorlet{gridColor}{gray!50} % Color of grid in a plot

\pgfmathdeclarefunction{gauss}{2}{% gives gaussian
  \pgfmathparse{1/(#2*sqrt(2*pi))*exp(-((x-#1)^2)/(2*#2^2))}%
}



\newcommand{\fullwidth}{}
\newcommand{\normalwidth}{}



%% makes a snazzy t-chart for evaluating functions
\newenvironment{tchart}{\rowcolors{2}{}{background!90!textColor}\array}{\endarray}

%%This is to help with formatting on future title pages.
\newenvironment{sectionOutcomes}{}{} 


\author{Gregory Hartman \and Matthew Carr}
\license{Creative Commons 3.0 By-NC}
\acknowledgement{https://github.com/APEXCalculus}

\begin{document}
\begin{exercise}

\outcome{Interpret an optimization problem as the procedure used to make a system or design as effective or functional as possible.}
\outcome{Set up an optimization problem by identifying the objective function and appropriate constraints.}
\outcome{Solve optimization problems by finding the appropriate absolute extremum.}
\outcome{Solve basic word problems involving maxima or minima.}

\tag{extrema}
\tag{constrained optimization}
\tag{first derivative test}
\tag{second derivative test}

Find the absolute maximum product of two numbers (not necessarily integers) whose sum is $100$. Write DNE is no absolute maximum exists. 

\begin{hint}
We want to find the absolute maximum the product of two numbers (call them $x$ and $y$). Thus, we want to maximize $xy$ subject to the constraint that $x+y=100$.
\end{hint}
\begin{hint}
Solving for $y$, we see that $y=100-x$. Then $xy=-x^2+100x$. We must find the absolute maximum of this product. 
\end{hint}
\begin{hint}
If there is an absolute maximum, then, since $-x^2+100x$ is defined everywhere, the absolute maximum must occur where there is a local maximum. Hence, we find the critical points of $-x^2+100x$: $\frac{d}{dx}(-x^2+100x)=0$ and $\frac{d}{dx}(-x^2+100x)=100-2x$, hence, solving for $x$ in $100-2x=0$, we have $x=50$. So $x=50$ and $y=50$.

But is this a maximum or a minimum? By the second derivative test, $\frac{d^2}{dx^2}(-x^2+100x)=-2$ and $-2<0$, so this is a maximum. We see by inspection that it is an absolute maximum because the function does not attain a larger value.

Thus, $xy$ is maximized with $xy=2500$. 
\end{hint}
\begin{prompt}
\[
\answer{2500}
\]
\end{prompt}


\end{exercise}
\end{document}
