\documentclass{ximera}

\usepackage{todonotes}

\usepackage{tkz-euclide}
\usetikzlibrary{backgrounds} %% for boxes around graphs
\usetikzlibrary{shapes,positioning}  %% Clouds and stars
\usetkzobj{all}
\usepackage[makeroom]{cancel} %% for strike outs
%\usepackage{mathtools} %% for pretty underbrace % Breaks Ximera
\usepackage{multicol}


\newcommand{\RR}{\mathbb R}
\renewcommand{\d}{\,d}
\newcommand{\dd}[2][]{\frac{d #1}{d #2}}
\renewcommand{\l}{\ell}
\newcommand{\ddx}{\frac{d}{dx}}
\newcommand{\zeroOverZero}{$\boldsymbol{\tfrac{0}{0}}$}
\newcommand{\numOverZero}{$\boldsymbol{\tfrac{\#}{0}}$}
\newcommand{\dfn}{\textbf}
\newcommand{\eval}[1]{\bigg[ #1 \bigg]}
\renewcommand{\epsilon}{\varepsilon}
\renewcommand{\iff}{\Leftrightarrow}

\DeclareMathOperator{\arccot}{arccot}
\DeclareMathOperator{\arcsec}{arcsec}
\DeclareMathOperator{\arccsc}{arccsc}


\colorlet{textColor}{black} 
\colorlet{background}{white}
\colorlet{penColor}{blue!50!black} % Color of a curve in a plot
\colorlet{penColor2}{red!50!black}% Color of a curve in a plot
\colorlet{penColor3}{red!50!blue} % Color of a curve in a plot
\colorlet{penColor4}{green!50!black} % Color of a curve in a plot
\colorlet{penColor5}{orange!80!black} % Color of a curve in a plot
                                      \colorlet{fill1}{blue!50!black!20} % Color of fill in a plot
\colorlet{fill2}{blue!10} % Color of fill in a plot
\colorlet{fillp}{fill1} % Color of positive area
\colorlet{filln}{red!50!black!20} % Color of negative area
\colorlet{gridColor}{gray!50} % Color of grid in a plot

\pgfmathdeclarefunction{gauss}{2}{% gives gaussian
  \pgfmathparse{1/(#2*sqrt(2*pi))*exp(-((x-#1)^2)/(2*#2^2))}%
}



\newcommand{\fullwidth}{}
\newcommand{\normalwidth}{}



%% makes a snazzy t-chart for evaluating functions
\newenvironment{tchart}{\rowcolors{2}{}{background!90!textColor}\array}{\endarray}

%%This is to help with formatting on future title pages.
\newenvironment{sectionOutcomes}{}{} 


\author{Gregory Hartman \and Matthew Carr}
\license{Creative Commons 3.0 By-NC}
\acknowledgement{https://github.com/APEXCalculus}

\begin{document}
\begin{exercise}

\outcome{Estimate limits using nearby values.}
\outcome{Estimating limits numerically and possible errors to this method.}

\tag{limit}
\tag{derivative}

% The r@{.}l aligns everything at a common decimal point by constructing two columns and 
% aligning them together at that decimal point to create the effect that only a single column is
% there. It's a quick fix to a very minor blemish in the source.

% We need the multicolumn{2}{c} to align the 'h' over the center of the two columns we have     
% made, since we cannot just change the number of columns midway through our construction. 
% By indicating that there are 2 columns we are merging, TeX understands our construction. 

% Still need a vertical spacing fix to make it so that the hline doesn't collide with the frac.          
% Fin ch�han dal vino calda la testa, una gran festa fa� preparar! It must also satisfy that the h is 
% centered with the frac, rather than hovering unnaturally far above the hline

Let $f(x) = -7x+2$ and $a=3$. Observe the table of values for $\frac{f({a+h})-f({a})}{h}$:
\begin{center}
 \begin{tabular}{r@{.}lc}
  \multicolumn{2}{c}{$h$} & $\frac{f(a+h)-f(a)}{h}$\\ \hline 
  $-0$ & $1$ & $-7$ \\
  $-0$ & $01$ & $-7$ \\
  $0$ & $01$ & $-7$ \\
  $0$ & $1$ & $-7$
 \end{tabular}
\end{center}
Estimate the limit as $h\to 0$. \begin{prompt}
  $\answer{-7}$\end{prompt}

\end{exercise}
\end{document}
