\documentclass{ximera}

\usepackage{todonotes}

\usepackage{tkz-euclide}
\usetikzlibrary{backgrounds} %% for boxes around graphs
\usetikzlibrary{shapes,positioning}  %% Clouds and stars
\usetkzobj{all}
\usepackage[makeroom]{cancel} %% for strike outs
%\usepackage{mathtools} %% for pretty underbrace % Breaks Ximera
\usepackage{multicol}


\newcommand{\RR}{\mathbb R}
\renewcommand{\d}{\,d}
\newcommand{\dd}[2][]{\frac{d #1}{d #2}}
\renewcommand{\l}{\ell}
\newcommand{\ddx}{\frac{d}{dx}}
\newcommand{\zeroOverZero}{$\boldsymbol{\tfrac{0}{0}}$}
\newcommand{\numOverZero}{$\boldsymbol{\tfrac{\#}{0}}$}
\newcommand{\dfn}{\textbf}
\newcommand{\eval}[1]{\bigg[ #1 \bigg]}
\renewcommand{\epsilon}{\varepsilon}
\renewcommand{\iff}{\Leftrightarrow}

\DeclareMathOperator{\arccot}{arccot}
\DeclareMathOperator{\arcsec}{arcsec}
\DeclareMathOperator{\arccsc}{arccsc}


\colorlet{textColor}{black} 
\colorlet{background}{white}
\colorlet{penColor}{blue!50!black} % Color of a curve in a plot
\colorlet{penColor2}{red!50!black}% Color of a curve in a plot
\colorlet{penColor3}{red!50!blue} % Color of a curve in a plot
\colorlet{penColor4}{green!50!black} % Color of a curve in a plot
\colorlet{penColor5}{orange!80!black} % Color of a curve in a plot
                                      \colorlet{fill1}{blue!50!black!20} % Color of fill in a plot
\colorlet{fill2}{blue!10} % Color of fill in a plot
\colorlet{fillp}{fill1} % Color of positive area
\colorlet{filln}{red!50!black!20} % Color of negative area
\colorlet{gridColor}{gray!50} % Color of grid in a plot

\pgfmathdeclarefunction{gauss}{2}{% gives gaussian
  \pgfmathparse{1/(#2*sqrt(2*pi))*exp(-((x-#1)^2)/(2*#2^2))}%
}



\newcommand{\fullwidth}{}
\newcommand{\normalwidth}{}



%% makes a snazzy t-chart for evaluating functions
\newenvironment{tchart}{\rowcolors{2}{}{background!90!textColor}\array}{\endarray}

%%This is to help with formatting on future title pages.
\newenvironment{sectionOutcomes}{}{} 


\author{Gregory Hartman \and Matthew Carr}
\license{Creative Commons 3.0 By-NC}
\acknowledgement{https://github.com/APEXCalculus}

\begin{document}
\begin{exercise}

\outcome{Interpret an optimization problem as the procedure used to make a system or design as effective or functional as possible.}
\outcome{Set up an optimization problem by identifying the objective function and appropriate constraints.}
\outcome{Solve optimization problems by finding the appropriate absolute extremum.}
\outcome{Solve basic word problems involving maxima or minima.}

\tag{extrema}
\tag{constrained optimization}
\tag{first derivative test}
\tag{second derivative test}

A standard soda can is roughly cylindrical and holds $355$cm$^3$ of liquid. What dimensions should the cylinder be to minimize absolutely the material needed to produce the can? (Recall that the surface area of a cylinder with its caps included is given by $S=2\pi r h+2\pi r^2$ where $h$ is its height and $r$ is the radius of its base, and its volume is given by $V=\pi r^2 h$).
\begin{enumerate}
\item		The radius should be\[r=\answer{\left(\frac{355}{2\pi}\right)^{1/3}}\,cm.\]
\item		The height should be \[h=\answer{2^{1/3}\left(\frac{710}{\pi}\right)^{1/3}}\,cm.\]
\end{enumerate}
\end{exercise}
\end{document}