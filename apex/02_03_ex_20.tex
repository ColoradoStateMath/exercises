\documentclass{ximera}

\usepackage{todonotes}

\usepackage{tkz-euclide}
\usetikzlibrary{backgrounds} %% for boxes around graphs
\usetikzlibrary{shapes,positioning}  %% Clouds and stars
\usetkzobj{all}
\usepackage[makeroom]{cancel} %% for strike outs
%\usepackage{mathtools} %% for pretty underbrace % Breaks Ximera
\usepackage{multicol}


\newcommand{\RR}{\mathbb R}
\renewcommand{\d}{\,d}
\newcommand{\dd}[2][]{\frac{d #1}{d #2}}
\renewcommand{\l}{\ell}
\newcommand{\ddx}{\frac{d}{dx}}
\newcommand{\zeroOverZero}{$\boldsymbol{\tfrac{0}{0}}$}
\newcommand{\numOverZero}{$\boldsymbol{\tfrac{\#}{0}}$}
\newcommand{\dfn}{\textbf}
\newcommand{\eval}[1]{\bigg[ #1 \bigg]}
\renewcommand{\epsilon}{\varepsilon}
\renewcommand{\iff}{\Leftrightarrow}

\DeclareMathOperator{\arccot}{arccot}
\DeclareMathOperator{\arcsec}{arcsec}
\DeclareMathOperator{\arccsc}{arccsc}


\colorlet{textColor}{black} 
\colorlet{background}{white}
\colorlet{penColor}{blue!50!black} % Color of a curve in a plot
\colorlet{penColor2}{red!50!black}% Color of a curve in a plot
\colorlet{penColor3}{red!50!blue} % Color of a curve in a plot
\colorlet{penColor4}{green!50!black} % Color of a curve in a plot
\colorlet{penColor5}{orange!80!black} % Color of a curve in a plot
                                      \colorlet{fill1}{blue!50!black!20} % Color of fill in a plot
\colorlet{fill2}{blue!10} % Color of fill in a plot
\colorlet{fillp}{fill1} % Color of positive area
\colorlet{filln}{red!50!black!20} % Color of negative area
\colorlet{gridColor}{gray!50} % Color of grid in a plot

\pgfmathdeclarefunction{gauss}{2}{% gives gaussian
  \pgfmathparse{1/(#2*sqrt(2*pi))*exp(-((x-#1)^2)/(2*#2^2))}%
}



\newcommand{\fullwidth}{}
\newcommand{\normalwidth}{}



%% makes a snazzy t-chart for evaluating functions
\newenvironment{tchart}{\rowcolors{2}{}{background!90!textColor}\array}{\endarray}

%%This is to help with formatting on future title pages.
\newenvironment{sectionOutcomes}{}{} 


\author{Gregory Hartman \and Matthew Carr}
\license{Creative Commons 3.0 By-NC}
\acknowledgement{https://github.com/APEXCalculus}

\begin{document}
\begin{exercise}

\tag{derivative}

\outcome{Know and use the properties of exponential and logarithmic functions.}
\outcome{State the derivative of the natural log function.}

%% BADBAD 
%% I feel it is in bad taste to do this problem without some sort of hint because,
%% they don't know the chain rule here. So, using the properties of logarithms, 
%% one obtains ln(5)+ln(x^2) and, assuming they don't know the chain rule,
%% they would "obtain" ln(5)+2*ln(x) but ln(x^2) has domain R\{0} while 2*ln(x) has domain R^{+} 
%% so it would be misleading to indicate that this problem is tractable without
%% the observation that ln(5x^2) can be written in an equivalent form as 
%% a function g(x), where g(x)=ln(5)+2ln(x) when x>0 and g(x)=ln(5)+2ln(-x) when x<0





Let $f(x)=\ln(5x^2)$ and define 
\[
g(x)=\begin{cases}
2\ln(\sqrt{5}x) & x>0\\
2\ln(-\sqrt{5}x) & x<0.
\end{cases}
\]
Then $g(x)=f(x)$ for every $x\ne0$. (Can you see why? Apply some properties of logarithms.)

Find the derivative of $f(x)=\ln(5x^2)$ with respect to $x$. 
\[
f'(x)
\begin{prompt} 
= \answer{\frac{2}{x}}
\end{prompt}
\]
\begin{hint}
The properties of the logarithm tell us that, if $x>0$, then $g(x)=2\ln(\sqrt{5}x)=\ln(\left(\sqrt{5}x\right)^2)=\ln(5x^2)$. If $x<0$, then, since $-x>0$, $g(x)=2\ln(-\sqrt{5}x)=\ln(\left(-\sqrt{5}x\right)^2)=\ln(5x^2)$.  
\end{hint}
\begin{hint}
Using $g(x)$, if $x>0$, then we see that $2\ln(\sqrt{5}x)=2\ln(\sqrt{5})+2\ln(x))$. The derivative for all $x>0$ of $g(x)$ is $\frac{d}{dx}\left(2\ln(\sqrt{5})\right)+\frac{d}{dx}\left(2\ln(x)\right)=\frac{2}{x}$ recalling that $\frac{d}{dx}\left(\ln(x)\right)=\frac{1}{x}$ when $x>0$. What can you say when $x<0$?
\end{hint}
\begin{hint}
Similarly, when $x<0$, $-x>0$, so $g(x)=2\ln(-\sqrt{5}x)=\ln(\left(-\sqrt{5}x\right)^2)=\ln(5x^2)$. The derivative for all $x<0$ of $g(x)$ is thus $\frac{2}{x}$. So $f'(x)=\frac{2}{x}$.
\end{hint}
\end{exercise}
\end{document}