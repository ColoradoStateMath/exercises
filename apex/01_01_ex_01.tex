\documentclass{ximera}

\usepackage{todonotes}

\usepackage{tkz-euclide}
\usetikzlibrary{backgrounds} %% for boxes around graphs
\usetikzlibrary{shapes,positioning}  %% Clouds and stars
\usetkzobj{all}
\usepackage[makeroom]{cancel} %% for strike outs
%\usepackage{mathtools} %% for pretty underbrace % Breaks Ximera
\usepackage{multicol}


\newcommand{\RR}{\mathbb R}
\renewcommand{\d}{\,d}
\newcommand{\dd}[2][]{\frac{d #1}{d #2}}
\renewcommand{\l}{\ell}
\newcommand{\ddx}{\frac{d}{dx}}
\newcommand{\zeroOverZero}{$\boldsymbol{\tfrac{0}{0}}$}
\newcommand{\numOverZero}{$\boldsymbol{\tfrac{\#}{0}}$}
\newcommand{\dfn}{\textbf}
\newcommand{\eval}[1]{\bigg[ #1 \bigg]}
\renewcommand{\epsilon}{\varepsilon}
\renewcommand{\iff}{\Leftrightarrow}

\DeclareMathOperator{\arccot}{arccot}
\DeclareMathOperator{\arcsec}{arcsec}
\DeclareMathOperator{\arccsc}{arccsc}


\colorlet{textColor}{black} 
\colorlet{background}{white}
\colorlet{penColor}{blue!50!black} % Color of a curve in a plot
\colorlet{penColor2}{red!50!black}% Color of a curve in a plot
\colorlet{penColor3}{red!50!blue} % Color of a curve in a plot
\colorlet{penColor4}{green!50!black} % Color of a curve in a plot
\colorlet{penColor5}{orange!80!black} % Color of a curve in a plot
                                      \colorlet{fill1}{blue!50!black!20} % Color of fill in a plot
\colorlet{fill2}{blue!10} % Color of fill in a plot
\colorlet{fillp}{fill1} % Color of positive area
\colorlet{filln}{red!50!black!20} % Color of negative area
\colorlet{gridColor}{gray!50} % Color of grid in a plot

\pgfmathdeclarefunction{gauss}{2}{% gives gaussian
  \pgfmathparse{1/(#2*sqrt(2*pi))*exp(-((x-#1)^2)/(2*#2^2))}%
}



\newcommand{\fullwidth}{}
\newcommand{\normalwidth}{}



%% makes a snazzy t-chart for evaluating functions
\newenvironment{tchart}{\rowcolors{2}{}{background!90!textColor}\array}{\endarray}

%%This is to help with formatting on future title pages.
\newenvironment{sectionOutcomes}{}{} 


\author{Gregory Hartman \and Matthew Carr}
\license{Creative Commons 3.0 By-NC}
\acknowledgement{https://github.com/APEXCalculus}

\begin{document}
\begin{exercise}

\outcome{Calculate limits using the limit laws.}

\tag{limit} 
\tag{quadratic} 
\tag{continuous}
  
  
  Find 
  \[
  \lim_{x\to 1}\p{x^2+3x-5}
  \begin{prompt}
  = \answer{-1}.
  \end{prompt}
  \]
  
   \begin{hint}
      This function is continuous everywhere. Therefore, left-hand and right-hand limits exist at every point and are equal. Use this to your advantage by applying limit laws. Namely, the limit of a sum is the sum of the limits. Hence, 
    \[
    \lim_{x\to 1} \left( x^2+3x-5 \right)  
    = \lim_{x\to 1} \left( x^2 \right) +
    \lim_{x\to 1} \left( 3x \right) +
    \lim_{x\to 1} \left( -5 \right).
    \]
    \end{hint}
    
     \begin{hint}
    Take a look at the graph of the function
    \begin{center}
     \begin{tikzpicture}
	\begin{axis}
	[ymin=-3,ymax=3, xmin=-2, xmax=2, axis lines=center,xlabel=$x$,ylabel=$y$,every axis y 
	label/.style={at=(current axis.above origin),anchor=south},every axis x label/.style={at=(current axis.right of origin),anchor=west},
	domain=-3:3,
	yticklabels={},
	ymajorgrids=true,
	grid = major
	]
	\addplot[very thick,smooth]
	{\x^2+3*\x-5};
	\end{axis}
       \end{tikzpicture}
      \end{center}
    Additional limit laws imply that
    \[
    \lim_{x\to 1} \left( x^2+3x-5 \right)  
    = \left( \lim_{x\to 1} \left( x \right) \right)^2 +
    3 \cdot \lim_{x\to 1} \left( x \right) +
    \lim_{x\to 1} \left( -5 \right).
    \]
    \end{hint}
    \begin{hint}
     Evaluating $\lim_{x\to 1} \left(x^2+3x-5\right)  
    = \p{\lim_{x\to 1} \left( x \right)}^2 +
    3\cdot\lim_{x\to 1} \left( x \right) -
    \lim_{x\to 1} \left(5\right)$
    we see the answer is $-1$, since, as we know, $\lim_{x\to1}\p{x}=1$.
    \end{hint}
    
\end{exercise}
\end{document}