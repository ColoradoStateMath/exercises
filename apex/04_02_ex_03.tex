\documentclass{ximera}

\usepackage{todonotes}

\usepackage{tkz-euclide}
\usetikzlibrary{backgrounds} %% for boxes around graphs
\usetikzlibrary{shapes,positioning}  %% Clouds and stars
\usetkzobj{all}
\usepackage[makeroom]{cancel} %% for strike outs
%\usepackage{mathtools} %% for pretty underbrace % Breaks Ximera
\usepackage{multicol}


\newcommand{\RR}{\mathbb R}
\renewcommand{\d}{\,d}
\newcommand{\dd}[2][]{\frac{d #1}{d #2}}
\renewcommand{\l}{\ell}
\newcommand{\ddx}{\frac{d}{dx}}
\newcommand{\zeroOverZero}{$\boldsymbol{\tfrac{0}{0}}$}
\newcommand{\numOverZero}{$\boldsymbol{\tfrac{\#}{0}}$}
\newcommand{\dfn}{\textbf}
\newcommand{\eval}[1]{\bigg[ #1 \bigg]}
\renewcommand{\epsilon}{\varepsilon}
\renewcommand{\iff}{\Leftrightarrow}

\DeclareMathOperator{\arccot}{arccot}
\DeclareMathOperator{\arcsec}{arcsec}
\DeclareMathOperator{\arccsc}{arccsc}


\colorlet{textColor}{black} 
\colorlet{background}{white}
\colorlet{penColor}{blue!50!black} % Color of a curve in a plot
\colorlet{penColor2}{red!50!black}% Color of a curve in a plot
\colorlet{penColor3}{red!50!blue} % Color of a curve in a plot
\colorlet{penColor4}{green!50!black} % Color of a curve in a plot
\colorlet{penColor5}{orange!80!black} % Color of a curve in a plot
                                      \colorlet{fill1}{blue!50!black!20} % Color of fill in a plot
\colorlet{fill2}{blue!10} % Color of fill in a plot
\colorlet{fillp}{fill1} % Color of positive area
\colorlet{filln}{red!50!black!20} % Color of negative area
\colorlet{gridColor}{gray!50} % Color of grid in a plot

\pgfmathdeclarefunction{gauss}{2}{% gives gaussian
  \pgfmathparse{1/(#2*sqrt(2*pi))*exp(-((x-#1)^2)/(2*#2^2))}%
}



\newcommand{\fullwidth}{}
\newcommand{\normalwidth}{}



%% makes a snazzy t-chart for evaluating functions
\newenvironment{tchart}{\rowcolors{2}{}{background!90!textColor}\array}{\endarray}

%%This is to help with formatting on future title pages.
\newenvironment{sectionOutcomes}{}{} 


\author{Gregory Hartman \and Matthew Carr}
\license{Creative Commons 3.0 By-NC}
\acknowledgement{https://github.com/APEXCalculus}

\begin{document}
\begin{exercise}

\outcome{Solve basic related rates word problems.}
\outcome{Identify word problems as related rates problems.}
\outcome{Understand the process of solving related rates problems.}

\tag{related rates}

Water flows onto a flat surface at a rate of $5$cm$^3$/s forming a circular puddle $10$mm deep. How fast is the radius growing when the radius is:
\begin{enumerate}
\item		$1$cm? \[\answer{\frac{5}{2\pi}}\,cm/s\]
\item		$10$cm? \[\answer{\frac{1}{4\pi}}\,cm/s\]
\item		$100$cm? \[\answer{\frac{1}{40\pi}}\,cm/s\]

\end{enumerate}

\begin{hint}
Recall that $1$ millimeter is $\frac{1}{10}$ centimeters, so $10$mm is $1$cm. Hence, the volume of the puddle as a function of time is $V(t)=\pi r^2$ where $r$ is the radius of the puddle as a function of time. What is the derivative of $V(t)$ with respect to $t$. What about the derivative of $r^2$ with respect to $t$, given that $r$ is a function of time?
\end{hint}
\begin{hint}
The derivative as a function of time is $5=\pi(2r\frac{dr}{dt})$ since $\frac{dV}{dt}=5$cm$^3$/s.
\end{hint}
\begin{hint}
Then $\frac{dr}{dt}=\frac{5}{2r\pi}$. For any value of $r$, we can find $\frac{dr}{dt}$ by plugging in that value.
\end{hint}
\end{exercise}
\end{document}